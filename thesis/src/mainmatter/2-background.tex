\chapter{Background}
\label{ch:background}

\blockquote{The focus of this chapter is to explain the concept of lexical polyfunctionality, present the various approaches which have been adopted towards it, consider some of their criticisms, and offer a more robust, functionally grounded analysis of lexical polyfunctionality instead. I first briefly describe how approaches which view lexical polyfunctionality as a kind of productive lexical flexibility developed as a response to weaknesses in traditional theories of parts of speech. I then survey the landmark studies and major themes in the previous research on polyfunctionality, along with criticisms of this research. Following that, I present the typological markedness theory of lexical categories, which states that lexical categories are epiphenomenal markedness patterns regarding how different semantic classes of words are used for different discourse functions. I conclude by offering an analysis of polyfunctionality which is in line with typological markedness theory.}

\section{Introduction: Approaches to lexical polyfunctionality}
\label{sec:2.1}

The field of linguistics as a whole and the subfield of typology in particular is undergoing a radical shift in how we understand lexical categories, along primarily two dimensions. The first dimension is our understanding of what lexical categories are a property \emph{of}. Early researchers viewed categories as universal properties of both language generally and specific languages. I call this the \dfn{universalist} position. After Boas, many researchers then came to view categories as language-specific, with patterned similarities across languages. I call this the \dfn{relativist} approach. Most recently, some researchers view categories as typological patterns rather than properties of any particular language. This is the \dfn{typological} position, and the one I adopt here.

The second dimension of historical change in linguistic theories of categories is in the \emph{nature} of the categories themselves. In the Classical tradition, categories were thought to be categorical and well-defined by a set of necessary and sufficient conditions in the tradition of Aristotle. After the cognitive turn in the 1960s and 1970s, many linguists came to view categories as prototypal, with some members of a category being more central, or better exemplars, than others. Cognitive research into the nature of idioms then led to the development of construction grammar, which sees language as consisting of a network of constructions rather than monolithic categories. I adopt a constructional approach to categories in this dissertation.

These theoretical paradigm shifts are summarized in \exref{ex:2.1}. At each stage of development, there has not been a wholesale displacement of previous theories. There are still many who regard word classes as universal and categorical, and the typological-constructional approach is still nascent.

\begin{exe}
  \ex\label{ex:2.1}
  \begin{xlist}
    \ex universal > language-specific > typological
    \ex categorical > prototypal > constructional
  \end{xlist}
\end{exe}

\secref*{sec:2.2} gives a synopsis of these theoretical positions and shows how research on lexical polyfunctionality developed in recognition of the shortcomings of traditional approaches. \secref*{sec:2.3} summarizes the key concepts and themes that have arisen from the research on lexical polyfunctionality. Such research, however, is not without its own shortcomings. \secref*{sec:2.3} also presents the main criticisms that have been leveled against analyses of polyfunctionality as lexical flexibility in particular. \secref*{sec:2.4} then presents an alternate, functionally-oriented approach—the typological-constructional perspective. The final section of this chapter (\secref*{sec:2.5}) then applies this functional perspective to formulate an improved understanding of lexical polyfunctionality.

\section{Traditional approaches}
\label{sec:2.2}

This section is a necessarily brief history of approaches to lexical categories up until the cognitive turn of the 1960s. It covers the universalist position that developed in the Classical tradition, the relativist position that developed as a result of Boas' cultural relativism, and the structuralist (or \enquote{distributionalist}) position that developed in the tradition of Saussure. Depending on how one understands and applies these different perspectives, none of them are mutually exclusive. It is especially common for linguists to simultaneously hold that lexical categories must be identified on the basis of language-internal evidence alone (the relativist position) and that lexical categories are universal in some sense or another (the universalist position).

\subsection{Universalism}
\label{sec:2.2.1}

Historically and still presently, many researchers assumed that a small set of lexical categories are basic and universal to all languages \parencites[81]{BolingerSears1981}[2]{Croft1991}[32]{Payne1997}[95]{Stassen2011}. The set typically consists of some variation of the following: Noun, Verb, Adjective, Adverb, Pronoun, Adposition, Conjunction, Numeral, and Interjection \parencite[16538]{Haspelmath2001}. This list has its origins in the \pubtitle{Τέχνη Γραμματική} / \pubtitle{Tékhnē Grammatikḗ} (\tln{The art of grammar}) of the \nth{2} century B.C.E. grammarian Dionysius Thrax. The \pubtitle{Tékhnē} synthesizes the work of Dionysius' predecessors, describing eight parts of speech for \idx{Classical Greek}. These parts of speech were based largely on morphological (especially inflectional) criteria \parencite[17--20]{Rauh2010}. The \pubtitle{Tékhnē} was then translated and its model applied to \idx{Latin} in the \pubtitle{Ars Grammatica} of Remnius Palaemon. The \pubtitle{Ars Grammatica} initiated a tradition wherein the languages of Europe and eventually the world \parentext{e.g. \idx{Mandarin} \parencite{McDonald2013}} were described using both Dionysius' categories (with occasionally additions / subtractions) as well as his method of identifying those categories on the basis of morphological criteria \parencite[20]{Rauh2010}. Because of the strong association of the term \dfn{parts of speech} with this Classical perspective, I prefer the term \dfn{lexical categories} in this dissertation.

Implicit in the Classical method is the assumption that lexical categories are universal in the sense of being instantiated in all languages. However, as European scholars began to encounter non-\idx{Indo-European} languages (or even non-\idx{Romance} languages) in both Europe and abroad, this assumption was challenged, as early as the first grammatical descriptions of \idx{Irish} in the \nth{7} century. At first, these languages either had Classical grammar imposed upon them or were deemed grammatically deficient \parencite[3]{Suarez1983}. Nonetheless, missionary linguists in the early colonial era were aware of the significant grammatical differences between these languages and \idx{Latin} and made their best attempts at describing them \parencite[3--4]{Suarez1983}. It is also important to realize that the project of describing the languages in the Americas and other zones of colonial influence was partially contemporaneous with the publication of the first grammars of the vernacular languages of Europe, as illustrated in \figref{fig:grammars}, the data for which are given in \tabref{tab:grammars}. Between 1524 and 1572, over 100 catechisms, manuals for confession, collections of sermons, grammars, and vocabularies were written in or about ten languages within the Viceroyalty of New Spain alone (an area smaller than present-day Mexico), mostly by Spanish Franciscan and Jesuit missionaries \parencite[2]{Suarez1983}. The task of converting the indigenous peoples to Christianity via the medium of their own languages was so important to the Spanish crown that the first bishop of Mexico, Francisco de Zumárraga, brought a printing press to Mexico in 1534 (just 15 years after the arrival of the first Spaniards in Mexico in 1519). The first book printed in Mexico was a \idx{Spanish}-\idx{Nahuatl} catechism by Alonso de Molina \parencite[2]{Suarez1983}. All this is merely to illustrate that language scholars in the colonial era were still in the early stages of discovering the complexities of the world's languages and how much they differed from Latin and Greek, and yet there has nonetheless been an awareness of the challenges that non-Indo-European languages pose to Classical theories of parts of speech from these early stages of language documentation and research.

\begin{figure}
  \includegraphics[width=\linewidth]{grammars.png}
  \caption{Timeline of early grammatical descriptions of European vs. American languages}
  \label{fig:grammars}
\end{figure}

\setlength\LTleft{0pt}
\setlength\LTright{0pt}
\renewcommand{\arraystretch}{2}
\singlespacing

\begin{longtable}[c]{ l l l l }%
  \caption{Some first grammatical descriptions of European vs. American languages}
  \label{tab:grammars}\\
  \toprule
    Language     & Year       & Title                                                                                                                                                                                         & Author\\
  \midrule
  \endfirsthead
  \caption[]{Some first grammatical descriptions of European vs. American languages}\\
  \toprule
    Language     & Year       & Title                                                                                                                                                                                         & Author\\
  \midrule
  \endhead
    \idx{Irish}        & 600s       & \parbox[t]{2.5in}{\pubtitle{Auraicept na n-Éces}\\\tln{The scholars' primer}}                                                                                                                 & Longarad\\
    \idx{Occitan}      & 1327       & \parbox[t]{2.5in}{\pubtitle{Leys d'amors}\\\tln{Laws of love}}                                                                                                                                & Guilhèm Molinièr\\
    \idx{Welsh}        & 1382--1410 & \parbox[t]{2.5in}{\pubtitle{Llyfr Coch Hergest}\\\tln{Red book of Hergest}}                                                                                                                   & unknown\\
    \idx{Tuscan}       & 1437--1441 & \parbox[t]{2.5in}{\pubtitle{Grammatica della lingua toscana}\\\tln{Grammar of the Tuscan language}}                                                                                           & Leon Battista Alberti\\
    Castilian\index{Spanish}    & 1492       & \parbox[t]{2.5in}{\pubtitle{Gramática de la lengua castellana}\\\tln{Grammar of the Castilian language}}                                                                                      & Antonio de Nebrija\\
    \idx{French}       & 1530       & \parbox[t]{2.5in}{\pubtitle{L'Éclaircissement de la langue francoyse}\\\tln{Explication of the French language}}                                                                              & John Palsgrave\\
    \idx{German}       & 1534       & \parbox[t]{2.5in}{\pubtitle{Ein Teutsche Grammatica}\\\tln{A German grammar}}                                                                                                                 & Valentin Ickelsamer\\
    \idx{Basque}       & 1545       & \parbox[t]{2.5in}{\pubtitle{Linguæ Vasconum Primitiæ}\\\tln{First fruits of the Basque language}}                                                                                             & Bernard Etxepare\\
    \idx{Totonac}      & 1539--1554 & \parbox[t]{2.5in}{\pubtitle{Arte de la lengua totonaca}\\\tln{Grammar of the Totonac language}}                                                                                               & Andrés de Olmos\\
    \idx{Nahuatl}      & 1547       & \parbox[t]{2.5in}{\pubtitle{Arte para aprender la lengua mexicana}\\\tln{Grammar for learning the Mexican language}}                                                                          & Andrés de Olmos\\
    \idx{Tarascan}     & 1558       & \parbox[t]{2.5in}{\pubtitle{Arte de la lengua tarasca de Michoacán}\\\tln{Grammar of the Tarascan language of Michoacán}}                                                                     & Maturino Gilberti\\
    \idx{Dutch}        & 1559       & \parbox[t]{2.5in}{\pubtitle{Den schat der Duytsscher Talen}\\\tln{The treasure of the Dutch language}}                                                                                        & John III van de Werve\\
    \idx{Quechua}      & 1560       & \parbox[t]{2.5in}{\pubtitle{Grammatica o arte de la lengua general de los Indios de los Reynos del Peru}\\\tln{Grammar or Art of the General Language of the Indians of the Royalty of Peru}} & Domingo de Santo Tomás\\
    \idx{Tzeltal Maya} & 1571       & \parbox[t]{2.5in}{\pubtitle{Ars Tzeldaica}\\\tln{Tzeltal Grammar}}                                                                                                                            & Fray Domingo de Hara\\
    \idx{Zapotec}      & 1578       & \parbox[t]{2.5in}{\pubtitle{Arte en lengua Zapoteca}\\\tln{Grammar in the Zapotec language}}                                                                                                  & Juan de Córdova\\
    \idx{English}      & 1586       & \parbox[t]{2.5in}{\pubtitle{Pamphlet for Grammar}}                                                                                                                                            & William Bullokar\\
    \idx{Mixtec}       & 1593       & \parbox[t]{2.5in}{\pubtitle{Arte de lengua Mixteca}\\\tln{Grammar of the Mixtec language}}                                                                                                    & Antonio de los Reyes\\
    \idx{Timucua}      & 1614       & \parbox[t]{2.5in}{\pubtitle{Gramatica de la lengua Timuquana de Florida}\\\tln{Grammar of the Timucua language of Florida}}                                                                   & Francisco Pareja\\
    \idx{Narragansett} & 1643       & \parbox[t]{2.5in}{\pubtitle{A key into the language of America}}                                                                                                                              & Roger Williams\\
  \bottomrule
\end{longtable}

\doublespacing
\renewcommand{\arraystretch}{1}

As documentary linguistics turned its attention to North American (as opposed to Mesoamerican) languages, lexical polyfunctionality in particular became a more prominent issue. In fact, even the first comprehensive survey of North American languages contains an entire section on \enquote{Conversion of nouns into verbs} \parencite[174--177]{Gallatin1836}, in which Gallatin depicts lexical polyfunctionality as a rampant feature of all languages on the continent:

\blockquote[{\cite[175--176]{Gallatin1836}}]{It is the substantive [i.e. copula / auxiliary] verb which we [speakers of Indo-European languages] conjugate; whilst the [Native American] conjugates what we call the adjective and even the noun itself, in the same manner as [s/he] does other intransitive verbs. […] I believe it must appear sufficiently obvious, that this general if not universal character of the [Native American] languages, the conversion into verbs and the conjugation, through all the persons, tense, and moods, of almost all the adjectives and of every noun which, without a palpable absurdity, is susceptible of it, is entirely due to the absence of the substantive verb.}

As evidenced by the above passage, increasing familiarity with non-\idx{Indo-European} languages prompted some writers to abandon the universalist commitment for word classes. However, categorial universalism is still a widely-held position today, either in the sense of a) categories being universally instantiated in all languages \parentext{commonly assumed by most generative frameworks; although see \textcite{Culicover1999}}, or b) categories being available to all languages, but only instantiated in some \parentext{sometimes called the \enquote{smörgåsbord} or \enquote{grab bag} approach, as exemplified by \possciteauthor{Dixon2010} Basic Linguistic Theory framework \parencites*[9, 11, 14, 27, 50]{Dixon2010}[26]{Dixon2011}; see \textcite[298]{Hieber2013} and \textcite[10]{Croft2001b} for discussion}.

\subsection{Relativism}
\label{sec:2.2.2}

American ethnographers in the tradition of Franz Boas questioned the universalist assumption in a programmatic and comprehensive way. Writing on grammatical categories, \citeauthor{Boas1911} states, \textquote[{\cite[35]{Boas1911}}]{Grammarians who have studied the languages of Europe and western Asia have developed a system of categories which we are inclined to look for in every language}. He concludes that this endeavor is a folly, and that \textquote[{\cite[35]{Boas1911}}]{in a discussion of the characteristics of various languages \qem{different fundamental categories} will be found}. Discussing lexical categories specifically, he notes the following:

\blockquote[{\cite[76]{Boas1911}}]{We might perhaps say that American languages have a strong tendency to draw the dividing line between denominating terms and predicative terms, not in the same way that we are accustomed to do. In American languages many of our predicative terms are closely related to nominal terms, most frequently the neutral verbs expressing a state, like \txn{to sit}, \txn{to stand}. These, also, often include a considerable number of adjectives.}

Boas' students all adopted his grammatical relativism, and it became a foundational principle of the American linguistics tradition. His student Edward Sapir, writing on lexical categories specifically, makes one of the best-known and strongest statements of this position in his influential textbook \pubtitle{Language}: \textquote[{\cite[125]{Sapir1921}}]{[N]o logical scheme of the parts of speech—their number, nature, and necessary confines—is of the slightest interest to the linguist. Each language has its own scheme. Everything depends on the formal demarcations which it recognizes.}.

Many linguists today hold to Boas' grammatical relativism in some fashion or another. Textbooks and typological surveys commonly state that languages have varying numbers of lexical categories, though usually with the caveat that all languages seem to differentiate at least noun and verb \parencite[e.g.][§6.2]{Velupillai2012}. Some researchers, especially those working in typology, argue that linguists are still not rigorous \emph{enough} in their application of grammatical relativism; they criticize certain kinds of crosslinguistic comparisons for imposing the categories of one language onto another \parencites{Croft2001b}{Gil2001}{Haspelmath2010a}{Haspelmath2012}{LaPolla2016}. This position is discussed further in \secref*{sec:2.4}.

\subsection{Structuralism}
\label{sec:2.2.3}

Developing alongside the early anthropological linguistics of Boas was the linguistic structuralism of Ferdinand de Saussure. His work informed both the Prague school under Nikolay Trubetzkoy and Roman Jakobson, and the distributional method of Leonard Bloomfield. The term \dfn{structuralism} has any number of uses \parencite[Ch.~1]{Matthews2001}; here I refer to the idea that \textquote[{\cite[383]{Matthews2014}}]{language is a […] self-contained, self-regulating system, whose elements are defined by their relationship to other elements}. In particular, I am referring to the positivistic flavor of structuralism as practiced by Bloomfield, which focused on the structural relations between elements, and on establishing a set of rigorous scientific discovery procedures for linguistic structures \parencite{Bloomfield1933}. Bloomfield saw lexical categories as something to be empirically discovered in the different syntactic distributions of words, rather than imposed on a language a priori \parencite[33]{Rauh2010}. Zellig Harris later refined and expanded this methodology \parencite{Harris1951}.

The signature methodological feature of this form of structuralism is the \dfn{distributional method}, a procedure for defining categories in terms of the set of contexts in which its words can appear—that is, their distributions \parencites[5]{Harris1951}[11]{Croft2001b}. As an illustration of distributional analysis applied to lexical categories, \textcite[11--12]{Croft1991} considers the distributions of the \idx{English} words \txn{cold}, \txn{happy}, \txn{dance}, and \txn{sing} in two constructions: in the Predicate construction after \txn{be}, and in the \nth{3} Person Singular Present Tense (\txn{-s}) construction. Example data are shown below.

\begin{exe}
  \ex\label{ex:2.2}
  \exinfo{\idx{English} (Indo-European > Germanic)}
  \begin{xlist}
    \ex
      \begin{xlist}
        \setlength{\itemsep}{0em}
        \ex[]{Jack is cold.}
        \ex[*]{Jack colds.}
      \end{xlist}
    \ex
      \begin{xlist}
        \setlength{\itemsep}{0em}
        \ex[]{Jack is happy.}
        \ex[*]{Jack happies.}
      \end{xlist}
    \ex
      \begin{xlist}
        \setlength{\itemsep}{0em}
        \ex[*]{Jack is dance.}
        \ex[]{Jack dances.}
      \end{xlist}
    \ex
      \begin{xlist}
        \setlength{\itemsep}{0em}
        \ex[*]{Jack is sing.}
        \ex[]{Jack sings.}
      \end{xlist}
  \end{xlist}
\end{exe}

\noindent We can see that \txn{cold} and \txn{happy} have the same distributions in these tests (both may appear in the Predicate construction but not the Person-Tense inflection construction), while \txn{dance} and \txn{sing} have the same distribution (the inverse situation as \txn{cold} and \txn{happy}). The results of these two distributional tests are summarized in \tabref{tab:English-distributions-a}. Data like these are used to justify categories like Adjective and Verb in English.

\begin{table}[h]
  \centering
  \caption[Distribution of \idx{English} Verbs and Adjectives]{Distribution of English Verbs and Adjectives \parentext{adapted from \textcite[12]{Croft2001b}}}
  \label{tab:English-distributions-a}
  \begin{tabular}{ l c c }
    \toprule
    { } & \parbox{1in}{\centering Predicate{\newline}Construction} & \parbox{1in}{\centering Inflectional{\newline}Construction}\\
    \midrule
    \textbf{Adjective}: \txn{cold}, \txn{happy}, etc. & ✔ & ✘\\
    \textbf{Verb}:      \txn{sing}, \txn{dance}, etc. & ✘ & ✔\\
    \bottomrule
  \end{tabular}
\end{table}

As applied in practice, however, the distributional method suffers from one serious drawback when used to argue for large, traditional categories like noun, verb, and adjective: distributional tests yield conflicting and overlapping results. Perhaps no two lexical items behave the same way in every distributional test. Each new test that is introduced therefore partitions the lexicon into smaller and smaller classes. This fact has been demonstrated empirically for \idx{English} temporal nouns \parencite[54]{Crystal1967}, \idx{Russian} numerals \parencite{Corbett1978}, and \idx{French} verbs \parencite{Gross1979}. Distributional tables from each of these studies are reproduced in \tabref{tab:English-distributions-b}, \tabref{tab:Russian-distributions}, and \tabref{tab:French-distributions} respectively. It is clear from these studies that distributional analysis does \emph{not} lead to large, unified categories like noun, verb, and adjective, but rather a myriad of small constructions \parencites[27]{Crystal1967}[434]{Croft2005}. Each distributional test is in fact its own construction \parencite[436]{Croft2005}. This fact is a major motivation underlying constructional approaches to language.

\begin{table}[h!]
  \includegraphics[width=\linewidth]{English-distributions.jpg}
  \caption[Distributional analysis of English (Indo-European > Germanic) temporal nouns]{distributional analysis of \idx{English} (Indo-European > Germanic) temporal nouns \parencite[54]{Crystal1967}}
  \label{tab:English-distributions-b}
\end{table}

\begin{table}[h!]
  \includegraphics[width=\linewidth]{Russian-distributions.jpg}
  \caption[Distributional analysis of Russian (Indo-European > Slavic) numerals]{distributional analysis of \idx{Russian} (Indo-European > Slavic) numerals \parencite[359]{Corbett1978}}
  \label{tab:Russian-distributions}
\end{table}

\begin{table}[h!]
  \includegraphics[width=\linewidth]{French-distributions.jpg}
  \caption[Distributional analysis of French (Indo-European > Romance) verbs]{distributional analysis of \idx{French} (Indo-European > Romance) verbs \parencite[860]{Gross1979}}
  \label{tab:French-distributions}
\end{table}

Many scholars nonetheless choose to retain lexical categories as a necessary component of their linguistic theories or descriptions, at the expense of consistent application of the distributional method. Rather than considering all possible distributional contexts for a word, these scholars instead treat certain constructions as definitional of the category. Other distributional tests which yield cross-cutting results are either ignored or treated as evidence of subcategories instead of categories. Many researchers even prefer the term \dfn{syntactic categories} over \dfn{lexical categories} for this reason, focusing on just the syntactic evidence for categories \parencites{Baker2003}{Rauh2010}. A severe methodological problem for this approach is that there are no generally agreed-upon principles for determining which distributional tests should be considered definitional. In this regard, \textcite[4]{SchachterShopen2007} note, \textquote{there may be considerable arbitrariness in the identification of distinct parts of speech rather than subclasses} \parentext{see also \textcite{Crystal1967}}. Different scholars choose or prioritize different kinds of evidence for lexical categories over others based on their theoretical commitments. This is the reason, as stated in \secref*{sec:1.1}, that disagreements about the existence of particular lexical categories in particular languages are typically \emph{not} about the empirical facts. The results of a given distributional analysis are not usually controversial \parentext{though see \textcite{Aarts2007}}; the choice of distributional tests used to support one's analysis is. Unsurprisingly, then, debates over how to analyze lexical categories in various languages have been largely unproductive and unresolved \parencite[435]{Croft2005}. The problem only worsens when scholars attempt to apply the same criteria across languages. Distributions of lexical items with similar meanings vary drastically across languages \parencite[§1.4.1]{Croft2001b}.

The real methodological problem here is \emph{not} that we have yet to ascertain the correct principles for selecting the right distributional tests. The problem is being selective regarding which tests to apply in the first place. If we take the distributional method seriously, then we must apply it consistently, without ignoring distributional evidence that contradicts our theoretical or pretheoretical assumptions. To do otherwise is a kind of \dfn{methodological opportunism} \parencite[30, 41]{Croft2001b}.

Other scholars treat polyfunctional items as members of \dfn{hybrid} or \dfn{mixed} categories simultaneously possessing properties of more than one part of speech \parencites[149]{Loisetal2017}{Malouf1999}{NikolaevaSpencer2020}. Adjectives are frequently described as a hybrid category \parencites{Wetzer1996}[343]{Stassen1997}[13--16]{Pustet2003}[95]{GenettiHildebrandt2004}{Lier2017}, as are participles \parencite[704]{HopperThompson1984} and gerunds \parencite{Denison2001}. \textcite[149]{Loisetal2017} also distinguish hybridity from polycategoriality, stating that polycategoriality applies to roots or stems, while hybridity is a matter of the syntactic context that a word appears in.

An analysis couched in mixed categories does not avoid the problem of methodological opportunism, however. The existence of a mixed category implies that there are other, more basic categories that the mixed category is a hybrid of. Hybrid models of parts of speech merely exacerbate the distributional problem. There is however a sense in which thinking of minor lexical categories as \enquote{mixed} categories is useful: typological markedness theory states that lexical categories are epiphenomenal patterns that arise from combinations of the semantic classes of object, action, or property words with the discourse functions of reference, predication, and modification. Categories frequently discussed as \enquote{mixed} in the literature are precisely those combinations which are non-prototypical and therefore more likely to be typologically marked. \secref*{sec:2.4.2} explains this approach to lexical categories in more detail.

Partly in response to these problems, a growing cadre of linguists in the last 30 years have adopted one of various \dfn{flexible} approaches to word classes. Flexible analyses of word classes come in many flavors, some of which arguably still commit methodological opportunism, and others of which introduce new difficulties. These flexible approaches are reviewed in the following section.

\section{Flexible approaches}
\label{sec:2.3}

In this section I summarize the key concepts (\secref{sec:2.3.1}), research themes (\secref{sec:2.3.2}), and criticisms (\secref{sec:2.3.2}) of research on lexical polyfunctionality. \secref*{sec:2.3.1} surveys the wide variety of definitions and theoretical perspectives on lexical polyfunctionality. This review of the literature reveals that there is little consensus as to what exactly constitutes lexical polyfunctionality; as such, there are numerous alternative terms for the phenomenon. Despite these incongruities, a few major themes do consistently surface across different empirical studies. These are summarized in \secref*{sec:2.3.2}. One tradition in particular views lexical polyfunctionality as productive, adopting the term \dfn{lexical flexibility}. \secref*{sec:2.3.3} looks at criticisms of such flexible analyses.

\subsection{Key concepts}
\label{sec:2.3.1}

It is only a small exaggeration to say that there are as many definitions and terms for what I am here calling \enquote{lexical polyfunctionality} as there are scholars who research it. The analytical or theoretical perspective adopted by each researcher generally determines their choice of terminology. The remainder of this section is devoted to explaining these perspectives in more detail.

Generally speaking, there are two ways researchers have analyzed polyfunctional items. The first method assigns flexible items to members of specific categories in a language, whether those categories are the canonical four major classes (Noun, Verb, Adjective, Adverb), or a new large supercategory subsuming multiple discourse functions (e.g. Contentives, Non-Verbs, Flexibles), or a smaller subcategory of an existing major lexical category (e.g. Adjectival Verbs, Verbonominals). The second method of analysis assumes that lexical items are uncategorized at some level (root, stem, or inflected word), and that items receive their categorial assignment from context. Different researchers posit different mechanisms for how lexical items receive their categorization in context.

The traditional approaches to lexical polyfunctionality summarized in \secref*{sec:2.2} are all instances of the former method of analysis, while the flexible approaches outlined in this section are a mix of categorial and acategorial analyses.

\subsubsection{Lexical flexibility}
\label{sec:2.3.1.1}

The term \dfn{lexical flexibility} is typically taken to imply that lexical items can be used productively across different lexical categories—in other words, that these different uses are all part of one unified, polysemous lexeme, and that the category information is then inferred from context. The term \dfn{lexical flexibility} itself seems to have originated with \textcite[Ch.~4]{Hengeveld1992}. This publication, perhaps because it was the first to assign a technical term to the concept, marks a shift in how scholars frame the concept of lexical polyfunctionality. Previously, the issue was framed in terms of whether particular languages (especially those of the Pacific Northwest) distinguished noun from verb \parencites{Kuipers1968}{Jacobsen1979}{Hebert1983}{Kinkade1983}{EijkHess1986}{JelinekDemers1994}. After this point, an increasing number of publications began to ask whether lexemes were \dfn{flexible} instead. Though the difference in emphasis seems subtle, this change constitutes a turning point because it fostered an increased interest in lexical polyfunctionality as a grammatical phenomenon in its own right instead of just a problem for traditional categorization schemes.

\posscitet[Ch.~4]{Hengeveld1992} typology of parts-of-speech systems is a whole-language typology wherein languages are either \dfn{specialized}, with one morphosyntactic category for each of the functions of reference (\enquote{Noun}), predication (\enquote{Verb}), referent modification (\enquote{Adjective}), and predicate modification (\enquote{Adverb}\footnote{Note that Hengeveld's typology only includes manner adverbs, not other semantic types of adverbs.}), or \dfn{non-specialized}. Non-specialized languages deviate from the four-category canon in one of two ways: one part of speech may assume more than one function with no additional morphosyntactic marking, in which case the language is considered \dfn{flexible}; or the language may lack a dedicated part of speech for that function entirely and use other, marked constructions instead, in which case the language is considered \dfn{rigid}.

\citeauthor{Hengeveld1992} gives examples from \idx{Dutch} and \idx{Wambon} to illustrate the distinction between rigid and flexible languages. In the Dutch examples in \exref{ex:2.3}, the same word \txn{mooi} is used for both referent modification \exref{ex:2.3a} and predicate modification \exref{ex:2.3b}, with no function-indicating morphology in either case. Wambon on the other hand uses medial verbs for manner expressions and must take the overt verbalizing suffix \txn{-mo} shown in \exref{ex:2.4}. In \possciteauthor{Hengeveld1992} framework, Dutch is a flexible language because one category subsumes both the functions of referent modification and predicate modification, while Wambon is a rigid language because derivational morphology (here, the verbalizing suffix \txn{-mo}) is required to indicate the function of predicate modification.

\begin{exe}

  \ex\label{ex:2.3}
  \exinfo{\idx{Dutch} (Indo-European > Germanic)}
  \begin{xlist}

    \ex\label{ex:2.3a}
    \gll een        \em{mooi}      kind\\
         \gl{indef} \em{beautiful} child\\
    \tln{a beautiful child}
    \exsource[65]{Hengeveld1992}

    \ex\label{ex:2.3b}
    \gll het      kind  dans‑t              \em{mooi}\\
         \gl{def} child dance‑\gl{3sg.pres} \em{beautifully}\\
    \tln{the child dances beautifully}
    \exsource[65]{Hengeveld1992}

  \end{xlist}

  \ex\label{ex:2.4}
  \exinfo{\idx{Wambon} (Trans-New Guinea > Greater Awyu)}
  \vfix
  \gll jakhov‑e       \em{matet‑mo}         ka-lembo\\
       they‑\gl{conn} \em{good‑\gl{vzr.ss}} go‑\gl{3pl.past}\\
  \vfix
  \tln{did they travel well?}
  \hfill\hspace*{1em}\mbox{\footnotesize\parentext{\textcite[49]{Vries1989}, cited in \textcite[65]{Hengeveld1992}}\normalsize}

\end{exe}

\possciteauthor{Hengeveld1992} analysis is of the categorial type discussed at the beginning of \secref*{sec:2.3.1}, specifically the supercategory kind. Each lexeme is assumed to have a category, and new supercategories are introduced for lexemes which have multiple functions: \dfn{Contentives} for lexemes which perform all four functions, \dfn{Non-Verbs} for lexemes which perform all non-predicating functions, and \dfn{Modifiers} for lexemes which perform referent modifier and predicate modifier functions.

\possciteauthor{Hengeveld1992} parts-of-speech typology and the subsequent research it inspired \parencites{DonLier2013}{HengeveldRijkhoff2005}{Lier2006}{HengeveldLier2012}{Luuk2010}{LierRijkhoff2013}{Lier2016} constitute important empirical contributions to the study of lexical polyfunctionality. However, Hengeveld's definition of flexible languages and his parts-of-speech typology still rely on large, language-specific categories of the kind that have been problematized by \textcite[§2.2.2]{Croft2001b} and \textcite{CroftLier2012}, and are therefore subject to the same difficulties as traditional approaches to parts of speech. However, numerous scholars have since adopted Hengeveld's term \dfn{lexical flexibility} to describe cases where lexical items serve more than one discourse function, regardless of their theoretical commitments or analysis of flexible items. As a convenient cover term, \dfn{lexical flexibility} is now well established. Nonetheless, I retain the use of \txn{lexical polyfunctionality} here for its precision, and since I do not adopt a flexible analysis to polyfunctional forms.

\subsubsection{Polycategoriality}
\label{sec:2.3.1.2}

\textcite[4]{VapnarskyVeneziano2017a} introduce the alternative term \dfn{polycategoriality} as their preferred characterization of polyfunctional items. \parentext{The term is also used by \textcite{Carter2006}, but he does not give a precise definition for it.} While \citeauthor{VapnarskyVeneziano2017a} use this term mostly interchangeably with \dfn{lexical flexibility}, there are important differences between the two concepts. \possciteauthor{Hengeveld1992} use of \dfn{lexical flexibility} is meant to imply the existence of large, flexible supercategories that subsume multiple discourse functions, whereas Vapnarsky \& Veneziano are not committed to any particular schema for parts of speech. Central to their notion of polycategoriality is the idea that lexical categories exist, but that \textquote[{\cite[4]{VapnarskyVeneziano2017a}}]{there are lexical forms that are not specified for lexical category (or are not specified fully, or univocally) on some level of representation.}. In other words, one lexeme may belong simultaneously to multiple lexical categories. Under this definition, a language could still have all four major lexical categories but nonetheless exhibit rampant polycategoriality; this is not a possibility in \possciteauthor{Hengeveld1992} framework. Like \citeauthor{Hengeveld1992}, however, \citeauthor{VapnarskyVeneziano2017a} are committed to the existence of large lexical categories in particular languages. Their analysis is therefore also of the categorial kind discussed at the beginning of \secref*{sec:2.3.1}.

\subsubsection{Multifunctionality}
\label{sec:2.3.1.3}

Another term for our phenomenon of interest, introduced by \parencite{Lier2012}, is \dfn{multifunctionality}, in which a single lexical item can have multiple discourse functions. An advantage of this analysis is that it takes no theoretical position on the issue of whether lexical items are categorial or acategorial; it just focuses on their functions. The term \dfn{multifunctionality} is meant to stand in contrast with \dfn{conversion} or \dfn{zero derivation}. \Citeauthor{Lier2012} takes conversion to be idiosyncratic and unproductive, producing meanings for forms in alternate discourse functions that are not predictable (see \secref*{sec:2.3.2.4} and \secref*{sec:2.3.3.2} for further discussion). Multifunctionality is also distinct from zero derivation from a common root. Instead, multifunctional lexemes are those whose semantic interpretation is entirely predictable from context, and whose uses in different contexts are productive. Their meanings should be \dfn{compositional}. For example: when an action word is used in a referring construction its predicted meaning is that of an \dfn{action nominalization}, \tln{(the act of) X‑ing}; and when an object word is used in a predicate construction its predicted meaning is that of a \dfn{predicate nominal}, \tln{be an X}. Examples of these predictable, compositional meanings for polyfunctional items in \idx{Chamorro} are shown in \exref{ex:2.5}.

\begin{exe}

  \ex\label{ex:2.5}
  \exinfo{\idx{Chamorro} (Austronesian > Malayo-Polynesian)}
  \begin{xlist}

    \ex\label{ex:2.5a}
    \gll para     \em{batångga‑n}     karabåo esti\\
         \gl{fut} \em{shed‑\gl{link}} carabao this\\
    \tln{this is going to be a carabao shed}
    \exsource[8]{Chung2012}

    \ex\label{ex:2.5b}
    \gll para     \em{gatbesa}    ha'\\
         \gl{fut} \em{decoration} \gl{emph}\\
    \tln{[she] is going to be a decoration}
    \exsource[20]{Chung2012}

  \end{xlist}

\end{exe}

\noindent In the two examples above, the meaning of the object words \tln{shed} and \tln{decoration} are predictable when used in a predicative context: \tln{be a shed/decoration}. However, lexical items used in their non-prototypical functions very frequently do not have predictable meanings. Consider the example in \exref{ex:2.6}.

\begin{exe}
  \ex\label{ex:2.6}
  \exinfo{\idx{Chamorro} (Austronesian > Malayo-Polynesian)}
  \vfix
  \gll ma       \em{se'si'} i   babui\\
       \gl{agr} \em{knife}  the pig\\
  \vfix
  \tln{they stabbed the pig}
  \exsource[29]{Chung2012}
\end{exe}

\noindent In this example, the meaning \tln{stab} cannot be predicted from the meaning of the object word \tln{knife}. It could have just as easily meant \tln{be a knife} or \tln{cut}.

\Citeauthor{Lier2012} takes examples like those in \exref{ex:2.5} to be instances of genuine multifunctionality, and those in \exref{ex:2.6} to be cases of conversion. Others have also adopted a position similar to \possciteauthor{Lier2012}, in which only the semantically compositional / predictable uses of a lexical item in different discourse functions are considered flexible \parencites[§2.2.2--§2.2.3]{Croft2001b}[§3.2]{EvansOsada2005}.

\subsubsection{Precategoriality / Acategoriality}
\label{sec:2.3.1.4}

The various approaches which analyze polyfunctional items as being at some level uncategorized until they receive their interpretation from context may be lumped together under the umbrella terms \dfn{precategoriality} or \dfn{acategoriality}. \possciteauthor{HopperThompson1984} influential \parencite*{HopperThompson1984} paper is an early application of the concept of acategoriality to the analysis of polyfunctional items:

\blockquote[{\cite[747]{HopperThompson1984}}]{[L]inguistic forms are in principle to be considered as \emph{lacking categoriality} completely unless nounhood or verbhood is forced on them by their discourse functions. To the extent that forms can be said to have an apriori existence outside of discourse, they are characterizable as \dfn{acategorial}; i.e. their categorical classification is irrelevant. Categoriality-the realization of a form as either a N or a V-is imposed on the form by discourse.}

The term \dfn{precategorial} has become a somewhat more common term for roughly the same concept (though some researchers use the term in more strictly-delineated ways) \parencites[357, 362--364]{EvansOsada2005}{Bisang2008}{Bisang2013}. It is especially preferred by morphological models that presuppose stages of derivation, such that lexical items are precategorial before they reach a certain stage of the derivation \parencites{HalleMarantz1994}{Arad2005}{McGinnisArchibald2016}{Siddiqi2018}. \textcite[5]{VapnarskyVeneziano2017a} distinguish polycategoriality from acategoriality by defining acategoriality as implying \textquote{no primitive / original categorial marking at all}, and polycategoriality as allowing a lexical item \textquote{to be only partially unspecified for category, with possible constraints on the relevant categories}. Languages for which precategorial analyses have been advanced include \idx{Cherokee} \parencite{Haag2017}, \idx{Gooniyandi} \parencite{McGregor2013}, \idx{Kuikuro} \parencite{FranchettoSantos2017}, Mundari \parencite{HengeveldRijkhoff2005}. \textcite{Pfeiler2017} also presents psycholinguistic evidence that the earliest utterances of L1 learners of \idx{Yucatec Maya} are acategorial.

A central concern in precategorial approaches is the precise mechanism by which a lexical item receives its categorization in context \parencite[§3.7]{HengeveldRijkhoffSiewierska2004}. There are two main theories of semantic indeterminacy in flexible items: \dfn{underspecificity} \parencites{Farrell2001}{RijkhoffLier2013} and \dfn{vagueness} \parencites{Tuggy1993}{HengeveldRijkhoffSiewierska2004}{HengeveldRijkhoff2005}. The essential difference is that underspecificity entails semantic minimalism, while vagueness entails semantic maximalism. An underspecified lexeme has a minimal, core meaning, and receives its categorial meaning from the discourse context it appears in; a vague lexeme has a maximal, broad meaning that covers all the possible discourse contexts it appears in. (There is of course quite a deal of variation in the literature as to how scholars use these terms, with many researchers conflating the two.) \textcite[414]{HengeveldRijkhoff2005} offer the example of \idx{English} \txn{cousin} as a word that is semantically underspecified for gender, such that the gender of the referent must be understood from context. \textcite{Denison2018} argues that the English word \txn{long} exhibits adjective \~{} adverb underspecification in \idx{Old English} and \idx{Middle English}.

In contrast, \textcite[539--541]{HengeveldRijkhoffSiewierska2004} outline a theory regarding exactly how vagueness operates in the context of precategoriality:

\blockquote[{\cite[541]{HengeveldRijkhoff2005}}]{[E]ach flexible lexeme has a single (vague) sense. By placing the flexible lexeme in a particular syntactic slot or by providing it with certain morphological markers, the speaker highlights those meaning components of the flexible lexeme that are relevant for a certain lexical (verbal, nominal, etc.) function. Thus we contend that the meaning of a flexible lexeme always remains the same, and that morphosyntactic and other contextual clues signal to the addressee how to interpret this lexeme in an actual utterance. In other words, it is the use of a vague lexeme in a certain context (an actual linguistic expression) that brings out certain parts of its meaning, giving the category-neutral lexeme a particular categorial (verbal, nominal, etc.) flavour.}

\noindent (Note that \citeauthor{HengeveldRijkhoff2005} distinguish \dfn{vagueness} and \dfn{ambiguity} by reserving the term \dfn{ambiguity} for cases of distinct, homophonous lexemes.)

\textcite[363--364]{EvansOsada2005} and \textcite{Kihm2017} criticize both precategorial approaches for their imprecision, claiming that it would be impossible to formulate a definition for many precategorial items that is broad enough to encompass all their uses. \textcite[87]{Kihm2017} illustrates this difficulty with the various meanings of the \idx{Arabic} root \txn{s-q-ṭ}, which could arguably be glossed \textsc{fall}. A selection of stems containing this root are given in \exref{ex:2.7}.

\begin{exe}
  \ex\label{ex:2.7}
  \hspace{0.5em}\exinfo{\idx{Standard Arabic} (Afroasiatic > Semitic)}\\
  \begin{tabular}[t]{ p{0.75in} l }
    saqaṭa   & \tln{to fall}\\
    saqiiṭ   & \tln{hail}\\
    saqqaaṭa & \tln{door latch}\\
    masqaṭ   & \tln{place where a falling object lands; waterfall}\\
    isqaaṭ   & \tln{overthrow; shooting down; miscarriage; substraction}\\
    tasaaquṭ & \tln{fall of hair}\\
    saaqiṭa  & \tln{fallen woman; harlot}\\
    suquuṭ   & \tln{fall; crash; collapse}\\
    saqṭ     & \tln{dew}\\
    siqṭ     & \tln{miscarried fetus}\\
    suqṭ     & \tln{sparks flying from a flint}\\
    saqaṭ    & \tln{offal; rubbish}\\
    saqṭa    & \tln{tumble; slip; mistake}\\
    saaqiṭ   & \tln{fallen; mean; missing}\\
  \end{tabular}
\end{exe}

It is difficult to imagine a single definition of \txn{s-q-ṭ} which could adequately demarcate just this set of meanings.

\subsubsection{Monocategoriality}
\label{sec:2.3.1.5}

In the extreme case where all lexical items in a language are precategorial, the language could be considered \dfn{monocategorial}, possessing a single, open syntactic category. This is effectively the same as saying that the language lacks lexical categories altogether, the difference being primarily one of emphasis. David Gil analyzes both \idx{Tagalog} \parencite*{Gil1995} and Riau \idx{Indonesian} \parencite*{Gil1994} as being of this extreme monocategorial type. Moreover, he argues that monocategoriality must have been typical of an earlier stage of language evolution in which dedicated morphological strategies for different discourse functions had yet to evolve \parencites{Gil2005}{Gil2006}{Gil2012}. He names this abstract language type an \dfn{isolating-monocategorial-associational} (IMA) language.

\subsubsection{Transcategoriality}
\label{sec:2.3.1.6}

It is also worth briefly mentioning \dfn{transcategoriality}, since the term arises occasionally in connection with lexical polyfunctionality and is potentially easily confused with other terms mentioned above. \textcite{Robert2003} uses \dfn{transcategoriality} to describe the ability of a single form to serve both lexical and grammatical functions. This is common in grammaticalization scenarios in which the original, lexical use of a form continues to exist alongside its newer, functional use. This is commonly referred to in the grammaticalization literature as \dfn{divergence} \parencite[118]{HopperTraugott2003}. Since the focus of lexical polyfunctionality is on \emph{lexical} items and categories rather than \emph{functional} ones, the concept of transcategoriality is not directly relevant to the study of lexical polyfunctionality.

\subsubsection{Conversion / Zero derivation}
\label{sec:2.3.1.7}

\dfn{Conversion} is the process whereby a lexical item simply changes its word class with no overt morphological marker of that change \parencite[114]{Crystal2008}. \dfn{Zero derivation} is an alternate analysis of the same phenomenon that posits the presence of a derivational marker with no phonological realization. Since the literature on conversion and zero derivation is extensive and the concepts are well-established, I will treat them only summarily here, focusing on their relationship to lexical polyfunctionality.

The concept of conversion is based on the premise that lexical items in a language are fully categorized for part of speech, meaning that an analysis of lexical polyfunctionality as conversion falls under the categorial (as opposed to acategorial) analyses  mentioned at the beginning of \secref*{sec:2.3.1}. Conversion is generally characterized as a kind of word formation, implying that a new lexeme has been created. Therefore, conversion and lexical flexibility are mutually exclusive analyses of multifunctional items: lexical flexibility implies the existence of one polysemous lexeme which can fulfill multiple discourse functions, while conversion implies the existence of two homonymous / heterosemous lexemes with different discourse functions. Remember too from \secref*{sec:2.3.1.3} that \textcite{Lier2012} distinguishes conversion from multifunctionality, where conversion is reserved for unproductive / unpredictable derivations. Not all scholars would delimit conversion in this way, however.

Conversion also implies directionality. In cases of conversion, one of the two uses of a form is in some way basic or prior to the other \parencites[156]{Mithun2017}[5]{VapnarskyVeneziano2017a}. Under a flexible analysis, by contrast, the different functions of a single flexible item have equal theoretical status. If it could be shown that certain putatively flexible uses of a lexical item were in some way marked in relation to each other, this would therefore constitute potentially disconfirming evidence against a flexible analysis. This is in fact one of the major arguments presented against flexible analyses, to be discussed in \secref*{sec:2.3.3}. There are at least four ways in which one member of a putatively flexible set of polyfunctional items might be considered more basic than the others: 1) diachronically, in which one use of the lexical item appears before the others historically; 2) semantically, in which the meaning of the derived item is more semantically complex than that of the basic one; 3) morphologically, in which the more basic item is irregularly inflected but the derived item is regularly inflected; or 4) frequentially, in which derived lexical items are used less frequently than their more basic counterpart \parencite[108--111]{Plag2003}. Speakers themselves also have intuitions about which member of a polyfunctional set is basic and which are derived \parencite[166]{Mithun2017}. As will be explained in \secref*{sec:2.4.2}, the idea that certain uses of a lexical item are marked in relation to each other is also central to the typological markedness theory of lexical categories.

\subsubsection{Functional shift / Functional expansion}
\label{sec:2.3.1.8}

Especially among researchers in North America, another common term for conversion is \dfn{functional shift} \parencite{Cannon1985}. In most research, the term is used essentially interchangeably with \dfn{conversion} or \dfn{zero derivation}. However, functional shift can be usefully distinguished from conversion by its emphasis on function over category, paralleling the distinction between polycategoriality (implying language-specific categories) and polyfunctionality (with no such implication). In its literal interpretation, the term suggests a shift in the meaning of a lexical item from one discourse function to another, an analysis amenable to a constructional approach, and one that is not committed to the existence of language-particular categories. A slight improvement on this term would be \dfn{functional expansion}, since it emphasizes the expansion of a linguistic form into new functions / contexts as opposed to the wholesale shift from one function to another implied by \dfn{functional shift}. I adopt the term \txn{functional expansion} in this dissertation, and discuss the concept in detail in \secref*{sec:2.5.2}.

\subsection{Themes in previous research}
\label{sec:2.3.2}

The emergence of lexical polyfunctionality as an object of study has yielded a number of edited collections or journal volumes (\cite{VogelComrie2000}, \cite{EvansOsada2005} (target article), \cite{AnsaldoDonPfau2010}, \cite{LoisVapnarsky2003}, \cite{RijkhoffLier2013}, \cite{SimoneMasini2014}, \cite{BlaszczakKlimekJankowskaMigdalski2015}, \cite{VapnarskyVeneziano2017b}, \cite{Lier2017} (target article), \cite{VapnarskyVeneziano2017a}, \cite{CuyckensHeyvaertHartmann2019}), plus any number of individual articles \parentext{see especially \textcite{Farrell2001}, \textcite{Rijkhoff2007}, \textcite{Lier2012}, and \textcite{Mithun2019}}. Out of these collections have emerged several recurring themes, each of which is summarized in this section.

It should be noted at the outset that many of these findings cannot be straightforwardly reinterpreted in the typological-constructional approach adopted here. In particular, much previous research relies heavily on large, language-specific categories, which I problematize in \secref*{sec:2.2.3} above. I discuss some of the other criticisms that have been leveled against these approaches in \secref*{sec:2.3.3} below. My use of terms like \txn{Noun}, \txn{Verb}, and \txn{Adjective} in this section should therefore not be taken as an endorsement of large, language-particular word classes, but instead as arising from a desire to accurately represent the perspectives of other researchers.

\subsubsection{Parts-of-speech hierarchy}
\label{sec:2.3.2.1}

In addition to laying out a theory of flexible categories, \textcite{Hengeveld1992} presents the results of a 30-language survey of parts of speech in which he finds that the categories which are most likely to occur as an independent class in a language are subject to an implicational hierarchy, shown in \exref{ex:2.8}, which Hengeveld refers to as the \dfn{parts-of-speech hierarchy}.

\begin{exe}
  \ex\label{ex:2.8} Verb > Noun > Adjective > Adverb
\end{exe}

\noindent Categories to the left of the hierarchy are more likely to occur as a distinct part of speech than categories to the right. Applying this hierarchy to \possciteauthor{Hengeveld1992} flexible vs. rigid distinction yields the parts-of-speech typology in \figref{fig:Hengeveld-pos-systems} \parentext{adapted from \textcite[69]{Hengeveld1992} and \textcite[718]{Rijkhoff2007}}. The terms for the different categories in flexible languages are from \textcite{HengeveldRijkhoffSiewierska2004}. \citeauthor{Hengeveld1992} points out that this is not a strict classification scheme; languages may sit at the boundaries between types and exhibit exceptions.

\begin{figure}[h]
  \centering
  \includegraphics[width=\linewidth]{pos-typology.png}
  \caption[Hengeveld's (1992) typology of parts-of-speech systems]{\possciteauthor{Hengeveld1992} \parencite*[69]{Hengeveld1992} typology of parts-of-speech systems}
  \label{fig:Hengeveld-pos-systems}
\end{figure}

As mentioned in \secref*{sec:2.3.1.1}, \possciteauthor{Hengeveld1992} typology could be criticized for its reliance on large, language-specific lexical categories instead of constructions. One could however reframe \possciteauthor{Hengeveld1992} implicational hierarchy in terms of functions rather than categories, as in \exref{ex:2.9}. I call this the \dfn{hierarchy of discourse functions}.

\begin{exe}
  \ex\label{ex:2.9} predicate > referent > predicate modifier > referent modifier
\end{exe}

\noindent In \exref{ex:2.9}, functions to the left of the hierarchy are more likely to have dedicated morphological strategies for expression than those to the right. This reformulation avoids a commitment to any language-particular categories while still capturing the implicational trend observed by Hengeveld.

This hierarchy of discourse functions has proven to be a fairly robust finding in the literature on lexical polyfunctionality, now supported by a number of subsequent studies \parencites{Anward2000}{Rijkhoff2000}{Vogel2000}{Beck2002}{Rijkhoff2002}{Rijkhoff2003}{HengeveldRijkhoffSiewierska2004}{Lier2006}{Hengeveld2007}{HengeveldLier2012}{HengeveldValstar2010}{Beck2013}{Bisang2013}{Hengeveld2013}.

\subsubsection{Reference-predication asymmetries}
\label{sec:2.3.2.2}

The hierarchy of discourse functions also hints at another important feature of lexical categories: there is something privileged about the predicating function. A survey of the literature on lexical polyfunctionality reveals patterned asymmetries in the behavior of lexical items with regard to predication vs. reference, even in highly polyfunctional. For starters, while it is quite common for languages to freely allow object words to be used as predicates with zero coding, the reverse case is much less likely \parencite[745]{HopperThompson1984}. The functional expansion of an item's uses from predication into reference always seems to be more marked (or at least as marked) as the shift from a referring function to a predicating function.

This fact has been observed independently by numerous researchers. For example, \textcite[251]{Stevick1968} and \textcite[373--374]{Marchand1969} both observe that conversion from noun to verb in \idx{English} has always been more common than from verb to noun, and \textcite[98]{Kastovsky1996} points out that English does not even have a native noun > verb derivational suffix—any affixes of this type are borrowed from \idx{Romance} languages. \idx{Central Alaskan Yup'ik} is another example of a language with very many nominalizers but few verbalizers \parencite[158]{Mithun2017}.

Polyfunctionality itself is frequently \dfn{unidirectional}, meaning that any object word may be used for predication, but that action words used for reference are marked \parencites[69]{Croft2001b}[§3.3]{EvansOsada2005}{Beck2013}. \textcite[44]{Nakayama2001} frames polyfunctionality in \idx{Nuuchahnulth} in terms of a stem's ability to predicate, reporting that \textquote{all inflectional stems are potentially predicative}, but the reverse is not true. Discussing \idx{Classical Nahuatl}, Launey \parencites*{Launey1994}{Launey2004} introduces the term \dfn{omnipredicativity} to describe languages in which all lexical items are potentially predicative. However, no corresponding term \dfn{omnireferentiality} has appeared in the literature. That said, languages which have undergone \dfn{insubordination} \parentext{in which subordinate clauses are reanalyzed as main clauses \parencites{Evans2007}{Mithun2008}{EvansWatanabe2016}} do often exhibit noun-oriented polyfunctionality in the sense that verbal inflection mirrors nominal inflection. This is because one common insubordination pathway is when nominalized subordinate clauses are reanalyzed as main clauses, so that nominal inflection marking is reanalyzed as verbal inflectional marking. This process of insubordination famously led to the claim that all lexical items in Eskimo languages are fundamentally nominal in nature \parencite{Sadock1999}. However, cases of insubordination do not constitute counterexamples to the predicating tendency in language. Even in these languages, the use of action words for reference is still less marked than the use of object words for predication.

\textcite{Kastovsky1996} argues that this asymmetry arises from the fact that \textquote[{\cite[96]{Kastovsky1996}}]{deverbal nouns have a much more diversified semantics than denominal verbs}, meaning that the range of possible meanings for a deverbal noun (a noun derived from a verb) is broader than for a denominal verb (a verb derived from a noun). Examining data from \idx{English}, \citeauthor{Kastovsky1996} shows that when an object word is used to predicate, its possible meanings are limited to combinations of \textsc{be}, \textsc{be like}, \textsc{be in}, \textsc{become}, \textsc{have}, \textsc{do}, \textsc{do with}, and \textsc{cause}. When an action word is used as a referent, however, the range of meanings include any abstract representation of the event itself (an action nominalization), or any one of the arguments associated with the verb, which come in a variety of semantic roles.

A similar, cognitively oriented explanation for reference-predication asymmetries is given by \textcite[745]{HopperThompson1984}:

\blockquote[{\cite[745]{HopperThompson1984}}]{[Deverbal] nominalization names an event taken as an entity; however, a \enquote{verbalization} does not name an \enquote{entity taken as an event}, but simply names an event associated with some entity. In other words, a nominalization still names an event, albeit one which is being referred to rather than reported on in the discourse; it is, accordingly, still in part a [verb], and not a \enquote{bona fide} [noun]. However, a denominal [verb] no longer names an entity at all, and thus has no nominal \enquote{stains} to prevent its being a bona fide [verb].}

\noindent\textcite[746]{HopperThompson1984} analyze nominalizations as a kind of metaphor following \textcite[3a]{LakoffJohnson1980}, in which an abstract event is conceptualized as a concrete object. However, they argue that verbalizations are not a type of metaphor but rather a kind of metonymic extension, thereby explaining the asymmetry in the directionality of lexical polyfunctionality.

\subsubsection{Locus of categoriality}
\label{sec:2.3.2.3}

The morphological level at which a language exhibits polyfunctionality—the root, the stem, or the fully inflected word—differs from one language to the next. In some languages, roots are strongly associated with a particular discourse function, but stems are polyfunctional; in other languages, the reverse is true. I refer to the linguistic level at which a language associates different discourse functions as its \dfn{locus of categoriality}. Some linguistic theories include a premise that the locus of categoriality in every language always sits at a certain level \parencites{HalleMarantz1994}{Baker2003}{Baker2015}{BooijAudring2018}{Siddiqi2018}, but the evidence from research on lexical polyfunctionality gives strong empirical support to the position that locus of categoriality varies from language to language. In contrast to either of these positions, \textcite{BlaszczakKlimekJankowskaMigdalski2015} argue that category information is distributed across different levels of representation.

As one illustration of how polyfunctionality depends on grammatical level, we have seen that roots in \idx{Central Alaskan Yup'ik} are generally categorical: except for 12\% of roots, they are typically strongly associated with just one discourse function, and derivational affixes select for roots of a particular category \parencite[162--167]{Mithun2017}. While many derived stems are also strictly associated with just one discourse function, a large but indeterminate number have both referential and predicative uses. Examples of such polyfunctional stems have already been shown in \exref{ex:1.7} in \secref*{sec:1.1}. Fully inflected words in Central Alaskan Yup'ik, however, never exhibit polyfunctionality \parencite[6]{Mithun2019}. So Central Alaskan Yup'ik displays partial polyfunctionality at the root and stem level but not the inflected word level.

As another example, in \idx{Mandinka} all stems are polyfunctional. No Mandinka stem except for \txn{sǎa} \tln{die} is used in just one discourse function \parencite[46]{Creissels2017}. At the level of the inflected word, however, lexical items in Mandinka belong unambiguously to one category or another \parencite[37]{Creissels2017}. Mandinka therefore shows total polyfunctionality at the stem level but total monofunctionality at the inflected word level. (\citeauthor{Creissels2017} does not include an analysis of roots in his discussion.)

Some languages display polyfunctionality even at the level of the fully inflected word. In many North American languages, it is common for fully morphological verbs to function as referents \parencite{Hieberfc}, as shown in the following examples.

\clearpage

\begin{exe}

  \ex\label{ex:2.10}
  \exinfo{\idx{Chitimacha} (isolate)}
  \begin{xlist}

    \ex
    \glll dzampuyna\\
          dza‑ma‑(p)uy‑na\\
          thrust‑\gl{plact}‑\gl{hab}‑\gl{nf.pl}\\
    \tln{they usually thrust / spear with it}\\
    \tln{spear}
    \exsource[56]{Swadesh1939a}

    \ex
    \glll pamtuyna\\
          pamte‑(p)uy‑na\\
          ford‑\gl{hab}‑\gl{nf.pl}\\
    \tln{they usually cross (it)}\\
    \tln{bridge}
    \exsource[17]{Swadesh1939a}

  \end{xlist}

  \ex\label{ex:2.11}
  \exinfo{\idx{Cayuga} (Iroquoian > Lake Iroquoian)}
  \begin{xlist}

    \ex\label{ex:2.11a}
    \glll ǫtekhǫnyáʔthaʔ\\
          ye‑ate‑khw‑ǫni‑aʔt‑haʔ\\
          \gl{indef.agt.refl}‑meal‑make‑\gl{instr}‑\gl{ipfv}\\
    \tln{one makes a meal with it}\\
    \tln{restaurant}
    \exsource[200]{Mithun2000}

    \ex\label{ex:2.11b}
    \glll kaǫtanéhkwih\\
          ka‑rǫt‑a‑nehkwi\\
          \gl{neut.agt}‑log‑\gl{ep}‑haul.\gl{ipfv}\\
    \tln{it hauls logs}\\
    \tln{horse}
    \exsource[200]{Mithun2000}

  \end{xlist}

  \ex\label{ex:2.12}
  \exinfo{\idx{Navajo} (Na-Dene)}
  \begin{xlist}

    \ex
    \gll tsinaaʼeeɬ\\
         tsi(n)‑naaʼeeɬ\\
         wood‑it.moves.about.floating\\
    \tln{ship, boat}
    \exsource[316]{Young1989}

    \ex
    \gll chahaɬheeɬ\\
         it.is.dark\\
    \tln{darkness}
    \exsource[316]{Young1989}

  \end{xlist}

\end{exe}

\noindent Each of these polyfunctional uses of a morphological verb sits somewhere on a continuum between being fully lexicalized as a referent, so that its predicating use is no longer available, to being a fully productive predicate, with both predicative and referential uses \parencite[413]{Mithun2000}.

The reason that lexical items may exhibit polyfunctionality at one level of analysis but not another is because \textquote[{\cite[1]{Mithun2019}}]{\emph{categorial} shift is often not \emph{categorical}}. When an item expands its use into new contexts, not all the morphological, syntactic, and semantic properties of the item shift to accommodate that new use at the same time. It takes time before the morphosyntactic properties of an item adjust to reflect its new use, a process referred to as \dfn{actualization} in the grammaticalization literature \parencite{DeSmet2012} and \dfn{post-constructionalization constructional changes} in the framework of diachronic construction grammar \parencite[27]{TraugottTrousdale2013}.

It is in part because the locus of categoriality can vary from language to language that I have used the vague term \dfn{lexical item} throughout this dissertation, which is intended to be a convenient cover term for root, stem, or inflected word.

\subsubsection{Item-specificity}
\label{sec:2.3.2.4}

A final significant finding to emerge from the empirical research on lexical polyfunctionality is the fact that polyfunctionality is \dfn{item-specific} and even \dfn{sense-specific}. Individual lexical items or even individual senses of an item that are otherwise very similar in their meanings and morphosyntactic behavior can nonetheless differ in terms of their functional diversity.

This fact is nicely illustrated by both \posscitet{Mithun2017} study of lexical polyfunctionality in \idx{Central Alaskan Yup'ik} and \posscitet{Creissels2017} study of \idx{Mandinka}. \textcite[163--164]{Mithun2017}, for example, considers roots for meteorological concepts, and shows that even within this small semantic domain, roots vary as to whether they exhibit polyfunctionality. In \exref{ex:2.13a} the meteorological roots have predicative counterparts but in \exref{ex:2.13b} the meteorological roots do not.

\begin{exe}
  \ex\label{ex:2.13}
  \exinfo{\idx{Central Alaskan Yup'ik} (Eskimo-Aleut > Yupik)}
  \begin{xlist}

    \ex\label{ex:2.13a}
    \begin{tabularx}{\linewidth}[t]{ p{0.75in} >{\raggedright\arraybackslash\hangindent=1.5em}X p{1em} p{0.75in} >{\raggedright\arraybackslash\hangindent=1.5em}X }
      \txn{amirlu} & \tln{cloud}          & { } & \txn{amirlu‑} & \tln{be cloudy}\\
      \txn{kaneq}  & \tln{frost}          & { } & \txn{kaner‑}  & \tln{be frosted}\\
      \txn{aniu}   & \tln{snow on ground} & { } & \txn{aniu‑}   & \tln{to snow}\\
    \end{tabularx}

    \ex\label{ex:2.13b}
    \begin{tabularx}{\linewidth}[t]{ p{0.75in} >{\raggedright\arraybackslash\hangindent=1.5em}X p{1em} p{0.75in} >{\raggedright\arraybackslash\hangindent=1.5em}X }
      \txn{taituk}    & \tln{fog, mist} & { } & *\txn{taitug‑}    & \tln{be foggy}\\
      \txn{kavtak}    & \tln{hailstone} & { } & *\txn{kavtag‑}    & \tln{to hail}\\
      \txn{mecaliqaq} & \tln{sleet}     & { } & *\txn{mecaliqar‑} & \tln{to sleet}\\
    \end{tabularx}

  \end{xlist}
  \exsourcebelow[163]{Mithun2017}
\end{exe}

\noindent \citeauthor{Mithun2017} also provides similar data illustrating functional gaps for the domains of clothing and instruments:

\begin{exe}

  \ex\label{ex:2.14}
  \exinfo{\idx{Central Alaskan Yup'ik} (Eskimo-Aleut > Yupik)}
  \begin{xlist}

    \ex
    \begin{tabularx}{\linewidth}[t]{ p{0.75in} >{\raggedright\arraybackslash\hangindent=1.5em}X p{1em} p{0.75in} >{\raggedright\arraybackslash\hangindent=1.5em}X }
      \txn{taqmak} & \tln{dress}                & { } & \txn{taqmag‑} & \tln{put on a dress}\\
      \txn{nacaq}  & \tln{hat, parka hood, cap} & { } & \txn{nacar‑}  & \tln{put on a hat, hood}\\
      \txn{atkuk}  & \tln{parka}                & { } & \txn{atkug‑}  & \tln{put on a parka}\\
    \end{tabularx}

    \ex
    \begin{tabularx}{\linewidth}[t]{ p{0.75in} >{\raggedright\arraybackslash\hangindent=1.5em}X p{1em} p{0.75in} >{\raggedright\arraybackslash\hangindent=1.5em}X }
      {*}\txn{piluk} & \tln{footwear} & { } & \txn{pilug‑} & \tln{put on footwear}\\
      {*}\txn{at'e}  & \tln{clothing} & { } & \txn{at'e‑}  & \tln{don, put on clothing}\\
      {*}\txn{kive}  & \tln{pants}    & { } & \txn{kive‑}  & \tln{pull down pants}\\
    \end{tabularx}

  \end{xlist}

  \ex\label{ex:2.15}
  \exinfo{\idx{Central Alaskan Yup'ik} (Eskimo-Aleut > Yupik)}
  \begin{xlist}

    \ex
    \begin{tabularx}{\linewidth}[t]{ p{0.75in} >{\raggedright\arraybackslash\hangindent=1.5em}X p{1em} p{0.75in} >{\raggedright\arraybackslash\hangindent=1.5em}X }
      \txn{ay'uytaq}  & \tln{hockey stick}        & { } & \txn{ay'utar‑}   & \tln{play hockey}\\
      \txn{iqsak}     & \tln{fishhook}            & { } & \txn{iqsag‑}     & \tln{to jig for fish}\\
      \txn{kapkaanaq} & \tln{trap}                & { } & \txn{kapkaanar‑} & \tln{to trap, get trapped}\\
      \txn{keviq}     & \tln{plug, cork, stopper} & { } & \txn{kevir‑}     & \tln{to plug, stuff, caulk}\\
      \txn{kuvya}     & \tln{fishnet}             & { } & \txn{kuvya‑}     & \tln{fish by driftnetting}\\
    \end{tabularx}

    \ex
    \begin{tabularx}{\linewidth}[t]{ p{0.75in} >{\raggedright\arraybackslash\hangindent=1.5em}X p{1em} p{0.75in} >{\raggedright\arraybackslash\hangindent=1.5em}X }
      {*}\txn{kagi}   & \tln{broom}                & { } & \txn{kagi‑}   & \tln{sweep}\\
      {*}\txn{ipuk}   & \tln{ladle}                & { } & \txn{ipug‑}   & \tln{ladle, move with bow of boat high in air}\\
      {*}\txn{pangeq} & \tln{double-bladed paddle} & { } & \txn{panger-} & \tln{paddle with a double-bladed paddle}\\
    \end{tabularx}

  \end{xlist}

\end{exe}

\noindent On the basis of data like these and discussion with speakers, \citeauthor{Mithun2017} observes, \textquote[{\cite[163]{Mithun2017}}]{Speakers simply know whether a given root functions as a noun and what its meaning is, and whether it functions as a verb and what its meaning is. Gaps are not predictable[.]}. These gaps also vary from dialect to dialect. While the dialect in the above examples has no predicative counterpart for \txn{taituk} \tln{fog}, the Nunivak Island dialect does have a pair of roots \txn{nugu} \tln{fog} and \txn{nungu-} \tln{be foggy}.

\posscitet{Creissels2017} study of \idx{Mandinka} is another good illustration of the item-specific nature of polyfunctionality. While Mandinka has nominal and verbal constructions that allow the predicative and referring functions of inflected words to be distinguished unambiguously, it is not as easy to separate stems into similar classes. In Mandinka, all items are polyfunctional, but the \emph{way} in which items are polyfunctional varies. Stems in Mandinka may be divided into three classes based on their semantic behavior with regards to polyfunctionality:

\begin{itemize}
  \singlespacing
  \item \textit{verbal} lexemes are those whose meaning is predictable when used to refer and therefore analyzable as a case of \enquote{morphologically unmarked nominalization}; these are always event nominalizations
  \item \textit{verbo-nominal} lexemes are those whose meaning in referring constructions is idiosyncratic and therefore not predictable
  \item \textit{nominal} lexemes are those whose meaning when used as predicates is predictable and limited to \tln{provide someone with X}
\end{itemize}

\noindent In \idx{Mandinka}, therefore, polyfunctionality must be assessed on an item-by-item basis since the behavior of each item with regard to polyfunctionality may differ.

In fact, polyfunctional behavior in \idx{Mandinka} is not just item-specific, but sense-specific as well. \textcite[54]{Creissels2017} reports that polysemous lexemes may show different behavior for their different senses. The stem \txn{díŋ}, for example, has two senses: \tln{child, young (of an animal)} and \tln{fruit}. However, only the \tln{fruit} sense is available for predication: when used as an intransitive verb, \txn{díŋ} may only mean \tln{bear fruit}, not \tln{give birth}, even though \tln{give birth} is a perfectly conceivable meaning of this stem in predication. In the sense of \tln{child, young (of an animal)}, \txn{díŋ} behaves as a nominal lexeme, but in the sense of \tln{fruit} it behaves as a verbo-nominal lexeme.

When lexical items undergo functional expansion into new discourse functions, it is also only specific senses that do so, not every one of its senses. More evidence for this comes from the diachronic development of the word \txn{run} in English: though the word \txn{run} when used as a predicate has numerous senses, the earliest attestations of \txn{run} used referentially are by and large with just the prototypical sense of \tln{fast pedestrian motion} (the exceptions to this stem from just one corpus file) \parencite[76]{Gries2006}. Other referential uses of \txn{run} did not develop until later.

The existence of dialectal differences in lexical polyfunctionality as well as the unpredictable meanings of lexical items when used in various discourse functions show that the development of polyfunctionality depends on conventionalization—whether a given form has assumed a conventionalized meaning in its role for a specific discourse function. These conventionalizations are language-specific, dialect-specific, item-specific, and even sense-specific \parencite[97]{Croft2000}. Speakers can and do playfully use existing lexical items for new discourse functions, and these constitute cases of genuine lexical \emph{flexibility}, but it is not until that combination of form and discourse function is conventionalized with a specific meaning in a community of speakers that we can say the lexical item has undergone functional expansion and become polyfunctional. (This point is discussed further in \secref*{sec:2.5}.) Speakers must memorize individual pairings of meaning and syntactic distributions \parencite[217]{Beck2013}. An excellent illustration of this is the word \txn{friend} in \idx{English}. Prior to the widespread adoption of the social networking sites MySpace and Facebook around 2006, the use of \txn{friend} as a predicate had not been widely conventionalized. The growth of social networking sites then led to the specific use of \txn{friend} to mean \tln{add as a connection on a social networking site}. Note that it does \emph{not} have the more general sense of \tln{be a friend} or \tln{befriend}. Like with Yup'ik\index{Central Alaskan Yup'ik} and \idx{Mandinka}, this shows not just that polyfunctionality is item-specific, but that the meanings of polyfunctional uses are often item-specific as well; in many cases the meaning is unpredictable and must be memorized by speakers.

\subsection{Problems \& critiques}
\label{sec:2.3.3}

Many researchers have challenged the notion of lexical flexibility, and/or its presence in various languages. Some of these challenges stem from the fact that certain conceptions of lexical flexibility are based on traditional ideas about the existence of large, language-specific parts of speech, and therefore subject to the same set of criticisms. Other challenges stem from precisely the facts presented in the previous section, namely that both polyfunctionality and the meaning of polyfunctional words are item-specific and often unpredictable, such that these words are not actually flexible. Moreover, languages must indicate the discourse function of their lexical items somehow—this is basic to our ability to communicate. In a certain sense, the idea that there are items which are fully ambiguous in their discourse function is doomed at the outset. The question is really where these indications of discourse function live—the root, the stem, the inflected word, or the clausal context. This section summarizes the main criticisms that scholars have raised against flexible analyses. In \secref*{sec:2.4}, we then look at alternative theories of word classes and their approach to polyfunctionality.

\subsubsection{Methodological opportunism}
\label{sec:2.3.3.1}

A methodological problem with certain theories of flexible items is that they, like traditional theories, commit the fallacy of \dfn{methodological opportunism} \parencite[30, 41]{Croft2001b} presented in \secref*{sec:2.2.3}. They do not apply the distributional method consistently. Instead, the criteria which separate lexical items into categories are determined on the basis of additional theoretical commitments. \textcite[§2.2.2]{Croft2001b} criticizes Hengeveld's parts-of-speech typology on this basis, noting that Hengeveld ignores distributional evidence for classes smaller than the ones he posits in his typology (noun, contentive, etc.). \textcite{EvansOsada2005} raise similar concerns for Hengeveld's theory as applied to \idx{Mundari}. They state that in order for two lexical items to be members of the same lexical class, they must have \dfn{equivalent combinatorics}, which is to say that their distributions should be identical \parencite[366]{EvansOsada2005}. \citeauthor{EvansOsada2005} also state that for a language to flexible, that flexibility must be \dfn{exhaustive} in the sense that all members of a putatively flexible class must show equal degrees of flexibility and \dfn{bidirectional} in the sense that nouns may be used as verbs and vice versa. Both these criteria are merely different ways of reframing the broader principle that items in a class should share the same distributions \parencite[434]{Croft2005}. \citeauthor{EvansOsada2005} proceed to show various ways in which these criteria are not applicable to Mundari, and that Mundari is therefore not a flexible language. At the same time, however, \citeauthor{EvansOsada2005} use these facts to argue for the existence of the equally problematic categories of Noun and Verb in Mundari, using just a \textquote[{\cite[434, fn. 17]{EvansOsada2005}}]{canonical subset of distributional facts}. \posscitet{Croft2005} commentary on \posscitet{EvansOsada2005} target article is partially devoted to critiquing them on this point. The problem of methodological opportunism is present for any analysis which assumes that languages have a small set of large lexical categories—whether that analysis is flexible or traditional.

\subsubsection{Semantic shift}
\label{sec:2.3.3.2}

Broadly speaking, however, the primary argument against theories of flexible word classes is that they ignore a great deal of item-specific knowledge that speakers have about lexical items and their uses in different functions \parencites[§3.2]{EvansOsada2005}[216]{Beck2013}. This issue has already been discussed in some detail in \secref*{sec:2.3.2.4}, but it bears explaining precisely why such item-specific knowledge constitutes a problem for theories of lexical flexibility.

For starters, when a lexical item expands into a new discourse function, there is a \dfn{semantic shift} in the direction of the meaning typically associated with the new context \parencites[74--77]{Croft1991}[73]{Croft2001b}. For example, when a property word is used in a referring expression, its meaning shifts to a person or thing possessing that property, not a reference to the abstract property itself. The precise meaning that results from these shifts, however, cannot be attributed to some broader pragmatic principles—they are a matter of convention and require broader uptake in a community of speakers in order to be conventionalized (as illustrated with the English word \txn{friend} above). Because the meaning that results from this semantic shift is conventional, language-specific, and often idiosyncratic, flexible items cannot be truly productive, as is implied by the term \enquote{flexible}. There is always a conventionalized component to their meanings.

Examples of idiosyncratic and unproductive shifts in the meaning of polyfunctional items abound in the literature. Consider again the examples from \idx{Mundari} in \exref{ex:1.3}, repeated here as \exref{ex:2.16}.

\begin{exe}
  \ex\label{ex:2.16}
  \exinfo{\idx{Mundari} (Austroasiatic > Munda)}
  \begin{xlist}

    \ex
    \gll buru=ko                bai‑ke‑d‑a.\\
         mountain=\gl{3pl.subj} make‑\gl{compl}‑\gl{tr}‑\gl{ind}\\
    \tln{They made the mountain.}
    \exsource[354]{EvansOsada2005}

    \ex
    \gll saan=ko                buru‑ke‑d‑a.\\
         firewood=\gl{3pl.subj} mountain‑\gl{compl}‑\gl{tr}‑\gl{ind}\\
    \tln{They heaped up the firewood.}
    \exsource[355]{EvansOsada2005}

  \end{xlist}
\end{exe}

\noindent As a predicate, the stem \txn{buru} means \tln{heap up}, but this meaning is not predictable from just the combination of the nominal sense \tln{mountain} and its predicative use. The word could have just as easily meant \tln{climb a mountain} or \tln{overcome} or simply \tln{be a mountain}. No general pragmatic principles could have predicted this meaning. Likewise consider the \idx{Central Alaskan Yup'ik} examples in \exref{ex:1.7c} from \chref{ch:introduction}. Why does the combination of \txn{iqeq-} \tln{corner of mouth} + \txn{-mik} \tln{thing held in one's mouth}, \tln{to put in one's mouth} result in \txn{iqmik} \tln{chewing tobacco}? Why not \tln{oral thermometer} or \tln{toothpick}? Mithun provides many more unpredictable examples, shown in \exref{ex:2.17}.

\begin{exe}
  \ex\label{ex:2.17}
  \exinfo{\idx{Central Alaskan Yup'ik} (Eskimo-Aleut > Yupik)}
  \begin{xlist}

    \ex
    \begin{tabularx}{\linewidth}[t]{ p{1in} >{\raggedright\arraybackslash\hangindent=1.5em}X }
      \txn{mecur-} & \tln{get blood poisoning}\\
      \txn{mecuq}  & \tln{liquid part of something, sap, juice, green/waterlogged wood}\\
    \end{tabularx}

    \ex
    \begin{tabularx}{\linewidth}[t]{ p{1in} >{\raggedright\arraybackslash\hangindent=1.5em}X }
      \txn{melug-} & \tln{suck; eat roe directly from the fish}\\
      \txn{meluk}  & \tln{fish eggs, roe, fish eggs prepared by allowing them to age and become a sticky mess}\\
    \end{tabularx}

    \ex
    \begin{tabularx}{\linewidth}[t]{ p{1in} >{\raggedright\arraybackslash\hangindent=1.5em}X }
      \txn{qager-} & \tln{explode, to pop}\\
      \txn{qageq}  & \tln{blackfish which is boiled, allowed to set in its cooled, jelled broth}\\
    \end{tabularx}

    \ex
    \begin{tabularx}{\linewidth}[t]{ p{1in} >{\raggedright\arraybackslash\hangindent=1.5em}X }
      \txn{qumig-} & \tln{hold inside (of clothing)}\\
      \txn{qumik}  & \tln{enclosed thing, thing inside, fetus}\\
    \end{tabularx}

    \ex
    \begin{tabularx}{\linewidth}[t]{ p{1in} >{\raggedright\arraybackslash\hangindent=1.5em}X }
      \txn{aveg-} & \tln{divide in half, to halve}\\
      \txn{avek}  & \tln{half}; also \tln{half-dollar; person who is half Native}\\
    \end{tabularx}

    \ex
    \begin{tabularx}{\linewidth}[t]{ p{1in} >{\raggedright\arraybackslash\hangindent=1.5em}X }
      \txn{napa-} & \tln{stand upright}\\
      \txn{napa}  & \tln{tree}\\
    \end{tabularx}

    \ex
    \begin{tabularx}{\linewidth}[t]{ p{1in} >{\raggedright\arraybackslash\hangindent=1.5em}X }
      \txn{yuurqar-} & \tln{sip}\\
      \txn{yuurqaq}  & \tln{hot beverage, tea}\\
    \end{tabularx}

  \end{xlist}
\end{exe}

\noindent Or consider the example from \idx{Cayuga} in \exref{ex:2.11b}, repeated here as \exref{ex:2.18}.

\begin{exe}
  \ex\label{ex:2.18}
  \exinfo{\idx{Cayuga} (Iroquoian > Lake Iroquoian)}
  \vfix
  \glll kaǫtanéhkwih\\
        ka‑rǫt‑a‑nehkwi\\
        \gl{neut.agt}‑log‑\gl{ep}‑haul.\gl{ipfv}\\
  \vfix
  \tln{it hauls logs}\\
  \tln{horse}
  \exsource[200]{Mithun2000}
\end{exe}

\noindent Of all the possible nominal meanings that could reasonably derive from \tln{it hauls logs}—cart, tractor, ox—the fact that its nominal use means \tln{horse} is specific to Cayuga and must be memorized by speakers.

Conventionalizations of lexical items used in new discourse functions also vary across languages. While the principle of semantic shift still broadly holds, the specific meanings of these conventionalizations are unpredictable. Croft exemplifies this point by comparing \idx{English} \txn{school} with \idx{Tongan} \txn{ako} \tln{school / study}.

\blockquote[{\cite[71]{Croft2000}}]{English \txn{school} used predicatively does not mean the same thing as Tongan \txn{ako} used predicatively, namely \tln{study}. Going in the opposite direction, English \txn{study} used referentially does not mean the same thing as Tongan \txn{ako} used referentially, namely \tln{school}. Finally, English \txn{small} used referentially does not mean the same thing as Tongan \txn{si'i} \tln{childhood} used referentially.}

\noindent Since the meanings of putatively flexible items in different discourse functions are not predictable, many scholars reason that these lexical items cannot be truly \enquote{flexible} in the sense of polycategorial or precategorial.

\subsubsection{Lexical gaps}
\label{sec:2.3.3.3}

Just as unpredictable in polyfunctional cases is which sense of a item will be co-opted into the new discourse function. In \idx{Wolof}, for example, the referential use of the word \txn{ndaw} can only mean \tln{young}, whereas the predicative use may mean either \tln{be young} or \tln{be little, small} \parencite[91]{Kihm2017}. Not all senses of a lexical item are available in all its discourse functions. Moreover, not all lexical items within a morphosyntactic or semantic class necessarily have the same range of discourse functions. We have already seen these kinds of lexical gaps for \idx{Central Alaskan Yup'ik} and \idx{Mandinka} in \secref*{sec:2.3.2.4} above. If a polyfunctional lexical item lacks any conventionalized use in different discourse functions, than it cannot rightly be considered flexible.

\section{Functional approaches}
\label{sec:2.4}

Functionalism as an approach to linguistic explanation is multifaceted. It looks to factors outside of the structural form of language as an explanation for that form—most especially cognition, usage effects from frequency, and information structuring in discourse \parencite[6323--6324]{Croft2001a}. In this section I present Croft's \parencites*{Croft1991}{Croft2000}{Croft2001b} functional theory of lexical categories, which explains crosslinguistic patterns in the coding of reference, predication, and modification as arising from the interaction between our mental categories and the needs of discourse. I then use this theory as a framework for defining lexical polyfunctionality in \secref*{sec:2.5}. I begin with a brief discussion of prototype theory as it pertains to lexical categories (\secref{sec:2.4.1}), before expounding upon typological markedness theory (\secref{sec:2.4.2}).

\subsection{Prototype theory}
\label{sec:2.4.1}

It has long been recognized that the categories of human cognition are prototypal. In a series of studies, Eleanor Rosch and colleagues demonstrate that category membership is a matter of degree, and that there are better and worse representatives of any given mental category \parencites{Rosch1973a}{Rosch1973b}{Rosch1975}{RoschMervis1975}{Roschetal1976}{Rosch1978}. Prototype theory was then popularized in linguistics by \textcite{Lakoff1987}, \textcite{Langacker1987}, \citeauthor{Taylor2003} \parentext{[1989] \citeyear{Taylor2003}}, and Croft \parentext{[1990] \citeyear{Croft2003}; \citeyear{Croft1991}}, among others.

The evidence for prototypal structure in mental categorization is robust \parencite[46--47]{Taylor2003}. When asked to rate whether an item is a good example of a category, participants consistently rate prototypical members as better examples of the category than non-prototypical ones. In listing experiments where participants are asked to list members of a category, prototypical members are listed earlier and more frequently than non-prototypical members. Finally, prototypical members of a category are identified by participants as being members of the category more quickly than non-prototypical members. Each of these effects is scalar, such that individual members of a category sit anywhere on a scale of more to less prototypical.

Prototype effects arise from the basic human need to interpret the world around us: \textquote[{\cite[xi]{Taylor2003}}]{Strictly speaking, every entity and every situation that we encounter is uniquely different from every other. In order to function in the world, all creatures, including humans, need to be able to group different entities together as instances of the same kind. […] [C]ategorization serves to reduce the complexity of the environment.}. This fact is often referred to as the \dfn{principle of cognitive economy}, whereby we group simlar stimuli together in order to maximize information while minimizing cognitive effort \parencite[255]{EvansGreen2006}. The gradience within these groupings results from the fact that \textquote[{\cite[149]{LewandowskaTomaszczyk2007}}]{concepts function as mental reference points. When we come across new phenomena, we tend to interpret them in terms of existing categories}.

Linguistic constructions are also subject to prototype effects \parencite[Ch.~12]{Taylor2003}. \textcite{HopperThompson1980}, though not yet working in a prototype framework, nonetheless demonstrate that transitivity is very much a prototype category, with individual clauses showing greater or lesser degrees of transitivity depending on their features. \textcite{Ross1972} shows that lexical items are graded in their ability to undergo various transformations, with human beings being close to prototypical noun phrases, while inanimates, events, abstract concepts are less prototypical. \textcite[§12.5]{Taylor2003} likewise points out that the transitive construction in \idx{English} has steadily expanded its functions over time \textquote[{\cite[235]{Taylor2003}}]{to encode states of affairs which diverge increasingly from prototypical transitivity}. The result of this diachronic development is significant gradation as to which verbs now lend themselves to transitivization. \textcite[236]{Taylor2003} gives the example of the transitive construction being used to imply a semantic path, in lieu of an explicit preposition. Compare the pairs of \idx{English} sentences in \exref{ex:2.21}.

\begin{exe}
  \ex\label{ex:2.21}
  \hspace{0.5em}\exinfo{\idx{English} (Indo-European > Germanic)}\\
  \begin{tabular}[t]{ l l }
    \textit{Preposition}                         & \textit{Transitive}\\
    He regularly \em{flies across} the Atlantic. & He regularly \em{flies} the Atlantic.\\
    He \em{swam across} the Channel.             & He \em{swam} the Channel.\\
    She \em{swam across} our new swimming pool.  & ?She \em{swam} our new swimming pool.\\
    We \em{drove across} the Alps.               & ?We \em{drove} the Alps.\\
    The child \em{crawled across} the floor.     & *The child \em{crawled} the floor.\\
  \end{tabular}
  \vspace{0.5em}
  \exsourcebelow[236]{Taylor2003}
\end{exe}

\noindent These examples illustrate that there are indeed better and worse members of the \idx{English} Transitive Path construction.

Individual lexemes are also a type of construction, and therefore also subject to prototype effects. This is unsurprising, since language forces speakers to map a non-discrete cognitive representation of the world onto discrete linguistic entities—we are forced to cut up and categorize the world around us into discrete objects and events/states so that we can refer to them and predicate statements about them. Reality, however, is not so neat. The result of this mapping is a linguistic form that imperfectly demarcates a portion of our mental world, centered on a clear prototype but with imprecise boundaries. Using a topological metaphor, we typically call some portion of our mental representation of the world a \dfn{semantic space} or \dfn{conceptual space} \parencite[92]{Croft2001b}, and that space can be graphically represented using a \dfn{semantic map} \parencites[§2.4.3]{Croft2001b}{Haspelmath2003}. Though semantic maps are most often used to represent a \emph{functional} space for grammatical morphemes, they are equally applicable to lexical spaces as well. \textcite[74]{Gries2006} provides one such semantic map for the meanings of the \idx{English} word \txn{run}, shown in \figref{fig:semantic-map-run}, based on a comprehensive corpus analysis. As another example, \textcite[485]{BowermanChoi2001} present a semantic map of spatial relations based on data from 38 languages (25 families), with a relation indicating prototypical support from below (\textsc{on}) at one end and a relation indicating prototypical containment (\textsc{in}) at the other. As pictured in \figref{fig:spatial-relations}, lexical items in different languages cut up this semantic space in different ways.

\clearpage
  \begin{figure}[h!]
    \includegraphics[keepaspectratio, height = \textheight, width = \textwidth]{semantic-map-run.jpg}
    \caption[Semantic map of English \textit{run}]{Semantic map of \idx{English} \textit{run} \parencite[74]{Gries2006}}
    \label{fig:semantic-map-run}
  \end{figure}
\clearpage

\begin{figure}[h!]
  \includegraphics[width=\linewidth]{spatial-relations.jpg}
  \caption[Crosslinguistic differences in the encoding of spatial relationships]{Crosslinguistic differences in the encoding of spatial relationships \parencite[485]{BowermanChoi2001}}
  \label{fig:spatial-relations}
\end{figure}

These examples illustrate that word meanings are polycentric and cover a range of possible uses, as mentioned in \secref*{sec:2.3.1.4}. Some of these uses may be more prototypical than others. The \idx{English} expression \txn{apple on a twig} is a slightly less prototypical use of \txn{on} than \txn{apple on a table}. The fact that lexical items cover a range of uses, and that some of these uses are more prototypical than others, is an important component of the typological markedness theory of lexical categories.

Even the formal categories that linguists use to describe linguistic structure tend to be prototypal \parencite[xii, 201]{Taylor2003}. \textcite[§11.1]{Taylor2003} argues that linguists' conceptions of the formal labels \dfn{word}, \dfn{affix}, and \dfn{clitic} are prototypal in nature, with better and worse members of the category. \textcite{Haspelmath2005} likewise shows that simple structural definitions of these categories are inadequate and reframes the word–affix continuum in functional terms instead. Much research in the Canonical Typology framework \parencite{Corbett2005} also demonstrates the prototypal nature of linguists' categories. Though Corbett is careful to distinguish between a \dfn{canon} and a \dfn{prototype} / \dfn{exemplar} \parencite[142]{Corbett2010}, his accumulated work nonetheless shows that linguists view phenomena in the world's languages as better or worse instances of various descriptive categories.

What type of category are lexical categories then? Are word classes categories of human cognition, categories within particular languages, categories of languages generally, or analytic categories of linguists? Or some combination of these? Typological markedness theory posits that parts of speech like noun, verb, and adjective are not categories of particular languages. Languages have constructions, not parts of speech. Speakers, however, have \emph{mental prototypes} of objects, actions, and properties. And although there is no \emph{one} Noun construction in \idx{English} that would correspond to the mental category of \textsc{object}, there are numerous constructions in English which have the function of indicating \emph{reference to an object}, such as the Definite Article construction or the Transitive Subject construction. Likewise, there is no one construction—in English or any language—that can be definitively called the Verb construction or the Adjective construction, but there are plenty of constructions which have the function of predicating or attributing properties. Naturally, then, speakers are more likely to use referring constructions when talking about something which they mentally categorize as an object, predicating constructions when talking about something they conceive of as an action, and modifying constructions when talking about something they conceptualize as a property.

Speakers' conceptualizations, however, are fluid. Speakers often conceptualize things in non-prototypical ways. They may construe events as bounded entities that they can refer to, or objects as properties with duration. As a result, speakers often use lexical items in constructions that do not align particularly well with the item's meaning, such as the appearance of an action word like \txn{sing} in a referring construction like the Gerund in the phrase \txn{his singing was beautiful}. When speakers use words in this atypical manner, those uses are much more likely to be marked in some way—whether morphologically, behaviorally, frequentially \parencite[§2.2]{Croft1991}. Non-prototypical uses of words also show a semantic shift in the direction of the discourse function they are being used for. As a consequence, clear asymmetries emerge between the prototypical vs. non-prototypical uses of object words, action words, and property words. It is the unmarked use of these lexical items that most closely aligns with linguists' traditional conceptions of noun, verb, and adjective. Parts of speech as traditionally conceived are nothing more than the emergent effects of our cognitive prototypes on language. They do not have any real status in grammar or individual grammars. This is the fundamental idea behind typological markedness theory. \secref*{sec:2.4.2} lays out this theory in more detail.

A last clarifying point is in order. Recognizing the existence of prototype-based categories, many linguists have described parts of speech as prototypal. \textcite[1--2]{Dixon2004}, for example, says that the word classes noun, verb, and adjective each have a \enquote{prototypical conceptual basis} and \enquote{prototypical grammatical functions}. \textcite[217]{Taylor2003} states, \enquote{A prototype view of \textsc{noun} entails that some nouns are better examples of the category, while others have a more marginal status.} But languages have constructions, not parts of speech, and individual constructions are not gradient \parencite{Croft2007}. What linguists are in fact observing when they say that parts of speech are prototypal is not gradation in \emph{linguistic categories} like noun, verb, and adjective (since those are not categories of particular languages), but rather gradation in the \emph{mental categories} of objects, actions, and properties, which do indeed exhibit prototype structures, and which therefore have emergent effects on the organization of constructions in languages.

\subsection{Typological markedness theory}
\label{sec:2.4.2}

I have already previewed various aspects of typological markedness theory at different points in this dissertation. In this section I present a concise overview of the specific claims made by this theory, and some of the evidence for those claims. The phrase \dfn{typological markedness} or \dfn{typological markedness asymmetries} simply refers to an implicational universal regarding the behavior of basic versus non-basic members of a conceptual category. At its simplest, the theory posits that less basic or prototypical members of a category are marked in some way; basic or prototypical category members are unmarked by comparison \parencite{Greenberg1966}. This \emph{cognitive} markedness is then realized \emph{linguistically} in several ways. The marked member of a category may be literally marked with an affix or other overt morphological indicator, but this is just one of the ways an item can be a marked member of a category. The marked member of a category may also be less frequent, or have a smaller range of inflectional / distributional possibilities. It is important to emphasize that typological markedness does not always entail formal markedness. Typological markedness is an implicational universal rather than an absolute universal. The more marked members of a category must be \emph{at least as marked} as the unmarked member, but this does not preclude the possibility of all members being equally marked. Formal markedness is merely an emergent tendency of structures to reflect cognitive markedness.

As applied to word classes, typological markedness theory states that the most unmarked discourse functions for object, action, and property words are reference, predication, and modification, respectively. Therefore, when a lexical item is used for a function that does not align with its prototypical meaning, typological markedness theory predicts that it will be marked. Again, it must be emphasized that not every instance of a lexical item being used in a non-prototypical function will be marked in comparison to its prototypical function; but it will always be \emph{at least as} marked. This theory of typological markedness for the major discourse functions is laid out in detail by Croft in various publications \parencites{Croft1991}{Croft2000}{Croft2001b}{CroftLier2012}. It is also important to understand that typological markedness theory is not a theory of parts of speech in the sense of large partitionings of the lexicon into categories like noun, verb, and adjective. Instead, noun, verb, and adjective are epiphenomenal, crosslinguistic markedness patterns that arise from the interaction of semantic prototypes (object, action, property) and their use in different discourse functions (reference, predication, and modification). They are not categories of particular languages.

Throughout this dissertation, I have used the term \dfn{discourse function} to refer to the functions of reference, predication, or modification. These are what \textcite[51]{Croft1991} calls \dfn{pragmatic functions} or \dfn{propositional act functions} following the tradition of pragmatics and speech act theory in philosophy \parencites{Austin1962}{Searle1969}. These three functions are taken as fundamental to human communication, arising out of the communicative intent behind what speakers are attempting to \emph{do} with language. This perspective was articulated early on by Sapir:

\blockquote[{\cite[87]{Sapir1921}}]{There must be something to talk about and something must be said about this subject of discourse once it is selected. This distinction is of such fundamental importance that the vast majority of languages have emphasized it by creating some sort of formal barrier between the two terms of the proposition.}

\noindent A similar point is made by Croft while articulating his theory of typological markedness as applied to lexical categories: \textquote[{\cite[124]{Croft1991}}]{[N]o matter how complex a given situation is in terms of the number of entities involved and the number and kinds of relations that hold between them, a human being attempting to describe it in natural language must split it into a series of reference-predication pairs[.]}

Modification is generally seen as less central a function than reference and predication, as illustrated by its lack of mention in the quotes above. For example, \textcite[55]{Hengeveld1992} takes the reference-predication dichotomy to be fundamental, yielding the major categories of noun and verb, while the modification function then combines with these two functions to yield the major categories of adjective and adverb, respectively. The primacy of the reference-predication distinction also appears to be reflected structurally in the world's languages, which do not always have dedicated morphological means for encoding modification but appear to always have morphological strategies dedicated to reference and predication.

\textcite[123]{Croft1991} defines the pragmatic functions in terms of their discourse functions, following work in the discourse-functional tradition \parencites{Chafe1976}{HopperThompson1984}{Chafe1987}{DuBois1987}. Previous research defines \dfn{referents} as \textquote[{\cite[711]{HopperThompson1984}; \cite{Kibrik2011}}]{discourse-manipulable participants}, \dfn{predicates} as reported events \parencite[726]{HopperThompson1984}, and \dfn{modifiers} as a mix of these two functions \parencite{Thompson1989}. \textcite[123]{Croft1991} synthesizes ideas from this body of research and offers the following revised definitions instead:

\begin{itemize}
  \item the act of \dfn{reference} identifies a referent and establishes a cognitive file for that referent
  \item the act of \dfn{predication} ascribes something to a referent
  \item the act of \dfn{modification} enriches the cognitive image of the referent with an additional feature
\end{itemize}

\noindent The exact pragmatic function chosen for any given mention of a concept is then just a matter of how the speaker chooses to portray or construe that concept—whether as a referent, predicate, or modifier \parencite[100]{Croft1991}; as \citeauthor{CroftLier2012} note, \textquote[{\cite[63]{CroftLier2012}}]{apparent instances of \enquote{fuzziness} are actually variable construals}.

With this understanding of discourse functions in mind, we can restate the thesis of typological markedness theory as applied to lexical categories: Noun, verb, and adjective are epiphenomenal markedness patterns that arise from the use of different semantic prototypes (objects, actions, and properties) in different discourse functions (reference, predication, modification). Uses of these semantic classes in non-prototypical functions are typologically marked. As mentioned, there are three ways in which non-prototypical uses can be marked: structurally, behaviorally, and/or frequentially.

The first type of marking, \dfn{structural coding} or \dfn{formal marking}, refers to the fact that non-prototypical uses of lexical items are at least as formally marked as prototypical ones. Structural coding in this context refers specifically to \textquote[{\cite[62]{CroftLier2012}}]{dedicated formal markers in a specific language that indicate a lexeme's syntactic function}. \figref{fig:typological-pos-prototypes} is a schematic representation of some of the formal realizations of these markedness patterns. It indicates the different morphosyntactic means that languages tend to develop for marking each of the non-prototypical uses of lexical items. For instance, participle constructions are one way that languages have of indicating the non-prototypical case of an action word being used for modification.

\begin{figure}[h!]
  \centering
  \includegraphics[width=\linewidth]{typological-pos-prototypes.png}
  \caption[Typological prototypes for noun, verb, and adjective]{Typological prototypes for noun, verb, and adjective (adapted from \textcite[89]{Croft2000} and \textcite[62]{Lier2012})}
  \label{fig:typological-pos-prototypes}
\end{figure}

The second way in which non-prototypical uses of lexical items can be marked is in terms of their \dfn{behavioral potential}, that is, the range of combinatorial possibilities for that lexical item. This is most clearly illustrated with an example from inflection: in many languages, property words used in predicate constructions are limited in their inflectional possibilities. In \idx{Munya}, for example, property words functioning as predicates cannot inflect for person and number of the subject, and cannot take the imperfective marker, perfective marker, or direct evidential marker \parencite[96--97]{Bai2019}. The only grammatical markers allowed in property predication clauses are the stative aspect marker, a clause-final particle, and an egophoric marker. \posscitet{HopperThompson1984} study of the discourse functions of different parts of speech is largely a study of behavioral potential. They conclude that \textquote[{\cite[703, abstract]{HopperThompson1984}}]{the closer a form is to signaling this prime [prototypical] function, the more the language tends to recognize its function through morphemes typical of the category—e.g. deictic markers for [Nouns], tense markers for [Verbs].}. Croft advances a cognitive explanation for these behavioral markedness patterns:

\blockquote[{\cite[86]{Croft1991}}]{In general, only the core members of the syntactic category will display the full grammatical behavior characteristic of their category because only they have all the semantic characteristics that the characteristic inflections tap into. This is to say that the inflectional categories of the major syntactic categories have been \enquote{tailored} to their semantically core members. This is an example of a processing constraint: languages inflect only for those properties that are of relevance to core members of the category; they do not inflect for properties of peripheral members of the category that are not of relevance to the core members of the category.}

Non-prototypical uses of lexical items also exhibit a \dfn{semantic shift} in their meaning towards the semantic class prototypically associated with the discourse function they are found in \parencites[96]{Croft2000}[73]{Croft2001b}[68]{CroftLier2012}. I have already discussed the semantic shifts that occur in functional expansion in some detail in \secref*{sec:2.3.3.2}. \textcite[73]{Croft2001b} observes the following empirical pattern about semantic shifts in cases of functional expansion:

\blockquote[{\cite[73]{Croft2001b}}]{If there is a semantic shift in zero coding of an occurrence of a word (i.e. flexibility) in a part-of-speech construction, even if it is sporadic and irregular, it is always towards the semantic class prototypically associated with the propostional act function.}

These semantic shifts are caused by a combination of conventionalization and \dfn{coercion}, wherein the meaning of the constructional context is imposed on the meaning of the lexical item \parencites{Pustejovsky1991}[69, 108]{Croft1991}[252]{PantherThornburg2007}{AudringBooij2016}. For example, predicate nominals (where an object word is used in a predicate construction) involve coercion of lexical items from denoting objects to denoting classifying or equational relations \parencite[69]{Croft1991}. In \idx{Nuuchahnulth}, for instance, nominal predicates are always semantically durative and interpreted as either existential, classifying, or identifying expressions \parencite[47]{Nakayama2001}.

The final way in which lexical items used in atypical functions may be marked is in terms of their frequency. \textcite[59, 87]{Croft1991} also refers to this as \dfn{textual markedness}. Frequential markedness predicts that lexical items are used more frequently in their prototypical functions than in non-prototypical ones. This means that object words should be most frequent in their use in referring constructions, and that referring constructions should most frequently denote objects \parencite[87]{Croft1991}.

The field of linguistics has accumulated a good deal of empirical evidence in support of the typological markedness theory of lexical categories. \textcite{Croft1991} provides empirical evidence from 12 languages for each of these markedness patterns. \textcite{Dixon1977} also provides evidence of typological markedness patterns as they relate to property words, using a combination of structural and behavioral evidence. As mentioned, \posscitet{HopperThompson1984} study also provides empirical support from a variety of languages for markedness in terms of behavioral potential. \textcite{Stassen1997} is a massive study of intransitive predication in 410 languages, demonstrating the marked behavior of non-action words when used in predicate constructions.

Having explicated the basic tenets of typological markedness theory, I now turn to reframing the concept of lexical polyfunctionality in a way that utilizes this framework.

\section{Lexical polyfunctionality: A functionalist definition}
\label{sec:2.5}

In \secref*{sec:2.5.1} I provide a functional definition of \dfn{lexical polyfunctionality} within the framework of typological markedness asymmetries. Lexical polyfunctionality is understood synchronically as the functional diversity of a lexical item. By contrast, \dfn{functional expansion} is the diachronic process whereby a lexical item expands into new discourse contexts and becomes polyfunctional. \secref*{sec:2.5.2} lays out a theory of functional expansion based on \dfn{conventionalization}. It also discusses some of the known diachronic pathways by which lexical polyfunctionality arises.

\subsection{Lexical polyfunctionality}
\label{sec:2.5.1}

Within the framework of typological markedness asymmetries, we can provide a structural definition of lexical polyfunctionality as follows:

\begin{description}
  \item[lexical polyfunctionality] The use of a lexical item (root, stem, or inflected word) in more than one discourse function (reference, predication, or modification) with zero coding.
\end{description}

\noindent This definition qualifies as a valid \dfn{comparative concept} in the sense of \textcite{Haspelmath2010a} because it is couched in terms of universal \emph{functions} rather than language-specific \emph{structures} \parencite{Croft2016}. It also has the advantage of being intentionally equivocal with respect to the morphological level (root, stem, or inflected word) at which the polyfunctionality is realized, and with respect to the lexical and cognitive unity of the item. In some cases when a single lexical form appears in more than one discourse function, speakers may have a close cognitive association between the two uses, and so those uses are highly lexically unified. This is most likely the case for the predicative and referential uses of the word \txn{run} in the phrases \txn{I run every morning} and \txn{I'm going for a run} respectively. In other cases, speakers may have little to no awareness of the diachronic connection between uses of a form. For example, the use of \txn{run} in the sense of \txn{to run a print job} is extremely distant from the prototypical \enquote{fast pedestrian motion} sense in the semantic network for that form \parentext{\cite[74]{Gries2006}; see also \figref{fig:semantic-map-run}}. It is unlikely that these two senses are closely cognitively connected by most speakers, even though they both share a predicating function.

This definition is also neutral with respect to the degree of semantic shift that occurs between polyfunctional uses of a lexical item. The definition of polyfunctionality simply delimits those cases where a language does not provide formal indicators of discourse function, specifically the constructions that use a zero coding strategy (no derivational affixes, no copulas, no overtly marked forms like gerunds, participles, infinitives, etc.). It is largely a structural definition. Once such cases are delimited structurally, it becomes possible to examine the semantic relationships involved within them. Indeed, describing the semantic shifts involved in polyfunctional cases is an important descriptive \foreign{desideratum}.

Moreover, by focusing on just those cases where use across multiple discourse functions is zero-coded, we are then in a position to make explicit comparisons between cases of polyfunctionality and cases of overt derivation. For example, an intriguing question is whether the semantic shifts that occur in polyfunctional cases differ in principled ways from those of overt derivation. This dissertation does not explore that question, but \textcite[165]{Mithun2017} shows that for \idx{Central Alaskan Yup'ik} the types of semantic relationships between polyfunctional uses of words mirror those between basic and derived words. This would suggest that functional expansion follows the same principles as overt derivation, but much more research is needed in this area.

A frequent question that arises in discussions of lexical polyfunctionality is whether any two given uses of a form constitute part of the same lexeme or distinct lexemes—in essence, whether those uses constitute heterosemy or polysemy. I call this the \dfn{problem of lexical unity}. Here, I would like to propose that the question of lexical unity is an unproductive one that dichotomizes a gradient phenomenon. Moreover, making a determination of the lexical unity of a given bundle of uses of a form is not necessary to an understanding of lexical polyfunctionality and the diachronic process by which it emerges.

Let us consider the common arguments for analyzing polyfunctionality as conversion / heterosemy. Pointing out that functional expansion involves both semantic shifts and functional gaps is generally intended to show that lexemes cannot be truly flexible in the sense of being multifunctional (\secref{sec:2.3.1.3}) or precategorial (\secref{sec:2.3.1.4}), and/or that uses of the same lexical item for different discourse functions should therefore be considered cases of conversion—that is, homonymy or heterosemy. As an argument against flexibility, this criticism is a sound one. As justification for a categorical distinction between heterosemy and polysemy, however, there are two important issues with this argument.

The first and most important point is that it creates a false dichotomy between homonymy and polysemy, when in fact the two phenomena are opposite endpoints on a continuum. Debates over the lexical unity of an item arise from an Aristotelian desire to neatly sort those uses into distinct lexemes, when in fact reality is much more complex. If this problem sounds familiar, that is because it is the same methodological problem that arises when trying to exclusively categorize lexical items into different classes that was discussed in \secref*{sec:2.2.1}. The complex adaptive nature of language makes categorical classification at either level impossible.

As discussed in \secref*{sec:2.3.1.4}, we know from cognitive research that mental categories are prototypal, and that the meanings of words display a polycentric, family resemblance structure. Two senses of a lexical item are often related only tenuously through a network of intervening semantic connections or meaning chains. \textcite{Langacker1988} calls this the \dfn{network model} of category structure. \citeauthor{Taylor2003} points out that \textquote[{\cite[167]{Taylor2003}}]{[o]ne consequence of adopting the network model is that the question of whether a lexical item is polysemous turns out to be incapable of receiving a definite answer.}

Over time, as this lexical network expands, the meanings of a lexical item can diverge so drastically that speakers no longer have a direct cognitive association between them. \citeauthor{Mithun2000} exemplifies this nicely for both \idx{Cayuga} and \idx{English}. Discussing morphological Verbs used as referents in Cayuga, she notes the following:

\blockquote[{\cite[413]{Mithun2000}}]{If asked the meaning of \txn{kaǫtanéhkwih} [lit. \tln{it hauls logs}], Cayuga speakers normally respond \tln{horse}. Though it has the morphological structure of a verb, it has been lexicalized as a nominal. The literal meanings of many verbal nominals are still accessible to speakers, but the origins of others have faded, and speakers express surprise at discovering them. Similarly, when asked \enquote{What would you like for breakfast?}, most English speakers do not think about breaking their night-time fast, though they can usually be made aware of the literal meaning of \txn{breakfast}.}

\noindent Lexicalization is a process and a continuum. Words can be lexicalized in new discourse functions to varying degrees. The first use of a lexical item in a new discourse function is innovative; each subsequent use then contributes further to its conventionalization in that function \parencite[166]{Mithun2017}. This point is discussed further in \secref*{sec:2.5.2} below.

Pointing out that functional expansion often creates idiosyncratic and unpredictable meanings essentially amounts to saying that senses of lexical items can be highly divergent. This point is not in itself an argument for or against lexical unity. polyfunctional items may sit anywhere on the continuum from having closely connected, productive and predictable meanings, to having extremely divergent, idiosyncratic, and unpredictable meanings. Importantly, this is not a special fact about polyfunctional items—it is simply true of words generally.

The second significant problem with using semantic shift as a diagnostic for heterosemy is that it proves too much. If semantic shift is taken as evidence against the lexical unity of polyfunctional items, then it must also be taken as evidence against the lexical unity of non-polyfunctional items. Put simply, semantic shift is an analytical problem for all words, not just polyfunctional ones.

This fact becomes clear when we ask, \enquote{What counts as a semantic shift? Just how \enquote{large} of a change in meaning (if it were even possible to quantify such a thing) does a semantic shift require?} To illustrate this problem, consider the semantic contribution of plural marking crosslinguistically. In the canonical case, plural marking is considered inflectional rather than derivational \parencite[2]{Corbett2000}, meaning that it does not create a new lexeme. Instead, it modifies the meaning of the existing lexeme slightly, in line with the classic distinction between inflection vs. derivation. However, there are numerous cases of lexical items in English with more or less drastic differences in meaning between the singular and plural, and/or senses that are only available in one of the two numbers. Consider the examples in \exref{ex:2.19}.

\begin{exe}
  \ex\label{ex:2.19}
  \exinfo{\idx{English} (Indo-European > Germanic)}
  \begin{xlist}

    \ex
    \begin{tabular}[t]{ p{0.75in} l }
      \txn{air}  & \tln{atmosphere}\\
      \txn{airs} & \tln{affected manners}\\
    \end{tabular}

    \ex
    \begin{tabular}[t]{ p{0.75in} l }
      \txn{arm}  & \tln{upper limb; anything resembling a limb}\\
      \txn{arms} & \tln{weapons, firearms}\\
    \end{tabular}

    \ex
    \begin{tabular}[t]{ p{0.75in} l }
      \txn{blind}  & \tln{unable to see}\\
      \txn{blinds} & \tln{screen for a window}\\
    \end{tabular}

    \ex
    \begin{tabular}[t]{ p{0.75in} l }
      \txn{custom}  & \tln{tradition; socially accepted behavior}\\
      \txn{customs} & \tln{department which levies duties on imports}\\
    \end{tabular}

    \ex
    \begin{tabular}[t]{ p{0.75in} l }
      \txn{force}  & \tln{strength, energy}\\
      \txn{forces} & \tln{collection of military units}\\
    \end{tabular}

    \ex
    \begin{tabular}[t]{ p{0.75in} l }
      \txn{good}  & \tln{excellent, high quality}\\
      \txn{goods} & \tln{merchandise or possessions}\\
    \end{tabular}

    \ex
    \begin{tabular}[t]{ p{0.75in} l }
      \txn{manner}  & \tln{way of doing something}\\
      \txn{manners} & \tln{social conduct; socially acceptable conduct}\\
    \end{tabular}

    \ex
    \begin{tabular}[t]{ p{0.75in} l }
      \txn{spectacle}  & \tln{visually striking performance or display}\\
      \txn{spectacles} & \tln{pair of glasses}\\
    \end{tabular}

    \ex
    \begin{tabular}[t]{ p{0.75in} l }
      \txn{wood}  & \tln{fibrous material in the trunk of trees or shrubs}\\
      \txn{woods} & \tln{area of land covered with trees}\footnote{In some dialects of English, this sense is available in the singular as well.}\\
    \end{tabular}

  \end{xlist}
\end{exe}

Semantic shifts for plural marking in \idx{English} are not limited to just a handful of specific lexical items. Generic uses of the plural as in the expression \txn{foxes are cunning} create a semantic shift away from a concrete entity (\txn{a/the fox}) to a generic, unperceivable one—a use which strays from the prototypical function of nouns as concrete perceptible entities \parencite[708]{HopperThompson1984}.

As with polyfunctional items, the semantic shifts that occur with plural marking can become so substantial that speakers no longer cognize the morphological singular and plural as members of the same lexeme. Such is the case in the historical development of \txn{brother} vs. \txn{brethren} in \idx{English}. The word \txn{brethren} became so strongly conventionalized with its religious meaning in the plural that it was independently lexicalized as a plural-only (\foreign{plurale tantum}) noun, and the original plural underwent renewal with the emergence of the form \txn{brothers}. This is exactly the kind of lexicalization process that occurred for many morphological verbs reanalyzed as nouns in \idx{Cayuga} and many other North American languages.

A similar example comes from \idx{Chitimacha}, which has a pluractional marker \txn{-ma} indicating verbal number (plural agents, plural patients, or repeated action). In some cases the use of \txn{-ma} is purely compositional, so that it can be considered merely an inflectional marker of verbal number. In other cases \txn{-ma} so significantly alters the meaning of the word that it must be considered derivational. Compare the uses of \txn{-ma} in each of the pairs of verbs in \exref{ex:2.20} (note that \exref{ex:2.20b} and \exref{ex:2.20c} are phrasal verbs with a preverbal particle).

\begin{exe}
  \ex\label{ex:2.20}
  \exinfo{\idx{Chitimacha} (isolate)}
  \begin{xlist}

    \ex\label{ex:2.20a}
    \begin{tabular}[t]{ p{1in} l }
        \txn{kow-}   & \tln{call}\\
        \txn{kooma-} & \tln{call multiple people}\\
    \end{tabular}

    \ex\label{ex:2.20b}
    \begin{tabular}[t]{ p{1in} l }
        \txn{qapx cuw-}   & \tln{come back; go about}\\
        \txn{qapx cuuma-} & \tln{travel; wander}\\
    \end{tabular}

    \ex\label{ex:2.20c}
    \begin{tabular}[t]{ p{1in} l }
        \txn{qapx qiy-}   & \tln{turn together; mix, join}\\
        \txn{qapx qiima-} & \tln{give a prayer, benediction; perform magic}\\
    \end{tabular}

  \end{xlist}
  \exsourcebelow{Swadesh1939a}
\end{exe}

\noindent In \exref{ex:2.20a}, the use of \txn{-ma} is entirely compositional. The presence of \txn{-ma} indicates that the verb has a plural patient argument. In \exref{ex:2.20b}, the use of \txn{-ma} is still arguably compositional, though perhaps somewhat lexicalized given the high frequency with which the stem appears in the texts. \tln{travel, wander} could reasonably be interpreted as a continued repetition of \tln{go about}. In \exref{ex:2.20c}, however, \txn{qapx qiima-} has become lexicalized with a new meaning not directly related to that of \txn{qapx qiy-}. The diachronic connection between the two meanings is that prayers and magical incantations were traditionally accompanied by circling gestures with the arms. \txn{qapx qiima-} originally meant \tln{turn/circle around repeatedly}, but over time lexicalized with its new religious meaning in the pluractional, \tln{give a prayer, benediction}. This lexicalization process parallels that of \txn{brethren} in \idx{English}. Such a range of inflectional vs. derivational uses of pluractionals is quite common crosslinguistically \parencites{Mithun1988}{Mattiola2020}.

Finally, there are many languages which do not typically mark plurality on nouns \parencite{Dryer2013}, and yet have senses available in semantically plural contexts but not singular ones (where the semantic number can be understood from the clausal context, usually through verbal number marking). For example, the word \txn{soq} in \idx{Chitimacha} may mean \tln{foot} or \tln{paw} in a singular context and \tln{feet} or \tln{paws} in a plural context, but may also mean \tln{tracks} (e.g. animal tracks) in a plural context—a significant and idiosyncratic shift in meaning, and one that is both language-specific and item-specific and thus conventional. This use constitutes a \emph{morphologically unmarked semantic shift} in the meaning of the word, just as idiosyncratic meanings of words in cases of functional expansion also constitute morphologically unmarked semantic shifts. If we take such unmarked semantic shifts as evidence against lexical unity in the cases of polyfunctional items, then we must also say that the \tln{foot} and \tln{tracks} meanings of \txn{soq} constitute two distinct lexemes as well.

What then are we to make of the often extremely divergent meanings that occur within polyfunctional items? We should first loosen the requirement that the meaning of a lexical item be unitary. Word meanings are \dfn{polycentric}, where different senses of an item are related through \dfn{meaning chains} rather than all through a single, central member \parencite[110]{Taylor2003}. This is often referred to as a \dfn{family resemblance} structure for categories. The difference between monocentric and polycentric categories is illustrated schematically in \figref{fig:monocentric-vs-polycentric}. In both diagrams, each letter A–E represents a sense of a lexical item. In the monocentric case, all the senses of the lexical item are related through its core sense A. In the polycentric case, senses A and E are related only through their intervening connections.\footnote{The terms \dfn{monothetic} and \dfn{polythetic} are sometimes used for this distinction instead \parencite[146]{LewandowskaTomaszczyk2007}.}

\begin{figure}[h!]
  \centering
  \includegraphics[width=\linewidth/2]{monocentric-vs-polycentric.png}
  \caption{Monocentric vs. polycentric categories}
  \label{fig:monocentric-vs-polycentric}
\end{figure}

Recognizing that word meanings are polycentric addresses the problem of lexical unity because it shows that the disparate senses of a lexical item can be related without having to share any core component of their meanings. The use of a lexical item in a certain context then profiles one of these senses over others.

The issue of lexical unity is exactly analogous to the problem of lumping vs. splitting in the context of lexical categories. The Radical Construction Grammar solution to this problem is to abandon the commitment to larger groupings of items (the major lexical categories) and acknowledge that languages consist of an interconnected network of smaller items (constructions) instead \parencite{Croft2001b}. This approach has the major advantage of sidestepping unproductive debates about the existence or unity of lexical categories in particular languages, and shifts the focus instead to understanding the relationships and patterns among individual constructions. This is precisely what I propose to do for lexemes as well. If we abandon the idea that all the meanings associated with a form must be in some way grouped into lexemes based on their morphosyntactic contexts of occurrence, and instead see meaning as a network of more-or-less-distantly related senses, we sidestep unproductive debates regarding homonymy vs. polysemy, and can instead focus on the relationships and patterns among the various senses associated with that form.

In sum, the existence of idiosyncratic semantic shifts does not invalidate the concept of polyfunctionality, even though it does provide evidence against flexible analyses of polyfunctionality.

\subsection{Functional expansion}
\label{sec:2.5.2}

This section addresses the question of how lexical polyfunctionality arises in language. I begin by defining the diachronic process which gives rise to lexical polyfunctionality as \dfn{functional expansion}, and then describe how lexical flexibility and conventionalization interact to create polyfunctional lexical items over time. I conclude by examining a few specific diachronic pathways by which a language can either increase or decrease its overall degree of lexical polyfunctionality.

As we have seen, a great abundance of evidence shows that the meaning of any given combination of form and discourse function is a matter of convention, and often highly idiosyncratic (\secref{sec:2.3.2.4}; \secref{sec:2.3.3.2}). This suggests that polyfunctional items are not in fact flexible in the sense that speakers can use any lexical item for any discourse function and expect hearers to be able to infer their meaning from context. We know that item-specific gaps in usage exist.

Yet we know that speakers \emph{do} in fact use words flexibly, or else it would not be possible for functional shift to happen in the first place. A word like \txn{friend} cannot have gotten its predicative use meaning \tln{add someone as a contact on social media} unless at some point one or more speakers began creatively used the word \txn{friend} as a predicate rather than a referent. Because this early innovation would have preceded the point at which its use had become widely adopted in the broader English-speaking community, those early speakers had to have relied on the ability of their interlocutors to infer their intended meaning from context. In other words, they used the word \txn{friend} flexibly. Flexible uses of words occur because of speakers' need to construe concepts in different ways—as objects, actions, or properties. The semantic shifts that occur during functional expansion are the result of coercion by the new constructional context. If there is sufficient uptake of this novel use by other members of the community, that resultant meaning then becomes the conventionalized meaning for that particular form in that particular discourse function \parencite[108]{Croft1991}.

These innovative or flexible uses are restrained by existing linguistic conventions \parencites[Ch.~4]{Croft2000}[102]{CroftCruse2004}. For example, the presence of an existing, synonymous form in a language will \dfn{pre-empt} or \dfn{block} certain flexible uses of a lexical item \parencite{ClarkClark1979}. The reason that English speakers do not use \txn{hospital} as a predicate meaning \tln{to place in a hospital for medical care} is because there is already the synonymous form \txn{hospitalize}. However, in cases where the preexisting form is \emph{not} synonymous with the intended novel meaning of a form, that novel use is generally acceptable. Consider again the word \txn{friend}: prior to the rise of social networking platforms the use of \txn{friend} as a predicate was pre-empted by the existence of the word \txn{befriend}. This is because the only obvious contextual interpretation of \txn{friend} as a predicate at the time was \tln{to make friends with}, and this meaning is synonymous with that of \txn{befriend}. The predicative use of \txn{friend} was therefore blocked. After the advent of social networking sites, however, a new social context appeared in which another obvious contextual interpretation of \txn{friend} as a predicate became possible—\tln{to add someone as a contact on social media}. Since this meaning was no longer synonymous with the existing form \txn{befriend}, its use was no longer pre-empted.

Flexible uses of lexical items are also restrained by the cognitive limits on our ability to deal with ambiguity. If it were truly the case that any lexical item could be used in any discourse function at any time, it would scarcely be possible for hearers to interpret the intended pragmatic effects of each word. In order to stick around after those first innovative uses, polyfunctionality requires a degree of conventionalization. Innovative uses of words in new discourse functions are by no means guaranteed uptake in the community. Polyfunctional lexical items only arise when a new combination of form and discourse function is conventionalized in a community of speakers. Conventionalization in turn implies \emph{time}—it is a diachronic process. Thus lexical polyfunctionality can be understood as a \emph{synchronic} pattern resulting from the \emph{diachronic} process of functional expansion, where \dfn{functional expansion} is defined as follows:

\begin{description}
  \item[functional expansion] A diachronic expansion in the conventionalized range of uses of a lexical item (root, stem, or inflected word) into a new discourse function (reference, predication, or modification) with zero coding for that new function.
\end{description}

Functional expansion is a multi-stage process in which an initial, innovative use of a lexical item for a new discourse function gradually becomes conventionalized as an accepted semantic meaning for that form. Each additional usage of the novel form-meaning combination gradually adds to its conventionalization; or, to use an apt metaphor: \textquote[Croft, p.c.]{Conventionalizations are sedimentations of innovation events over the history of the speech community[.]}. A novel usage gradually spreads along the continuum from the pragmatics of the immediate discourse context to the conventions of the speech community as a whole \parencites{ClarkClark1979}[100]{Croft2000}. This gradual change is sometimes even rapid, as in cases where technological innovations foster frequent use of new form-meaning combinations, such as the words \txn{friend} and \txn{text} used as predicates.

The first stage of the process of functional expansion is, as stated above, the initial innovative use of a lexical item in a new discourse function. This phenomenon is not rare or exceptional by any means. Each time a hearer encounters a novel use of a lexical item for the first time, they must accomplish the difficult task of discerning its meaning. This is no less true for flexible uses as it is for non-flexible uses. \emph{Every} use of a word is an instance of functional expansion because every use of a word is always in a slightly different discourse and social context than the one before. As \citeauthor{Croft2000} notes, \textquote[{\cite[104]{Croft2000}}]{The chief reason why even conventional language use is innovative is that there cannot be a word or phrase to describe every experience that people wish to communicate.}. Moreover, the meaning of a word in a given context is highly socially and situationally dependent, and that context can change completely from one utterance to the next. \citeauthor{Croft2000} makes this point well:

\blockquote[{\cite[104]{Croft2000}}]{The degree to which ordinary language use, apparently conforming to linguistic convention, requires nonconventional coordination devices makes it clear that virtually all language use involves nonconventional coordination. […] In other words, virtually every noun, verb and adjective in virtually every sentence requires nonconventional coordination in order to establish reference to the specific object, property and event being talked about.}

\noindent Every token of a word thus necessarily appears in a new pragmatic context, and that pragmatic context slightly shapes its meaning \parencite[99--105]{Croft2010}. Language use \emph{is} language change.

After this initial stage of innovation, the novel meaning gradually conventionalizes first as part of the pragmatic meaning of the form, and then finally part of the semantic meaning of the form. Of course, the spread of any given innovation might cease at any one of these stages. While functional innovation in everyday speech is ubiquitous, only a small fraction of these innovative uses ever become fully conventionalized as part of the semantic meaning of a word, creating polyfunctional items.

To summarize, polyfunctionality arises because speakers must use the limited linguistic resources at their disposal to convey an infinitude of experiences, and are thus constantly innovating in everyday language use. Most of these innovations never spread beyond the local discourse context in which they were first used, but occasionally a novel use is adopted by other speakers, and over time this usage may become conventionalized as part of the pragmatic and later the semantic meaning of the word.

With these understandings of lexical polyfunctionality and functional expansion in place, we are now in a position to reframe the major research question of this dissertation: How often, diachronically, have words in English and Nuuchahnulth become conventionalized in new discourse functions with zero coding? Answering this question generates new research questions in turn, just a few of which are mentioned below.

\begin{itemize}
  \item Why do some languages have so many more words that underwent functional expansion than other languages? What diachronic processes give rise to rampant lexical polyfunctionality?
  \item Are there semantic commonalities to the lexical items which frequently undergo functional expansion across languages?
  \item Are the semantic shifts in cases of functional expansion more productive or predictable for certain semantic classes of words than others?
\end{itemize}

This understanding of lexical polyfunctionality also makes a prediction regarding the interaction of polyfunctionality and frequency. Though I know of no studies on the matter, it seems empirically true that functional \emph{expansion} is drastically more common than functional \emph{shift}. That is, it is relatively common for a lexical item to expand from one disourse function into another, but it is rare for a lexical item to undergo a wholesale shift from one function to another. Functional expansion therefore increases the number of contexts that a lexical item appears in. It is thus a reasonable hypothesis that the overall frequency of the lexical item will increase as well, being the sum of the individual senses across different discourse functions. More functionally diverse items may have a higher overall frequency than less functionally diverse items. Of course, this is merely a hypothesis. It may be that functional expansion into new contexts is offset by a concommitant decrease in frequency for the original function, or that this varies depending on the lexical item. These questions are explored quantitatively in \secref*{sec:4.5}.

Returning to the first question above: If lexical polyfunctionality is the result of a diachronic process, it should be possible to enumerate some of the specific pathways which give rise to it. Here I will mention just a few. One pathway is insubordination, whereby subordinate clauses in a language are reanalyzed as main clauses \parencites{Evans2007}{Mithun2008}{EvansWatanabe2016} (see also \secref*{sec:2.3.2.2}). Insubordination frequently results in formal similarities between noun phrases and verb phrases, and this formal ambiguity can abet the functional expansion of lexical items from referential to predicative uses and vice versa.

A second pathway to lexical polyfunctionality is relexicalization (or more precisely, reconventionalization). This is the process that occurred in the case of morphological verbs being reanalyzed as nouns in many North American languages (see \secref{sec:2.3.2.3}) and certain \idx{English} plurals like \txn{brethren} or \txn{arms}. In these cases, the conventionalized meaning associated with the form changed (e.g. from \idx{Cayuga} \tln{it hauls logs} to \tln{horse}), and that meaning is reflected by its use in the new discourse context.

A third pathway is topicalization, exemplified in the \idx{Wakashan} family. \textcite[122, 142]{Jacobsen1979} observes the formal similarity between the Definite Article and the Third Person Singular Indicative markers in Wakashan languages, and argues for a diachronic connection between the two. It is likely that cleft constructions such as \tln{it was the dog that ran} became so common that speakers started to reanalyze the topicalized cleft as a definite noun phrase, \tln{the dog}, thereby creating a formal similarity between referring expressions and predicating expressions.

Each of these pathways results in the functional expansion of lexical items into new discourse contexts with no new overt structural coding. Of course, functional expansion can also occur without any other accompanying grammatical changes. This happens in any instance where speakers simply use stems in new discourse functions. Lexical polyfunctionality is the natural and expected result of the fact that non-prototypical uses of lexical items are \emph{not} always structurally marked—as allowed for by the fact that typological markedness is implicational and not absolute—even if they are marked in other ways. The use of additional structural coding in cases of functional expansion is not obligatory, but merely a statistical tendency. Lexical polyfunctionality occupies the theoretical space where structural coding asymmetries fail to apply. The result is that a single form (or set of forms) can serve both prototypical and non-prototypical functions, and it is these forms that I am here calling \dfn{polyfunctional}.

When viewed in this light, \emph{lexical polyfunctionality is not so much a problem as it is a design feature of language}. The presence of lexical polyfunctionality should be \emph{expected} in every language, not treated as exotic. The cognitive-typological approach outlined in this chapter inverts the lexical polyfunctionality question: the interesting question is not why some languages fail to make distinctions in parts of speech (thereby framing lexical polyfunctionality as a \emph{deficit} in a way similar to how colonial researchers framed non-Indo-European languages as deficient in their parts of speech), but rather why languages develop specialized constructions for different discourse functions in the first place \parentext{see \textcite{Gil2012} for an attempt to answer this question for predication}. Lexical polyfunctionality exists in any area of the grammar where specialized discourse-function–indicating morphology has yet to develop, or where such distinctions have been leveled as a result of diachronic changes. Lexical polyfunctionality should therefore be considered the default state of affairs for language. Gil \parencites*{Gil2005}{Gil2006} has in fact argued, partially on the basis of data from highly polyfunctional languages, that early human language must have been \dfn{isolating-monocategorial-associational} before the development of dedicated function-indicating morphology.

The idea that the \enquote{natural state} of language is monocategorial or acategorial would seem to conflict with the point made above that lexical polyfunctionality can result from diachronic processes, but the two positions are not mutually exclusive. Languages develop strategies for indicating different discourse functions, but languages are also subject to counteracting pressures. This is a classic case of competing motivations: on the one hand, the frequency with which speakers need to perform the discourse functions of reference, predication, and modification all but ensures the development of strategies dedicated to indicating those functions; on the other hand, speakers need to construe states of affairs in various ways—as objects, actions, or properties—creating pressures which have the potential to level or cut across those formally marked distinctions. Reconventionalization and the reanalysis of cleft constructions could both be viewed as diachronic processes motivated by this latter pressure.

In sum, lexical polyfunctionality is a natural result of the cognitive and diachronic forces at work in language. Defining lexical polyfunctionality in terms of typological markedness (or more accurately, the lack of formal marking for otherwise marked uses) provides a crosslinguistically applicable definition of the phenomenon which avoids methodological opportunism while still recognizing that lexical polyfunctionality requires some degree of semantic shift and conventionalization. This cognitive-typological definition of lexical polyfunctionality is a key theoretical contribution of this dissertation. With this definition in place, the remainder of this dissertation turns to exploring the prevalence of lexical polyfunctionality in \idx{English} and \idx{Nuuchahnulth}.
