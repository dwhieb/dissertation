\chapter{Background}
\label{ch:background}

\blockquote{The focus of this chapter is to explain the concept of lexical flexibility, consider its criticisms, and offer a more robust, functionally-grounded definition instead. I first briefly describe how flexible approaches to lexical categories developed as a response to weaknesses in traditional theories of parts of speech. I then survey the landmark studies and important findings on lexical flexibility, along with criticisms of this research. Following that, I summarize approaches to lexical categories from several functionalist perspectives—cognitive linguistics, typology, and construction grammar. I conclude by offering a revised formulation of lexical flexibility which is more in line with this functional research.}

\section{Introduction: Approaches to lexical flexibility}
\label{sec:2.1}

The field of linguistics as a whole, and the subfield of typology in particular, is undergoing a radical shift in how we understand lexical categories, along primarily two dimensions. The first dimension is our understanding of what lexical categories are a property \emph{of}. Early researchers viewed categories as universal properties of both language and languages \addcite{Haspelmath on g-language vs. p-language; add these terms in parentheses}. I call this the \dfn{universalist} position. After Boas, many researchers then came to view categories as language-specific, with patterned similarities across languages. I call this the \dfn{relativist} approach. Most recently, some researchers view categories as typological patterns rather than properties of any particular language. This is the \dfn{typological} position, and the one I adopt here.

The second dimension of historical change in linguistic theories of categories is in the \emph{nature} of the categories themselves. In the Classical tradition, categories were thought to be categorical and well-defined by a set of necessary and sufficient conditions (in the tradition of Aristotle). After the cognitive turn in the 1980s, many linguists came to view categories as prototypal, with some members of a category being more central, or better exemplars, than others. Cognitive research into the nature of idioms then led to the development of construction grammar, which sees language as consisting of a network of constructions rather than monolithic categories. I adopt a constructional approach to categories in this thesis.

These theoretical paradigm shifts are summarized in \exref{ex:2.1}. At each stage of development, there has not been a wholesale displacement of previous theories \addcite{Kuhn}. There are still many who regard word classes as universal and categorical, and the typological-constructional approach is still nascent.

\begin{exe}
  \ex\label{ex:2.1}
  \begin{xlist}
    \ex universal > language-specific > typological
    \ex categorical > prototypal > constructional
  \end{xlist}
\end{exe}

This chapter gives a synopsis of these earlier theoretical positions and shows how research on lexical flexibility developed in recognition of the shortcomings of traditional approaches (\secref{sec:2.2}). \secref*{sec:2.3} summarizes the key concepts and findings that have arisen from the research on lexical flexibility. Such research, however, is not without its own shortcomings. \secref*{sec:2.3} also presents the main criticisms that have been leveled against flexible analyses of word classes. \secref*{sec:2.4} then presents an alternate, functionally-oriented approach---the typological-constructional perspective. The final section of this chapter (\secref{sec:2.5}) then applies this functional perspective to formulate an improved definition of lexical flexibility.

\section{Traditional approaches}
\label{sec:2.2}

\section{Flexible approaches}
\label{sec:2.3}

\section{Functional approaches}
\label{sec:2.4}

\section{Lexical flexibility: A functional definition}
\label{sec:2.5}
