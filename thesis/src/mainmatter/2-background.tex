\chapter{Background}
\label{ch:background}

\section{Introduction: Approaches to lexical flexibility}
\label{sec:2.1}

\section{Traditional approaches}
\label{sec:2.2}

\section{Flexible approaches}
\label{sec:2.3}

\textcite{Creissels2017} is a careful lexicon-based of flexibility in \idx{Mandinka} (Mande). While Mandinka has nominal and verbal constructions that allow the predicative and referring functions of words to be distinguished unambiguously, it is not as easy to separate word stems themselves into similar classes, since no Mandinka lexemes are used exclusively in verbal constructions—all Mandinka lexemes may occur in nominal constructions as well. While Creissels does not dispute this fact, he shows that there is a crucial distinction to be made between two classes of word stems: 1) those whose nominal use is predictable and therefore analyzable as a case of \enquote{morphologically unmarked nominalization} (zero-marked conversion) from one category (verb) to another (noun)—these are always event nominalizations; and 2) those whose meaning in nominal constructions is idiosyncratic and therefore not predictable. Creissels calls the former \dfn{verbal} words and the latter \dfn{verbo-nominal}. He states that both word classes exhibit categorial flexibility, just of different natures. There is also a small set of nominal words used marginally as verbs. These cases are always semantically predictable. Even individual senses of a word can sometimes show varying behavior as to their flexibility. One sense of a word may allow for flexible uses, with others senses may not. Although Creissels' study unfortunately does not provide counts of the different stem types, it nonetheless adds to our understanding of lexical flexibility by showing how it may have varied realizations, within a single language or even a single word.

\section{Functional approaches}
\label{sec:2.4}

\section{Lexical flexibility: A functional definition}
\label{sec:2.5}
