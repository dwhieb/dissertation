\subsubsection{Counterarguments}
\label{sec:2.3.3.4}

Pointing out that functional expansion involves both semantic shifts and functional gaps is generally intended to show that lexemes cannot be truly flexible in the sense of being multifunctional (\secref{sec:2.3.1.3}) or precategorial (\secref{sec:2.3.1.4}), and that uses of the same lexical item for different discourse functions should therefore be considered cases of conversion—that is, homonymy or heterosemy. There are however two major problems with this argument.

The first is that it creates a false dichotomy between homonymy and polysemy, when in fact the two phenomena are opposite endpoints on a continuum. Debates over the lexical unity of an item—that is, whether two uses of a lexical item are homonymous or polysemous—arise from an Aristotelian desire to neatly sort those uses into distinct lexemes, when in fact reality is much more complex. If this problem sounds familiar, that is because it is the same methodological problem that arises when trying to exclusively categorize lexical items into different classes. The complex adaptive nature of language makes categorical classification at either level impossible.

As discussed in \secref*{sec:2.3.1.4}, we know from cognitive research that mental categories are prototypal, and that the meanings of words display a polycentric, family resemblance structure. Two senses of a lexical item are often related only tenuously through a network of intervening semantic connections or meaning chains. \textcite{Langacker1988} calls this the \dfn{network model} of category structure. \citeauthor{Taylor2003} points out that \textquote[{\cite[167]{Taylor2003}}]{[o]ne consequence of adopting the network model is that the question of whether a lexical item is polysemous turns out to be incapable of receiving a definite answer.}

Over time, as this lexical network expands, the meanings of a lexical item can diverge so drastically that speakers no longer have a direct cognitive association between them. \citeauthor{Mithun2000} exemplifies this nicely for both \idx{Cayuga} and \idx{English}. Discussing morphological Verbs used as referents in Cayuga, she notes the following:

\blockquote[{\cite[413]{Mithun2000}}]{If asked the meaning of \txn{kaǫtanéhkwih} [lit. \tln{it hauls logs}], Cayuga speakers normally respond \tln{horse}. Though it has the morphological structure of a verb, it has been lexicalized as a nominal. The literal meanings of many verbal nominals are still accessible to speakers, but the origins of others have faded, and speakers express surprise at discovering them. Similarly, when asked \enquote{What would you like for breakfast?}, most English speakers do not think about breaking their night-time fast, though they can usually be made aware of the literal meaning of \txn{breakfast}.}

\noindent Lexicalization is a process and a continuum. Words can be lexicalized in new discourse functions to varying degrees. The first use of a lexical item in a new discourse function is innovative; each subsequent use then contributes further to its conventionalization in that function \parencite[166]{Mithun2017}.

Pointing out that functional expansion often creates idiosyncratic and unpredictable meanings essentially amounts to saying that senses of lexical items can be highly divergent. This point is not in itself an argument against flexible analyses. Flexible items themselves may sit anywhere on the continuum from having closely connected, productive and predictable meanings, to having extremely divergent, idiosyncratic, and unpredictable meanings. This is not a special fact about flexible items—it is simply true of words generally.

\textcite[73]{Croft2001b} expresses concern that ignoring semantic shifts in the analysis of flexible items overlooks important insights about how such semantic shifts are patterned (specifically, the universal fact that semantic shifts are always in the direction of the item's new discourse function; see \secref*{sec:2.3.3.2}). Given that so many researchers have indeed ignored semantic shift when arguing for flexible analyses, Croft's concern is warranted. However, the focus of this dissertation is not to argue for a flexible analysis (hence my preference for the term \txn{functional expansion} over \txn{lexical flexibility}). It is merely to give a structural definition of functional expansion that allows us to empirically investigate the extent of the phenomenon. It is entirely possible to define lexical flexibility / functional expansion in a way that both allows for the meaning of a lexical item to encompass multiple discourse functions while acknowledging that such multifunctional uses involve patterned semantic shifts. The way to do this is to ground the definition of lexical flexibility in the discourse functions of reference, predication, and modification rather than language-specific categories like Noun, Verb, and Adjective. I offer such a definition in \secref*{sec:2.5}.

The second significant problem with using semantic shifts as an argument against the existence of flexible lexemes is that it proves too much. If semantic shift is taken as evidence against the lexical unity of putatively flexible items, then it must also be taken as evidence against the lexical unity of non-flexible items. Put simply, semantic shift is an analytical problem for all words, not just flexible ones.

This fact becomes clear when we ask, \enquote{What counts as a semantic shift? Just how \enquote{large} of a change in meaning (if it were even possible to quantify such a thing) does a semantic shift require?} To illustrate this problem, consider the semantic contribution of plural marking crosslinguistically. In the canonical case, plural marking is considered inflectional rather than derivational \parencite[2]{Corbett2000}, meaning that it does not create a new lexeme. Instead, it modifies the meaning of the existing lexeme slightly, in line with the classic distinction between inflection vs. derivation. However, there are numerous cases of lexical items in English with more or less drastic differences in meaning between the singular and plural, and/or senses that are only available in one of the two numbers. Consider the examples in \exref{ex:2.19}.

\begin{exe}
  \ex\label{ex:2.19}
  \exinfo{\idx{English} (Indo-European > Germanic)}
  \begin{xlist}

    \ex
    \begin{tabular}[t]{ p{0.75in} l }
      \txn{air}  & \tln{atmosphere}\\
      \txn{airs} & \tln{affected manners}\\
    \end{tabular}

    \ex
    \begin{tabular}[t]{ p{0.75in} l }
      \txn{arm}  & \tln{upper limb; anything resembling a limb}\\
      \txn{arms} & \tln{weapons, firearms}\\
    \end{tabular}

    \ex
    \begin{tabular}[t]{ p{0.75in} l }
      \txn{blind}  & \tln{unable to see}\\
      \txn{blinds} & \tln{screen for a window}\\
    \end{tabular}

    \ex
    \begin{tabular}[t]{ p{0.75in} l }
      \txn{custom}  & \tln{tradition; socially accepted behavior}\\
      \txn{customs} & \tln{department which levies duties on imports}\\
    \end{tabular}

    \ex
    \begin{tabular}[t]{ p{0.75in} l }
      \txn{force}  & \tln{strength, energy}\\
      \txn{forces} & \tln{collection of military units}\\
    \end{tabular}

    \ex
    \begin{tabular}[t]{ p{0.75in} l }
      \txn{good}  & \tln{excellent, high quality}\\
      \txn{goods} & \tln{merchandise or possessions}\\
    \end{tabular}

    \ex
    \begin{tabular}[t]{ p{0.75in} l }
      \txn{manner}  & \tln{way of doing something}\\
      \txn{manners} & \tln{social conduct; socially acceptable conduct}\\
    \end{tabular}

    \ex
    \begin{tabular}[t]{ p{0.75in} l }
      \txn{spectacle}  & \tln{visually striking performance or display}\\
      \txn{spectacles} & \tln{pair of glasses}\\
    \end{tabular}

    \ex
    \begin{tabular}[t]{ p{0.75in} l }
      \txn{wood}  & \tln{fibrous material in the trunk of trees or shrubs}\\
      \txn{woods} & \tln{area of land covered with trees}\footnote{In some dialects of English, this sense is available in the singular as well.}\\
    \end{tabular}

  \end{xlist}
\end{exe}

Semantic shifts for plural marking in \idx{English} are not limited to just a handful of specific lexical items. Generic uses of the plural as in the expression \txn{foxes are cunning} create a semantic shift away from a concrete entity (\txn{a/the fox}) to a generic, unperceivable one—a use which strays from the prototypical function of nouns as concrete perceptible entities \parencite[708]{HopperThompson1984}.

As with flexible items, the semantic shifts that occur with plural marking can become so substantial that speakers no longer cognize the morphological singular and plural as members of the same lexeme. Such is the case in the historical development of \txn{brother} vs. \txn{brethren} in \idx{English}. The word \txn{brethren} became so strongly conventionalized with its religious meaning in the plural that it was independently lexicalized as a plural-only (\foreign{plurale tantum}) noun, and the original plural underwent renewal with the emergence of the form \txn{brothers}. This is exactly the kind of lexicalization process that occurred for many morphological verbs reanalyzed as nouns in \idx{Cayuga} and many other North American languages.

A similar example comes from \idx{Chitimacha}, which has a pluractional marker \txn{-ma} indicating verbal number (plural agents, plural patients, or repeated action). In some cases the use of \txn{-ma} is purely compositional, so that it can be considered merely an inflectional marker of verbal number. In other cases \txn{-ma} so significantly alters the meaning of the word that it must be considered derivational. Compare the uses of \txn{-ma} in each of the pairs of verbs in \exref{ex:2.20} (note that \exref{ex:2.20b} and \exref{ex:2.20c} are phrasal verbs with a preverbal particle).

\begin{exe}
  \ex\label{ex:2.20}
  \exinfo{\idx{Chitimacha} (isolate)}
  \begin{xlist}

    \ex\label{ex:2.20a}
    \begin{tabular}[t]{ p{1in} l }
        \txn{kow-}   & \tln{call}\\
        \txn{kooma-} & \tln{call multiple people}\\
    \end{tabular}

    \ex\label{ex:2.20b}
    \begin{tabular}[t]{ p{1in} l }
        \txn{qapx cuw-}   & \tln{come back; go about}\\
        \txn{qapx cuuma-} & \tln{travel; wander}\\
    \end{tabular}

    \ex\label{ex:2.20c}
    \begin{tabular}[t]{ p{1in} l }
        \txn{qapx qiy-}   & \tln{turn together; mix, join}\\
        \txn{qapx qiima-} & \tln{give a prayer, benediction; perform magic}\\
    \end{tabular}

  \end{xlist}
  \exsourcebelow{Swadesh1939a}
\end{exe}

\noindent In \exref{ex:2.20a}, the use of \txn{-ma} is entirely compositional. The presence of \txn{-ma} indicates that the verb has a plural patient argument. In \exref{ex:2.20b}, the use of \txn{-ma} is still arguably compositional, though perhaps somewhat lexicalized given the high frequency with which the stem appears in the texts. \tln{travel, wander} could reasonably be interpreted as a continued repetition of \tln{go about}. In \exref{ex:2.20c}, however, \txn{qapx qiima-} has become lexicalized with a new meaning not directly related to that of \txn{qapx qiy-}. The diachronic connection between the two meanings is that prayers and magical incantations were traditionally accompanied by circling gestures with the arms. \txn{qapx qiima-} originally meant \tln{turn/circle around repeatedly}, but over time lexicalized with its new religious meaning in the pluractional, \tln{give a prayer, benediction}. This lexicalization process parallels that of \txn{brethren} in \idx{English}. Such a range of inflectional vs. derivational uses of pluractionals is quite common crosslinguistically \parencites{Mithun1988}{Mattiola2020}.

Finally, there are many languages which do not typically mark plurality on nouns \parencite{Dryer2013}, and yet have senses available in semantically plural contexts but not singular ones (where the semantic number can be understood from the clausal context, usually through verbal number marking). For example, the word \txn{soq} in \idx{Chitimacha} may mean \tln{foot} or \tln{paw} in a singular context and \tln{feet} or \tln{paws} in a plural context, but may also mean \tln{tracks} (e.g. animal tracks) in a plural context—a significant and idiosyncratic shift in meaning, and one that is both language-specific and item-specific and thus conventional. This use constitutes a \emph{morphologically unmarked semantic shift} in the meaning of the word, just as idiosyncratic meanings of words in cases of functional expansion also constitute morphologically unmarked semantic shifts. If we take such unmarked semantic shifts as evidence against lexical unity in the cases of flexible items, then we must also say that the \tln{foot} and \tln{tracks} meanings of \txn{soq} constitute two distinct lexemes as well.

One might ask, if we start splitting up lexemes based on every degree of semantic shift, where does the splitting stop? This is exactly analogous to the problem of lumping vs. splitting in the context of lexical categories. The Radical Construction Grammar solution to this problem is to abandon the commitment to larger groupings of items (the major lexical categories) and acknowledge that languages consist of an interconnected network of smaller items (constructions) instead \parencite{Croft2001b}. This approach has the major advantage of sidestepping unproductive debates about the existence or unity of lexical categories in particular languages, and shifts the focus instead to understanding the relationships and patterns among individual constructions. This is precisely what I propose to do for lexemes as well. If we abandon the idea that all the meanings associated with a form must be in some way grouped into lexemes based on their morphosyntactic contexts of occurrence, we sidestep unproductive debates regarding homonymy vs. polysemy, and can instead focus on the relationships and patterns among the various senses associated with that form—specifically, the nature of the semantic shifts that occur between uses of the form in different discourse functions. Rather than using semantic shift as a diagnostic for delimiting / distinguishing lexemes, we can first give a strutural definition of lexical flexibility, and then examine the semantic shifts that occur among different flexible uses of a lexical item. Semantic shifts become descriptive desiderata rather than definitional criteria.

In sum, idiosyncratic semantic shifts do not invalidate the concept of lexical flexibility, at least not when properly understood as functional expansion rather than true \enquote{flexibility}. Indeed, functional expansion would not be possible if hearers were not capable of determining the meaning of a form when used in even highly unusual contexts. Innovative uses of lexical items in new functions would be all but impossible, providing no opportunity for such innovations to receive broader adoption in the linguistic community. Each time a hearer encounters a novel use of a lexical item for the first time, they must accomplish the difficult task of discerning its meaning. This is no less true for flexible items as it is for non-flexible items, or for items whose meaning is predictable vs. unpredictable. \emph{Every} use of a word is an instance of functional expansion because every use of a word is always in a slightly different discourse and social context than the one before. The meaning of a word in a given context is highly socially and situationally dependent, and that context can change completely from one utterance to the next. Every token of a word thus necessarily appears in a new pragmatic context, and that pragmatic context slightly shapes its meaning \parencite[99--105]{Croft2010}. Language use \emph{is} language change. Semantic shift is therefore an integral and ubiquitous part of language use.
