\chapter{Conclusion}
\label{ch:conclusion}

\blockquote{This chapter summarizes the methods and main findings of this study, and the considers the implications of those results for theories of lexical categories. I argue that the data provide compelling evidence in favor of functional approaches to lexical categorization, most especially cognitive prototype theory and Croft's theory of lexical categories as typological markedness patterns. I also argue for a reversal of the canonical position on parts of speech: instead of working from the default assumption that all languages have clearly-defined or even loosely-defined parts of speech, we should begin from the understanding that dedicated referring, predicating, or modifying constructions develop diachronically, and that even when they do, they do not do so for the entire lexicon, or in all areas of the grammar equally. Even languages like English whose lexemes pattern strongly with the standard prototypes of noun, verb, and adjectives nonetheless exhibit varying degrees of flexibility for different lexemes. Lexical categories are not a given in grammar. I conclude by discussing some limitations of the present study and avenues for future research, followed by closing remarks.}

\section{Introduction}
\label{sec:5.1}

This chapter presents a summary of the study (\secref*{sec:5.2}) and its major findings (\secref*{sec:5.3}). It provides a discussion of the theoretical implications of those findings (\secref*{sec:5.4}) and directions for future research (\secref*{sec:5.5}). I conclude that researchers should shift from treating lexical flexibility as an exotic analytical problem to a foundational feature of language (\secref*{sec:5.6}).

\section{Summary of the study}
\label{sec:5.2}

Lexical flexibility—the use of a lexical item in more than one discourse function (reference, predication, or modification) with no overt derivational morphology—has historically been an intractable problem for theories of parts of speech. The Classical tradition inherited from Ancient Greek and Latin requires that each lexeme be sorted into mutually exclusive lexical categories defined by a clear set of necessary and sufficient conditions. Forms that seem to cross-cut these categorial boundaries thus present a theoretical quandry.

One common solution to this problem is to analyze any form used for more than one discourse function as a case of heterosemy—a special case of homonymy in which two lexemes share the same form but belong to distinct word classes \parencite{Lichtenberk1991}. A second common solution is to adjust the features that define the relevant word classes so as to preserve the traditional classification scheme. This always involves privileging certain kinds of evidence for lexical categories over others, or excluding certain morphosyntactic evidence entirely. A final solution is to define new kinds of lexical categories such as \enquote{contentives}, \enquote{flexibles}, or \enquote{non-verbs} \parencites{HengeveldRijkhoff2005}{Luuk2010} for the purpose of accommodating the flexible forms.

What these approaches have in common is their commitment to a small set of well-defined word classes. They also generally agree on the empirical facts of the matter. Disagreements over the analysis of flexible forms arise primarily from disagreements over the relative importance of different pieces of evidence rather than the accuracy of the evidence itself \parencites[235]{Wetzer1992}[32]{Stassen1997}[58]{CroftLier2012}. Yet, though researchers have debated the definitional criteria for lexical categories for as long as modern linguistics has existed, there is still no consensus. Analyses of lexical flexibility depend primarily on the theoretical commitments of the researcher rather than any crucial pieces of evidence. Methodological opportunism, in which researchers select the definitional criteria for lexical categories that best support their theoretical commitments while dismissing or deemphasizing contradictory criteria \parencite[30]{Croft2001b}, is a rampant problem in research on word classes.

A consequence of this methodological opportunism is that until recently lexical flexibility was not appreciated as the interesting phenomenon it is. Flexible forms were placed into one category or another and the problem was considered solved. But to lump flexible forms in with overtly derived forms ignores the fact that there is something unique about them—namely that they can appear in different discourse functions with no overt derivational morphology. Just how prevalent is this phenomenon? Why do these words in particular behave this way while others do not? How productive is it? Are the meaning shifts that occur in functional shift different from or the same as the meaning shifts that occur in cases of overt derivation? An attitude that treats flexible forms as a problem to be solved preempts these kinds of questions—or at least shifts focus away from them. Regardless of one's theoretical analysis of flexible forms, their behavior is substantively different from non-flexible ones, and this fact merits investigation.



\section{Summary of findings}
\label{sec:5.3}

\section{Discussion}
\label{sec:5.4}

\section{Future research}
\label{sec:5.5}

\section{Conclusion}
\label{sec:5.6}
