\chapter{Conclusion}
\label{ch:conclusion}

\blockquote{This chapter summarizes the methods and main findings of this study, and the considers the impliations of those results for theories of lexical categories. I argue that the data provide compelling evidence in favor of functional approaches to lexical categorization, most especially cognitive prototype theory and Croft's theory of lexical categories as typological markedness patterns. I also argue for a reversal of the canonical position on parts of speech: instead of working from the default assumption that all languages have clearly-defined or even loosely-defined parts of speech, we should begin from the understanding that dedicated referring, predicating, or modifying constructions develop diachronically, and that even when they do, they do not do so for the entire lexicon, or in all areas of the grammar equally. Even languages like English whose lexemes pattern strongly with the standard prototypes of noun, verb, and adjectives nonetheless exhibit varying degrees of flexibility for different lexemes. Lexical categories are not a given in grammar. I conclude by discussing some limitations of the present study and avenues for future research, followed by closing remarks.}
