% referring to Kihm's critique of precategorial / vagueness accounts; originally in Section 2.3.1.4.

This difficulty could perhaps be overcome, however, by loosening the requirement that the meaning of a lexical item be unitary. Word meanings are \dfn{polycentric}, where different senses of an item are related through \dfn{meaning chains} rather than all through a single, central member \parencite[110]{Taylor2003}. This is often referred to as a \dfn{family resemblance} structure for categories. The difference between monocentric and polycentric categories is illustrated schematically in \figref{fig:monocentric-vs-polycentric}. In both diagrams, each letter A–E represents a sense of a lexical item. In the monocentric case, all the senses of the lexical item are related through its core sense A. In the polycentric case, senses A and E are related only through their intervening connections.\footnote{The terms \dfn{monothetic} and \dfn{polythetic} are sometimes used for this distinction instead \parencite[146]{LewandowskaTomaszczyk2007}.}

\begin{figure}[h!]
  \centering
  \includegraphics[width=\linewidth/2]{monocentric-vs-polycentric.png}
  \caption{Monocentric vs. polycentric categories}
  \label{fig:monocentric-vs-polycentric}
\end{figure}

Recognizing that word meanings are polycentric addresses \posscitet{EvansOsada2005} and \posscitet{Kihm2017} criticisms of vagueness theory because it shows that the disparate senses of a lexical item can be related without having to share any core component of their meanings. The use of a lexeme in a certain context then profiles one of these senses over others. Kihm himself hints at this solution by referring to the related \idx{Arabic} stems in \exref{ex:2.7} as a \dfn{lexical family}.
