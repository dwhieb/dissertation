\chapter{Introduction}
\label{ch:introduction}

\blockquote{This chapter motivates the need for research on lexical flexibility by situating it within broader concerns regarding linguistic categories more generally, and categories in human cognition. The specific problem that this study seeks to address is our lack of understanding regarding what lexical flexibility looks like, and how it varies across languages. This thesis contributes to answering these questions via a quantitative corpus-based study of lexical flexibility in English (Indo-European)\index{English} and Nuuchahnulth (Wakashan). It is the first study to examine lexical flexibility using natural discourse from corpus data. This chapter provides an overview of the thesis, including the specific research questions addressed, the data and methods used, a concise summary of the results, and a preview of the conclusions.}

Word classes such as noun, verb, and adjective (traditionally called \dfn{parts of speech}) were once thought to be universal, easily identifiable, and easily understood. Today they are one of the most controversial and least understood aspects of language. While language scientists generally agree that word classes exist, there is much disagreement as to whether they are categories of individual languages, categories of language generally, categories of human cognition, categories of language science, or some combination of these possibilities \addcite{Mithun 2017: 166; Haspelmath 2018; Hieber forthcoming: 1}. Lexical categorization—how languages separate words into categories—is of central importance to theories of language because it is tightly interconnected with linguistic categorization generally, which in turn informs (and is informed by) our understanding of cognition. Categorization is a fundamental feature of human cognition \addcite{Taylor 2003: xi}, and lexical categorization is perhaps the most foundational issue in linguistic theory \addcite{Croft 1991: 36; Vapnarsky \& Veneziano 2017: 1}.

One challenge for traditional theories of word classes is the existence of \dfn{lexical flexibility}—the use of a word in more than one discourse function, whether to refer (as a noun), to predicate (as a verb), or to modify (as an adjective). In traditional terms, flexible words are those which may be used for more than one part of speech. (A more precise definition of lexical flexibility is given in \addcrossref{Chapter 2}.) Examples of flexible words in several languages are shown below.

\vspace{1em}
\todo[
  caption = {add lexical flexibility examples to intro},
  inline,
  size = normalsize
]{\begin{spacing}{1}Give examples here. Discuss all the examples together in the lead out. Examples: English; Nuuchahnulth: Kingfisher 202\end{spacing}}

Flexible words like those in the examples above create an analytical problem for traditional theories of parts of speech. Traditional theories assume that words can be partitioned into mutually exclusive categories on the basis of a clear set of criteria, an approach that has its roots in the Aristotelian tradition of defining a category via its necessary and sufficient conditions. Flexible words would seem to violate this assumption because they appear to be members of more than one category at once, and the criteria for classification yield conflicting results.

Researchers have proposed numerous solutions to this problem. The most common response is to adjust the selectional criteria so that only certain features are considered definitional of the class, allowing these researchers to dismiss other, potentially contradictory evidence as irrelevant \addcite{Baker 2003; Dixon 2004; Floyd 2011 for Quechua; Chung 2012 for Chamorro; Palmer 2017}. It is also common to analyze different uses of a putatively flexible word as instances of \dfn{heterosemy}—that is, entirely distinct words which share the same form but belong to different word classes \addcite{Lichtenberk 1991}. In this view, heterosemous words are related only historically, via a process of conversion or functional shift, in essence denying the existence of lexical flexibility \addcite{Evans \& Osada 2005}. Another approach is to claim that, while all words can be neatly categorized, some words in some languages may nonetheless be used for functions typically associated with other categories. A notable example of this is Launey's \addcite{1994?, 2001?} analysis of Classical Nahuatl, which he calls an \dfn{omnipredicative} language. In this analysis, Classical Nahuatl has the traditional, clearly-delineated word classes of noun and verb, but allows for any word to function as a verb regardless of its category (hence the term \dfn{omnipredicative}). The reverse is not true however; only some verbs may function as nouns. \TODO[confirm Launey's analysis]{Make sure that this is an accurate description of the empirical facts of Classical Nahuatl.} This difference in behavior is taken as the basis for a categorical\footnote{Throughout this thesis, I use the term \txn{categorical} to mean \tln{without exception; unconditional} and the term \txn{categorial} to mean \tln{having to do with categories}.} distinction between nouns and verbs.

Some researchers enthusiastically embrace the existence of lexical flexibility and abandon a commitment to the traditional categories of noun, verb, and adjective. Instead they analyze flexible lexemes as belonging to a broader, flexible word classes such as \enquote{flexibles}, \enquote{contentives} or \enquote{non-verbs}, etc. \addcite{Hengeveld \& Rijkhoff 2005; Luuk 2010?}. Other researchers abandon the commitment to word classes entirely. Mandarin, Tagalog, Tongan, and Riau Indonesian have each been analyzed as lacking parts of speech \addcite{Simon [1937], McDonald [2013], and Sun [2020] for discussions of early analyses of Mandarin; Gil [XXXX] for Tagalog; Broschart [XXXX] for Tongan; Gil [XXXX] for Riau Indonesian}. Within generative linguistics, the Distributed Morphology framework takes it as an assumption that all word roots are category-neutral \addcite{Siddiqi 2018}.

Note that these differences in perspective do not arise from disagreements about the empirical facts of each language. Researchers mostly agree on the empirical data, but disagree on the relative importance of various pieces of evidence, and on which criteria should be taken as diagnostic of a category \addcite{Croft \& van Lier 2012: 58}. \TODO[add examples of disagreements]{Add example from Stassen [1997: 32] about two researchers coming to different conclusions about Sudanese. Also the example of Quechua or perhaps Iroquoian (Chafe). Also Mundari. Chamorro: Topping and Chung. Among many others.} It is rare that an argument for flexibility is refuted on the basis of the linguistic facts alone \addcite{cite Mithun's response to Sasse's analysis of Cayuga as a flexible language}.

Since analyses of lexical flexibility depend more on the particular theoretical commitments of the researchers involved rather than any particular crucial pieces of evidence, this leads to an intractable problem: researchers cannot agree on the criteria that should be considered diagnostic for a given category in a specific language (let alone crosslinguistically). Instead they partake in \dfn{methodological opportunism} \addcite{Croft 2003?: ??}, choosing the evidence and criteria which best support their theoretical commitments. Discussions in the literature about the existence of a particular category in a particular language are therefore often unproductive, and devolve into debates about theoretical assumptions or the relevance or importance of various pieces of evidence, which are ultimately unresolvable \addcite{Croft 2005: 435}.

This is particularly unfortunate because lexical flexibility is by no means an isolated or minor problem. Additional examples like those above could be provided for many or perhaps even all of the world's languages. Lexical flexibility is not as rare or marginal as traditional approaches to word classes lead one to believe. In a survey of word classes in 48 indigenous North American languages \addcite{Hieber forthcoming}, every one of the languages surveyed exhibited lexical flexibility in at least some area of the grammar (although not all authors analyzed these cases as such). In my own experience researching lexical flexibility over the last decade, I have yet to encounter a language that does not exhibit a degree of flexibility in at least some words, however marginally. The prevalence with which different areas of the grammars of the world's languages lack sensitivity to the distinctions between reference (nouns), predication (verbs), and modification (adjectives) suggests that lexical flexibility may not be so much of a problem as it is a design feature of language \addcite{Hieber forthcoming}. Indeed, as discussed in \addcrossref{Chapter 2}, existing research shows that lexical flexibility is constrained and shaped by the very principles that give rise to the crosslinguistic categories of noun, verb, and adjective in the first place \addcite{Croft 2000; Croft 2005; Croft \& van Lier 2012}.
