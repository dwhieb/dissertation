\chapter{Introduction}
\label{ch:introduction}

\blockquote{This chapter motivates the need for research on lexical polyfunctionality by situating it within broader concerns regarding linguistic categories more generally, and categories in human cognition. The specific problem addressed is our lack of understanding regarding what lexical polyfunctionality looks like and how it varies across languages. This dissertation contributes to answering these questions via a quantitative corpus-based study of lexical polyfunctionality in English (Indo-European > Germanic) and Nuuchahnulth (Wakashan > Southern Wakashan). I analyze approximately 380,000 tokens of English and 8,300 tokens of Nuuchahnulth for their discourse function to determine the overall degree of functionality in these languages. This is the first study to examine lexical polyfunctionality using natural discourse data from corpora. This chapter provides an overview of the dissertation, including the specific research questions addressed, the data and methods used, a concise summary of the results, and a preview of the conclusions.}

\section{The problem of lexical polyfunctionality}
\label{sec:1.1}

Word classes such as noun, verb, and adjective (traditionally called \dfn{parts of speech}) were once thought to be universal, easily identifiable, and easily understood. Today they are one of the most controversial and least understood aspects of language. While language scientists agree that word classes exist, there is much disagreement as to whether they are categories of individual languages, categories of language generally, categories of human cognition, categories of language science, or some combination of these possibilities \parencites[166]{Mithun2017}{Haspelmath2018}{Hieberfc}. Lexical categorization—how languages assign lexical items\footnote{I use the term \dfn{lexical item} as a convenient cover term for root, stem, or fully inflected word. This term does not here refer to the phonological word, syntactic word, or any other concept of word. The reason for this vague usage is because languages vary as to which morphological level bears category information. This issue is discussed more fully in \secref*{sec:2.3.2.3}. I use \dfn{lexical item} instead of \dfn{lexeme} because the concept of a lexeme implies lexical unity, that is, that we are discussing a single polysemous item rather than two homophonous ones. Use of the term \dfn{lexical item} is intended to bypass this distinction in favor of a focus on form. However, I also avoid the term \dfn{(lexical) form} because some lexical items have multiple forms (in the case of suppletion).} to categories—is of central importance to theories of language because it is tightly interconnected with linguistic categorization generally, which in turn informs (and is informed by) our understanding of cognition. Categorization is a fundamental feature of human cognition \parencites[xi]{Taylor2003}[2--3]{LierRijkhoff2013}, and lexical categorization is perhaps the most foundational issue in linguistic theory \parencites[36]{Croft1991}[1]{VapnarskyVeneziano2017b}.

One challenge for traditional theories of word classes is the existence of \dfn{lexical polyfunctionality}—the use of a lexical item in more than one discourse function with zero coding, whether that item is used to refer (like a noun), to predicate (like a verb), or to modify (like an adjective). Other recent terms for this phenomenon include \dfn{lexical flexibility}, \dfn{polycategoriality}, and \dfn{precategoriality} (see \secref*{sec:2.3.1} for detailed explanations of the differences between these concepts). In traditional terms, polyfunctional words have been described as those which may be used for more than one part of speech with no overt derivational morphology. This has often been termed \dfn{functional shift}, \dfn{conversion}, or \dfn{zero derivation}. A more thorough and precise definition of lexical polyfunctionality is given in \secref*{sec:2.5}. Examples of polyfunctional lexical items in several languages are shown below. In the examples, \textbf{Ref} stands for a lexical item being used for reference, \textbf{Pred} for a lexical item being used for predication, and \textbf{Mod} for a lexical item being used for modification. The polyfunctional item in each set of examples is shown with \em{emphasis}. Here and throughout this dissertation, I use the terms \dfn{reference}, \dfn{predication}, and \dfn{modification} so as to focus on the functions of lexical items and avoid committing to any analysis regarding their part-of-speech classification.

\begin{exe}

  \ex\label{ex:1.1}
  \exinfo{\idx{English} (Indo-European > Germanic)}
  \begin{xlist}[\textbf{Pred:}]

    \exi{\textbf{Ref:}} And the spots of \em{paint} would change every hundred degrees.
    \exsourcebelow[FrancisClem]{OANC}

    \exi{\textbf{Pred:}} One story does come to my mind though where you \em{painted} the foundation coating on the house and got tar all over you.
    \exsourcebelow[BorelRaymondHydellII]{OANC}

    \exi{\textbf{Mod:}} And it happened to be one of the rare \em{paint} jobs.
    \exsourcebelow[sw2236]{OANC}

  \end{xlist}

  \ex\label{ex:1.2}
  \exinfo{\idx{Mandinka} (Mande > Manding)}
  \begin{xlist}[\textbf{Pred:}]

    \exi{\textbf{Ref:}}
    \gll \em{Kuuráŋ}‑o      mâŋ          díyaa.\\
         \em{sick}‑\gl{def} \gl{pfv.neg} pleasant\\
         \tln{Sickness is not pleasant.}
    \exsource[46]{Creissels2017}

    \exi{\textbf{Pred:}}
    \gll Díndíŋ‑o       máŋ          \em{kuraŋ}.\\
         child‑\gl{def} \gl{pfv.neg} \em{sick}\\
         \tln{The child is not sick.}
    \exsource[46]{Creissels2017}

  \end{xlist}

  \ex\label{ex:1.3}
  \exinfo{\idx{Mundari} (Austroasiatic > Munda)}
  \begin{xlist}[\textbf{Pred:}]

    \exi{\textbf{Ref:}}
    \gll \em{buru}=ko                bai‑ke‑d‑a.\\
         \em{mountain}=\gl{3pl.subj} make‑\gl{compl}‑\gl{tr}‑\gl{ind}\\
         \tln{They made the mountain.}
    \exsource[354]{EvansOsada2005}

    \exi{\textbf{Pred:}}
    \gll saan=ko                \em{buru}‑ke‑d‑a.\\
         firewood=\gl{3pl.subj} \em{mountain‑}\gl{compl}‑\gl{tr}‑\gl{ind}\\
         \tln{They heaped up the firewood.}
    \exsource[355]{EvansOsada2005}

  \end{xlist}

  \ex\label{ex:1.4}
  \exinfo{\idx{Nuuchahnulth} (Wakashan > Southern Wakashan)}
  \begin{xlist}[\textbf{Pred:}]

    \exi{\textbf{Ref:}}
    \gllll watqšiƛ              ʔaƛimt            …\\
           watq‑ši(ƛ)           \em{ʔaƛa}‑imt      …\\
           swallow‑\gl{mom}     \em{two}‑\gl{past} …\\
           completely.swallowed two               …\\
           \tln{He swallowed two of them […]}
    \exsource[Qawiqaalth 57]{Louie2003}

    \exi{\textbf{Pred:}}
    \gllll wik̓aƛ        haʔukšiƛ     ʔaƛiičiƛ\\
           wik‑ʼaƛ      haʔuk‑ši(ƛ)  \em{ʔaƛa}‑ʽi·čiƛ\\
           not‑\gl{fin} eat‑\gl{mom} \em{two}‑\gl{incep}\\
           didn’t       ate          became.two\\
           \tln{He (Mink) didn’t eat them and the crabs became two.}
    \exsource[Mink 266]{Louie2003}

    \exi{\textbf{Mod:}}
    \gllll hiiɬtqyaap̓up             ʔaƛa      qʷayac̓iik\\
           hiɬ‑tqya·p̓i‑up           \em{ʔaƛa} qʷayac̓iːk\\
           there‑back‑\gl{mom.caus} \em{two}  wolf\\
           put.on.the.back          two       wolf\\
           \tln{Two wolves put (the dead wolf) on their back.}
    \exsource[FoodThief 46]{Louie2003}

  \end{xlist}

  \ex\label{ex:1.5}
  \exinfo{\idx{Quechua} (Quechuan)}
  \begin{xlist}[\textbf{Pred:}]

    \exi{\textbf{Ref:}}
    \gll rikaškaː \em{hatun}‑(kuna)‑ta\\
         I.saw    \em{big}‑(\gl{pl})‑\gl{acc}\\
         \tln{I saw the big one(s)}
    \exsource[17]{SchachterShopen2007}

    \exi{\textbf{Pred:}}
    \gll chay runa \em{hatun} (kaykan)\\
         that man  \em{big}   is\\
         \tln{that man is big}
    \exsource[17]{SchachterShopen2007}

    \exi{\textbf{Mod:}}
    \gll chay \em{hatun} runa\\
         that \em{big}   man\\
         \tln{that big man}
    \exsource[17]{SchachterShopen2007}

  \end{xlist}

  \ex\label{ex:1.6}
  \exinfo{\idx{Tongan} (Austronesian > Polynesian)}
  \begin{xlist}[\textbf{Pred:}]

    \exi{\textbf{Ref:}}
    \gll naʼe      lele e         kau         \em{fefiné}\\
         \gl{past} run  \gl{spec} \gl{pl.hum} \em{woman}.\gl{def}\\
         \tln{The women were running.}
    \exsource[134]{Broschart1997}

    \exi{\textbf{Pred:}}
    \gll naʼe      \em{fefine} kotoa e         kau         lelé\\
         \gl{past} \em{woman}  all   \gl{spec} \gl{pl.hum} run.\gl{def}\\
         \tln{The ones running were all female.}
    \exsource[134]{Broschart1997}

  \end{xlist}

  \ex{
    \label{ex:1.7}
    \exinfo{Central Alaskan Yup'ik\index{Yup'ik} (Eskimo-Aleut > Yup'ik)}
    \begin{xlist}

      \ex\label{ex:1.7a}
      \begin{tabularx}{\linewidth}[t]{ l p{6em} l }
        { }            & \txn{iqa‑}            & \tln{dirt}; \tln{be dirty}\\
        { }            & \txn{‑ngtak}          & \tln{very}\\
        \textbf{Ref:}  & \em{\txn{iqa‑ngtak}}  & \tln{one that is very dirty}\\
        \textbf{Pred:} & \em{\txn{iqa‑ngtaq‑}} & \tln{be very dirty}\\
      \end{tabularx}
      \exsourcebelow[159]{Mithun2017}

      \ex\label{ex:1.7b}
      \begin{tabularx}{\linewidth}[t]{ l p{6em} l }
        { }            & \txn{tangerr‑}          & \tln{see}\\
        { }            & \txn{‑uaq}              & \parbox[t]{3in}{\tln{imitation, inauthentic};\\\tln{pretend to, without serious purpose}\vspace{0.25em}}\\
        \textbf{Ref:}  & \em{\txn{tangerr‑uaq}}\footnote{This form is spelled \txn{tangrr‑} in the original, but Mithun (p.c.) confirms that this is a typo. The correct form is \txn{tangerr‑}.} & \tln{movie, vision, hallucination}\\
        \textbf{Pred:} & \em{\txn{tangerr‑uar‑}} & \tln{hallucinate, watch a movie}\\
      \end{tabularx}
      \exsourcebelow[159]{Mithun2017}

      \ex\label{ex:1.7c}
      \begin{tabularx}{\linewidth}[t]{ l p{6em} l }
        { }            & \txn{iqeq‑}        & \tln{corner of mouth}\\
        { }            & \txn{‑mik}         & \parbox[t]{3in}{\tln{thing held in one's mouth};\\\tln{to put in one's}\vspace{0.25em}}\\
        \textbf{Ref:}  & \em{\txn{iq‑mik}}  & \tln{chewing tobacco}\\
        \textbf{Pred:} & \em{\txn{iq‑mig‑}} & \tln{put in one's mouth}\\
      \end{tabularx}
      \exsourcebelow[160]{Mithun2017}

    \end{xlist}
  }

\end{exe}

\noindent In the \idx{English} example in \exref{ex:1.1}, the predicative use of \txn{paint} takes the English Past Tense suffix \txn{-ed} like any prototypical verb in English, but there is no morpheme present that explicitly indicates the shift from a referential use to a predicative use (or vice versa). The remaining examples illustrate the same situation for a variety of language families around the world. Even though in some cases inflectional morphology indicates the function of the word, none of these examples have overt derivational morphology converting the target lexical items from one discourse function to another.

Polyfunctional items like those in the examples above create an analytical problem for traditional theories of parts of speech. Traditional theories assume that lexical items can be partitioned into mutually exclusive categories based on a clear set of criteria, an approach that has its roots in the Aristotelian tradition of defining a category via its necessary and sufficient conditions. Polyfunctional items would seem to violate this assumption because at first glance they appear to be members of more than one category at once, and the criteria for classifying them yield conflicting results.

Researchers have proposed numerous solutions to this problem. One response is to analyze different uses of a polyfunctional item as instances of \dfn{heterosemy}—a special case of homonymy in which two distinct lexemes share the same form but belong to different word classes \parencite{Lichtenberk1991}. In this view, heterosemous items are related only historically, via a process of conversion or functional shift, in essence denying any synchronic connection between them \parencite{EvansOsada2005}. However, this perspective fails to answer why polyfunctionality is rampant in some languages but not others, or why some lexemes are polyfunctional but not others, or what motivates a lexical item to expand its uses into new discourse functions. Morever, it is difficult to maintain a principled distinction between polysemy and heterosemy. Semantic, distributional, and formal similarity between words are continua, meaning that questions like \enquote{are uses X and Y of a form instances of the same or different lexemes?} cannot be answered categorically. Questions about multifunctional uses of the same form—call it polyfunctionality, lexical flexibility, conversion, or something else—merit empirical investigation irrespective of one's analytical position on the matter.

A more common approach to analyzing polyfunctionality is to adjust the selectional criteria so that only certain features are considered definitional of a word class, allowing these researchers to dismiss other, potentially contradictory evidence as irrelevant \parentext{\textcite{Baker2003}; \textcite{Dixon2004}; \textcite{Palmer2017}; \textcite{Floyd2011} for \idx{Quechua}; \textcite{Chung2012} for \idx{Chamorro}}. Another approach is to say that languages exhibiting polyfunctionality have only some of the traditional categories. A notable example of this is Launey's \parencites*{Launey1994}{Launey2004} analysis of Classical \idx{Nahuatl}, which he calls an \dfn{omnipredicative} language. In this analysis, all lexical items are predicates, so there is just one giant class of verbs.

Some researchers enthusiastically embrace the existence of polyfunctionality and abandon a commitment to the traditional categories of noun, verb, and adjective. This has various realizations. One approach analyzes polyfunctional forms as \dfn{flexible}—that is, as single, multifunctional lexemes which can be productively deployed in different (traditional) parts of speech. Another approach expands the traditional slate of parts of speech to include new, broader word classes specifically for these polyfunctional forms, such as \enquote{flexibles}, \enquote{contentives} or \enquote{non-verbs}, etc. \parencites{HengeveldRijkhoff2005}{Luuk2010}.

Other researchers abandon the commitment to word classes entirely. \idx{Mandarin}, \idx{Tagalog}, \idx{Tongan}, \idx{Riau Indonesian}, and Proto-\idx{Indo-European} have each been analyzed as lacking parts of speech by some researchers \parentext{see \textcite{Simon1937}, \textcite{McDonald2013}, and \textcite{Sun2020} for discussions of early analyses of Mandarin; \textcite{Gil1995} for Tagalog; \textcite{Broschart1997} for Tongan; \textcite{Gil1994} for Riau Indonesian; \textcite{Kastovsky1996} for Proto-Indo-European}. Within generative linguistics, the Distributed Morphology assumes that all roots are category-neutral \parencite{Siddiqi2018}. \textcite{Farrell2001} argues that \emph{all} instances of polyfunctional items (which he describes as cases of \enquote{functional shift}) involve roots that are unspecified for category.

In this dissertation, the term \txn{polyfunctionality} is intended to be neutral with respect to these approaches. Polyfunctionality merely describes the synchronic state of affairs in which a lexical item has uses in more than one discourse function with no overt coding strategy for that function. It makes no commitment as to the lexical unity of those different uses (in other words, I purposefully avoid making any judgment as to whether different polyfunctional uses of the same lexical item are instances of polysemy or heterosemy). I present an alternate, typological-constructional analysis of polyfunctionality in \secref*{sec:2.5}. We can also speak of the \dfn{functional diversity} of a lexical item—that is, the degree to which it is polyfunctional. And the diachronic process by which a lexical item expands its use into new discourse functions, thus increasing its functional diversity and giving rise to synchronic polyfunctionality, is referred to in this dissertation as \dfn{functional expansion}. These terms will be clarified in more detail in \secref*{sec:2.5}.

Note that the different perspectives above do not arise from disagreements about the empirical facts. Researchers mostly agree on the empirical data, but disagree on the relative importance of various pieces of evidence, and on which criteria should be taken as diagnostic of a category \parencites[235]{Wetzer1992}[32]{Stassen1997}[58]{CroftLier2012}. Examples of contested languages include those of the \idx{Iroquoian} family \parencite{Chafe2012}, \idx{Mundari} \parencites{EvansOsada2005}{HengeveldRijkhoff2005}, \idx{Quechua} \parencites[17]{SchachterShopen2007}{Floyd2011}, and \idx{Sundanese} \parencites[352]{Robins1968}[62--63]{Hardjadibrata1985}, with many others that could be cited as well. It is rare that an argument for flexibility is refuted by linguistic facts alone \parentext{though see \posscitet{Mithun2000} response to \textcite{Sasse1988} regarding \idx{Cayuga}}.

Since analyses of lexical polyfunctionality depend more on the theoretical commitments of the researchers involved rather than any crucial pieces of evidence, this leads to an intractable problem: researchers cannot agree on the criteria that should be considered diagnostic for a given category in a specific language, let alone crosslinguistically. Instead they partake in \dfn{methodological opportunism} \parencite[30]{Croft2001b}, choosing the evidence and criteria which best support their theoretical commitments. Discussions in the literature about the existence of a particular category in a particular language are therefore often unproductive, and devolve into debates about theoretical assumptions or the relevance or importance of various pieces of evidence, which are ultimately unresolvable \parencite[435]{Croft2005}.

This is particularly unfortunate because polyfunctionality is by no means an isolated or minor phenomenon. Additional examples like those above could be provided for many or perhaps even all the world's languages. Lexical polyfunctionality is not as rare or marginal as traditional approaches to word classes lead one to believe. In a survey of word classes in 48 indigenous North American languages \parencite{Hieberfc}, every language surveyed exhibits lexical polyfunctionality in at least some area of the grammar (although not all authors analyzed these cases as such). In my experience studying lexical polyfunctionality over the last decade, I have yet to encounter a language that does not exhibit a degree of polyfunctionality in at least some lexical items, however marginally. The prevalence with which different areas of the grammars of the world's languages lack sensitivity to the distinctions between reference, predication, and modification suggests that the existence of lexical categories in a language is not necessarily a given \parencite{Hieberfc}.

Given what we know from both cognitive science and diachronic linguistics, it would be surprising if clear-cut categories \emph{did} exist. Cognitive science tells us that mental categories, word meanings, and lexical categories are all prototypal\footnote{In this dissertation, I use the term \txn{prototypical} to mean \tln{having the properties of the prototype, exemplar, or central member of a category} and the term \txn{prototypal} to mean \tln{having a prototype structure, with central and less central members}. The term \txn{prototypal} is borrowed from the programming community, where it is used to describe programming languages (such as JavaScript) in which objects inherit properties from shared prototypes. Word classes may be described as prototypal, and their members as prototypical or non-prototypical.} \parencite{Taylor2003}. What it means for a category to be \dfn{prototypal} is that category membership is graded so that some members of the category are perceived as better representatives of that category than others. The prototypical meaning or concept within a category is the one that speakers conceive of as the most basic. The fact that mental categories are prototypal leads to various \dfn{prototype effects} in both everyday life and language. More prototypical members of a category are learned earlier in development and acquisition, are used more frequently, can be recalled more quickly, are more likely to be represented using a simple lexical item rather than a complex word or compound, and are more strongly primed by the name of the category itself \parencite[78--79]{CroftCruse2004}. Exactly which of these observed effects best picks out the most prototypical meaning of a category is an open question and an area of active research \parencites[75]{Gries2006}[58--59]{GriesDivjak2009}. Regardless, given the prototypal nature of mental categories, it would be quite surprising if lexical categories did not also exhibit prototype effects.

We also know from diachronic linguistics that language change is both gradual and gradient \parencites{HopperTraugott2003}{TraugottTrousdale2010}. At any given point in time a lexical item might be in a stage of transition or expansion from one function into another, meaning that it will show attributes of both. Likewise, languages develop constructions dedicated to signaling the discourse functions of reference, predication, and modification over time, but at any given point in time, a language may have more or fewer of these constructions, and they may be at various stages of development \parencite{Vogel2000}. Given these facts, the real curiosity is how discourse functions come to be grammaticalized in language over time, not why it is that some languages lack such distinctions in certain areas of their grammars. Lexical polyfunctionality is not so much a problem as it is a design feature of language. It is precisely the liminal categorial\footnote{In this dissertation, I use the term \txn{categorical} to mean \tln{without exception; unconditional} and the term \txn{categorial} to mean \tln{having to do with categories}.} status of polyfunctional items that makes them interesting:

\blockquote[{\cite[23]{Croft1991}}]{In the functionalist view, linguists should recognize the boundary status of the cases in question and try to understand why they are boundary cases. The major empirical fact that has led to concrete results for typology is the discovery that the cross-linguistic variation in such things as the basic grammatical distinctions is patterned.}

It is only recently that lexical polyfunctionality has become an object of study in itself, rather than a problem to be solved. As explained above, most prior studies aim to advance a particular analysis rather than to expand empirical coverage of the phenomenon. While they often provide numerous examples, they are neither quantitative nor comprehensive. As yet, there are only a small number of empirical investigations into the extent and nature of lexical polyfunctionality in individual languages (let alone crosslinguistically). What follows is a brief synopsis of the existing studies of this latter type.

\section{Previous research}
\label{sec:1.2}

The existing studies on the empirical extent of lexical polyfunctionality are of two types: lexicon-based studies which examine dictionaries to determine whether lexical items may be used for multiple functions, and corpus-based studies which examine whether and how often lexical items are used for multiple functions in discourse.

\textcite{Cannon1985} is an early lexicon-based study of functional shift (conversion) in the history of \idx{English}. Functional shift became an especially common pattern of word formation in early Middle English as inflectional paradigms were leveled \parencite[414]{Cannon1985}. \citeauthor{Cannon1985} examines 13,805 lexical items from three English dictionaries with etymological information, and finds that just 541 entries (3.92\%) were created via conversion. Conversion from noun > verb is the most common, adjective > noun conversion the second most common, and verb > noun conversion the third most common. The full results from the study are shown in \tabref{tab:Cannon-1985}.

\begin{table}[h]
  \centering
  \caption[Types of conversion in English (Indo-European > Germanic)]{Types of conversion in \idx{English} (Indo-European > Germanic) \parencite[416]{Cannon1985}}
  \label{tab:Cannon-1985}
  \begin{tabular}{ l l r }
    \toprule
    from      & to        & count\\
    \midrule
    noun      & verb      & 189\\
    adjective & noun      & 121\\
    verb      & noun      & 114\\
    noun      & adjective & 77\\
    verb      & adjective & 19\\
    adjective & verb      & 11\\
    adverb    & adjective & 10\\
    \midrule
    { }       & Total     & 541\\
    \bottomrule
  \end{tabular}
\end{table}

Another lexicon-based study, though not explicitly focused on lexical polyfunctionality, is \posscitet{Croft1984} study of categories of \idx{Russian} roots \parentext{summarized in \textcite[66]{Croft1991}}. \citeauthor{Croft1991} finds that Russian roots are unmarked, or among the least marked forms, when their semantic category (object, action, or property) aligns with their discourse function (reference, predication, or modification respectively). When roots are used for discourse functions that are atypical for their meaning—in other words, when they exhibit polyfunctionality—they are marked in some way (or at least as marked as their prototypical uses). These data suggest that lexical polyfunctionality is constrained in a principled way, by what Croft calls the \dfn{typological markedness of parts of speech} (explained in detail in \secref*{sec:2.4}).

In a study of \idx{Mundari}, \textcite{EvansOsada2005} conduct a dictionary analysis using a focused 105-entry sample as well as a larger 5,000-entry-sample. In the 105-entry sample, 74 stems (72\%) could be used as either noun or verb. In the larger sample, 1,953 stems (52\%) could be used as both noun and verb. The complete figures for the large sample are shown in \tabref{tab:Evans-Osada-2005}. \citeauthor{EvansOsada2005} argue on the basis of these data that, because not all the items in the Mundari lexicon are polyfunctional, Mundari is not a flexible language. As with any whole-language typology, however, this is an oversimplification. To overlook the polyfunctionality of these items ignores the behavior of a vast portion of the lexicon. It is exactly this functional diversity which is of interest in this dissertation. \possciteauthor{EvansOsada2005} study nonetheless constitutes an important contribution to our knowledge of the empirical extent of lexical polyfunctionality across languages.

\begin{table}[h]
  \centering
  \caption[Percentage of lexical items used as nouns, verbs, or both in Mundari (Austroasiatic > Munda)]{Percentage of lexical items used as nouns, verbs, or both in \idx{Mundari} (Austroasiatic > Munda) \parencite[383]{EvansOsada2005}}
  \label{tab:Evans-Osada-2005}
  \begin{tabular}{ l r r }
    \toprule
    noun only     &   772 &  20\% \\
    verb only     & 1,099 &  28\% \\
    noun and verb & 1,953 &  52\% \\
    \midrule
    Total         & 3,824 & 100\% \\
    \bottomrule
  \end{tabular}
\end{table}

\textcite[163]{Mithun2017} also conducts a lexicon-based analysis of roots in Central Alaskan Yup'ik\index{Yup'ik} using \posscitet{Jacobson2012} exhaustive dictionary, and shows that only a small minority of roots (12\%) are polyfunctional, and can be used as both nouns and verbs. The results of this study are shown in \tabref{tab:Mithun-2017}. \citeauthor{Mithun2017} reports that the words in these groups cannot be characterized in any general or semantic way. \possciteauthor{Mithun2017} finding that polyfunctionality in Yup'ik is rather marginal is surprising given that Yup'ik was the focus of an extensive debate about whether the language distinguished nouns and verbs \parencite{Sadock1999}. The fixation with these marginal cases in the literature seems disproportionate to their actual frequency of occurrence, again illustrating the disconnect between research advancing a particular analysis and research aiming to improve empirical coverage of the phenomenon. Just as with \idx{Mundari}, however, it would be an oversight to simply ignore these polyfunctional cases. Instead we should ask what accounts for the large difference in the functional diversity of the lexicons of Mundari versus Yup'ik.

\begin{table}[h]
  \centering
  \caption[Percentage of roots used as nouns, verbs, or both in Central Alaskan Yup'ik (Eskimo-Aleut > Yupik)]{Percentage of roots used as nouns, verbs, or both in Central Alaskan Yup'ik\index{Yup'ik} (Eskimo-Aleut > Yupik) \parencite[163]{Mithun2017}}
  \label{tab:Mithun-2017}
  \begin{tabular}{ l r }
    \toprule
    noun only     &  35\% \\
    verb only     &  53\% \\
    noun and verb &  12\% \\
    \midrule
    Total         & 100\% \\
    \bottomrule
  \end{tabular}
\end{table}

In summary, existing lexicon-based studies have yielded differing results, each contributing to our understanding of lexical polyfunctionality, but there are still too few such studies to draw any general conclusions. Since lexicon-based studies report only type frequencies, we do not know whether the polyfunctional lexemes in these studies account for a greater or lesser portion of tokens in a corpus.

Corpus-based studies of lexical polyfunctionality are also scarce. In a study of the discourse functions of property words in \idx{English} and \idx{Mandarin}, \textcite{Thompson1989} reports that predicative uses of adjectives are in fact more common than attributive (modifying) uses of adjectives in conversation. The resulting figures from this study are shown in \tabref{tab:Thompson-1989}. Some of the attributive adjectives reported in \tabref{tab:Thompson-1989} have \enquote{anaphoric head nouns} \parencite[258]{Thompson1989}, meaning that they are adjectives functioning to refer, so the figures presented are not entirely representative of the discourse functions of these items. The study also does not discuss the extent to which \emph{individual} lexical items exhibit this predicate-modifier polyfunctionality—we only have the data in aggregate—and it also excludes any prototypical nouns being used to modify. These methodological choices are appropriate for a study of the discourse uses of prototypical adjectives, but the result is that we cannot infer much about the degree of functional diversity in English or Mandarin from this study.

\begin{table}[h]
  \centering
  \caption[Distribution of functions of property words in English (Indo-European > Germanic) and Mandarin (Sino-Tibetan > Sinitic)]{Distribution of functions of property words in \idx{English} (Indo-European > Germanic) and \idx{Mandarin} (Sino-Tibetan > Sinitic) \parencite[253, 257]{Thompson1989}}
  \label{tab:Thompson-1989}
  \begin{tabular}{ l r r r r }
    \toprule
      {                    } & \multicolumn{2}{c}{English} & \multicolumn{2}{c}{Mandarin} \\
    \midrule
      predicative adjectives & 209 & 86\%                  & 243 & 71\% \\
      attributive adjectives &  34 & 14\%                  &  97 & 29\% \\
    \bottomrule
  \end{tabular}
\end{table}

Nonetheless, \possciteauthor{Thompson1989} study suggests a functional underpinning to the observed polyfunctionality in prototypical property words. She finds that property words have primarily two functions in discourse: 1) to introduce new referents; and 2) to predicate an attribute about a referent. It is therefore no surprise that property words in some languages have their own specialized constructions since they represent a unique mix of referring and predicating functions. However it is equally unsurprising that some languages encode property concepts using either referring or predicating constructions, since prototypical adjectives exhibit behavior related to both functions.

A similar study to \possciteauthor{Thompson1989} is \posscitet[§2.5]{Croft1991} investigation of the frequencies with which different semantic classes of lexical items (object words, action words, and property words)\footnote{I use the terms \dfn{object word}, \dfn{action word}, and \dfn{property word} when referring to the semantic class of a word rather than its discourse function. Object words are object-denoting, action-words are action-denoting, and property words are property-denoting.} are used for different discourse functions (reference, predication, and modification) in four languages: \idx{Quiché Maya} (Mayan), \idx{North Efate} (Austronesian), \idx{Soddo} (Austroasiatic), and \idx{Ute} (Uto-Aztecan). In all four languages, the most frequent use of lexical items is when their discourse function aligns with their semantic class. Object words are most frequently used to refer, action words are most frequently used to predicate, and property words are most frequently used to modify. Together with data from morphological markedness, semantic shifts, and combinatorial possibilities, Croft takes this as evidence that these are the most prototypical discourse functions for those semantic classes. As with other prototype categories, then, lexical categories display prototype effects in grammar. This fact is a key component of Croft's typological-markedness theory of lexical categories, to be explained fully in \secref*{sec:2.4.2}. Like \posscitet{Thompson1989} study, however, Croft's study does not tell us the distributions for individual lexemes. Additionally, \possciteauthor{Croft1991} data include cases of overtly marked uses of lexical items in non-prototypical functions, which would not be considered instances of lexical polyfunctionality.

In sum, no existing studies examine the distribution of discourse functions for individual items while limiting themselves to only polyfunctional (morphologically unmarked) cases. To my knowledge, the studies just reviewed exhaust those that take an empirical approach to determining the degree of polyfunctionality in or across languages. There are numerous additional studies of lexical polyfunctionality, but these either a) focus on particular analyses or theories of polyfunctional items rather than attempt to expand the empirical coverage of them, as mentioned earlier; or b) focus on various dimensions of the \emph{behavior} of polyfunctional items rather than studying the overall \emph{prevalence} of polyfunctionality. This point is not a criticism, but simply a recognition of a lacuna in existing research. The emergent literature which treats lexical polyfunctionality as a phenomenon of interest in its own right and applies empirical data to the task of understanding its behavior has advanced our knowledge of the various ways lexical polyfunctionality can be realized, and what the constraints on that variation are. Existing research shows, for example, that lexical polyfunctionality is constrained and shaped by the very principles that give rise to the crosslinguistic categories of noun, verb, and adjective in the first place \parencites{Croft2000}{Croft2005}{CroftLier2012}. This literature and its many findings are reviewed in \secref*{sec:2.3}.

There is however still much to discover about lexical polyfunctionality. Most significantly, we do not yet know the overall prevalence of the phenomenon. Most grammatical descriptions of polyfunctionality present a relatively small set of handpicked examples, so that we do not know how representative these examples are. \textcite[70]{Croft2001b} makes this point nicely:

\blockquote[{\cite[70]{Croft2001b}}]{Does English have too few N/V lexemes to qualify as a flexible N/V language? If not, then how many is enough? […] How do we know that when we read a grammar of an obscure \enquote{flexible} language X that the author of the grammar has systematically surveyed the vocabulary in order to identify what proportion is flexible? If English were spoken by a small tribe in the Kordofan hills, and all we had was a 150 page grammar written fifty years ago, might it look like a highly flexible language?}

\noindent Equally significant (and equally unknown) is whether there are any commonalities among lexical items or languages which exhibit more polyfunctionality than others. These questions are relevant even if one adopts the position that polyfunctional uses of lexical items are truly heterosemous, related only historically. There remains the question of how such rampant heterosemy arises in the first place. Are there patterns or principles that guide the emergence of heterosemous forms? Whether one prefers to analyze this phenomenon as conversion, zero derivation, functional shift, polycategoriality, heterosemy, acategoriality, or something else, the fact is we do not yet have a strong empirical grasp of just how this phenomenon is realized in the world's languages. This dissertation is a first foray into filling that empirical gap. The following section describes the contribution made by this dissertation to addressing this gap and gives an overview of the present study.

\section{Overview of this study}
\label{sec:1.3}

This dissertation is a quantitative corpus-based study of lexical polyfunctionality in English (Indo-European > Germanic) and Nuuchahnulth (Wakashan > Southern Wakashan). It is exploratory and descriptive, with the primary goal of describing the prevalence of lexical polyfunctionality within and across languages. The specific research questions investigated are as follows:

\begin{enumerate}[
  label      = {\textbf{R\arabic*:}},
  leftmargin = *,
  ref        = {R\arabic*}
]
  \item\label{R1} How polyfunctional are lexical items in English and Nuuchahnulth?
  \item\label{R2} Is there a correlation between degree of lexical polyfunctionality and the size of the corpus?
  \item\label{R3} Is there a correlation between degree of lexical polyfunctionality for a lexical item and frequency (or corpus dispersion)?
  \item\label{R4} How do the semantic properties of lexical items pattern with respect to their polyfunctionality?
\end{enumerate}

I explore each of these questions from several angles. \ref{R1}, \enquote{How polyfunctional are lexical items in English and Nuuchahnulth?} is the core empirical focus of this dissertation. To answer it, I count the frequency with which stems are used for each of the three functions of reference, predication, and modification in a corpus of spoken texts for each language. In total I annotated nearly 400,000 tokens of English and 9,000 tokens of Nuuchahnulth for their discourse function. Based on these data, each stem is then given a functional diversity rating from $0$ to $1$ based on how evenly its uses are distributed across the three functions, computed using a normalized Shannon diversity/entropy index \parencite{Shannon1948}. A rating of $0$ indicates that the stem is monofunctional, with all its occurrences being used for a single function; a rating of $1$ indicates that the stem is maximally polyfunctional, with its occurrences evenly distributed across the three functions. By quantifying the polyfunctionality of each stem in this way, it then becomes possible to look for statistical correlations between the functional diversity of a stem and other factors, such as those addressed by the other two research questions. It also enables us to answer the question of just how pervasive polyfunctionality is in the two languages.

\ref{R2}, \enquote{Is there a correlation between degree of lexical polyfunctionality and the size of the corpus?}, is motivated by claims made by some researchers that all items display polyfunctionality if you examine enough of their tokens \parencite[77]{MoselHovdhaugen1992}. If true, this would lend some empirical support to the claim that all items are (or least can be) to some degree polyfunctional.

\ref{R3}, \enquote{Is there a correlation between degree of lexical polyfunctionality for a lexical item and frequency (or corpus dispersion)?}, uses the functional diversity ratings calculated in \ref{R1} to consider whether the functional diversity of a stem correlates with either its overall frequency or with its corpus dispersion. \dfn{Corpus dispersion} refers to how evenly/regularly the item appears in a corpus, a measure which is thought to more accurately capture the notion of frequency of exposure \parencites{Gries2008}{Griesfc}. This question has two motivations: First, higher-frequency items often preserve irregular or atypical forms or functions \parencite[Ch.~13]{Bybee2007}, such that items with higher frequencies might be more likely to retain their non-prototypical, polyfunctional uses. Second, the fact that a lexical item is polyfunctional means that there is a wider range of constructions it can appear in. This could reasonably result in a higher overall frequency for polyfunctional items. Both of these potential factors invite inquiry into the relationship between frequency and polyfunctionality.

\ref{R4}, \enquote{How do the semantic properties of lexical items pattern with respect to their polyfunctionality?}, is investigated using a mix of quantitative and qualitative methods. Unlike the other two research questions, which are intended to capture the extent of lexical polyfunctionality in and across languages, \ref{R4} is an inquiry into the semantic behavior of functionally diverse (and non-diverse) lexical items. This research question is directly motivated by Croft's \parencites*{Croft1991}{Croft2000}{Croft2001b}{Croftfc} typological markedness theory of lexical categories, which claims among other things that lexical items used in non-prototypical functions (for example, a property word being used to refer, as a noun) will always show a semantic shift in the direction of the meaning typically associated with that function. So, if a property word is used to refer, its meaning should be more object-like than property-like; that is, it should mean something like \tln{an entity with the property X} rather than \tln{the abstract property X}. \posscitet{Croft1991} work in this area provides empirical evidence for this principle of semantic shift, but is nonetheless somewhat preliminary. Croft himself has in various places implored linguists to investigate the lexical semantics of these functional shifts further \parencites[440]{Croft2005}[70]{CroftLier2012}, but as yet little research has responded to this call \parentext{though see \textcite{Rogers2016} and \textcite{Mithun2017}}. Investigating the semantic patterns that appear in cases of lexical polyfunctionality is therefore another contribution of this dissertation, addressed by question \ref{R4}.

A more complete description of the methods used in answering each research question is given in \chref{ch:methods}.

This study aims to be framework neutral in the sense of \textcite{Haspelmath2010b}. Its findings should be interpretable and of interest to researchers working in a range of linguistic theories and with different approaches to lexical categories. As mentioned in \secref*{sec:1.2}, the results of this study do not depend on whether one analyzes lexical polyfunctionality as polycategoriality, conversion, or something else. While my own perspective is decidedly functionalist, this is of little relevance to how I coded the data, the procedures for which are described in detail in \chref{ch:methods}. The relevant factors in this study are operationalized in a theory-neutral way (to the extent such a thing is possible), and I expect that my coding decisions for individual data points will be found largely unobjectionable. Thus some researchers may choose to view this study as an empirical investigation into the frequency of conversion in languages rather than frequency or degree of lexical polyfunctionality.

While the methods used in this study are compatible with a variety of theories of lexical polyfunctionality, I nonetheless argue in \chref{ch:background} for a cognitively informed, typological-constructional theory of word classes and flexible items. It is cognitively informed in that it treats mental categories as \dfn{prototypal} and recognizes the existence of various prototype effects in language. I also adopt a Radical Construction Grammar approach \parencite{Croft2001b} in which the basic categories in language are \dfn{constructions} rather than \dfn{parts of speech} \parentext{see also \cites{Langacker1987}{FillmoreKayOConnor1988}{Goldberg1995}{Goldberg2006}}. In construction grammar, language is viewed as a structured taxonomic network of constructions, whether those constructions are \dfn{substantive} (like words and morphemes) or \dfn{schematic} (like grammatical relations).

Several principles guided the choice of data used for this study. First, a self-imposed requirement for this project is that of empirical accountability and replicability. It should be possible for other researchers to apply the measure of lexical polyfunctionality defined in \chref{ch:methods} to new corpora, or to replicate the results of the present study on the existing dataset. As such, I only used data that were publicly available and, if possible, open access. Second, since the aim of this study is to investigate lexical polyfunctionality in actual language use, I rely solely on naturalistic data from spoken texts. This has the additional advantage of abetting comparison to other, less well documented languages since most corpora of minority languages consist mainly of spoken texts. Third, I sought to examine data from languages that have featured prominently in discussions of lexical polyfunctionality in the literature, with the intention of offering a more expansive empirical foundation for future discussions. With these principles in mind, I chose to focus this study on English and Nuuchahnulth.

\idx{English} has at various times been described as both a highly \enquote{flexible} language with fluid category membership \parencites[47--48]{Crystal1967}{Vonen1994}[75--76]{Croft2000}[69]{Croft2001b}{Farrell2001}{Cannon1985} and a fairly rigid language with clearly-delineated categories \parencites[710]{Rijkhoff2007}[4, 11, 12]{SchachterShopen2007}[122, 126]{Velupillai2012}. It is used as a point of comparison for nearly every discussion of lexical polyfunctionality, but we do not have a clear idea of just how polyfunctional items in English are. Its inclusion in this study is therefore well justified. The data for English are from the \href{http://www.anc.org/}{Open American National Corpus} (OANC), a 15-million-token corpus of American English comprising numerous genres of both spoken and written data, all of which is open access \parencite{OANC}. This study uses just the spoken portion of the corpus, consisting of approximately 3.2 million tokens, which is itself composed of two distinct subcorpora—the \href{https://newsouthvoices.uncc.edu}{Charlotte Narrative \& Conversation Collection} (or simply \enquote{the Charlotte corpus}) and the \href{https://catalog.ldc.upenn.edu/LDC97S62}{Switchboard Corpus}.

\idx{Nuuchahnulth} (formerly referred to in the literature as Nootka) is a Wakashan language presently spoken by a hundred or so people on and around Vancouver Island, British Columbia, in the Pacific Northwest. Nuuchahnulth, together with the other members of the Wakashan family (especially Makah and Kwakʼwala / Kwakiutl), is one of the widely discussed languages in the literature on lexical polyfunctionality \parencites{Swadesh1939b}{Jacobsen1979}{Braithwaite2015}. This is due largely to the following examples of flexible items from \textcite{Swadesh1939b}.

\exinfo{\idx{Nuuchahnulth} (Wakashan > Southern Wakashan)}
\begin{exe}

  \ex\label{ex:1.8}
  \begin{xlist}

    \ex
    \gll qo·ʔas‑ma        ʔi·ḥ‑ʔi\\
         man‑\gl{3sg.ind} large‑\gl{def}\\
    \tln{The large one is a man.}
    \exsource[78]{Swadesh1939b}

    \ex
    \gll ʔi·ḥ‑ma            ʔo·ʔas‑ʔi\\
         large‑\gl{3sg.ind} man‑\gl{def}\\
    \tln{The man is large.}
    \exsource[78]{Swadesh1939b}

  \end{xlist}

  \ex\label{ex:1.9}
  \begin{xlist}

    \ex
    \gll mamo·k‑ma         ʔo·ʔas‑ʔi\\
         work‑\gl{3sg.ind} man‑\gl{def}\\
    \tln{The man is working.}
    \exsource[78]{Swadesh1939b}

    \ex
    \gll ʔo·ʔas‑ma        mamo·k‑ʔi\\
         man‑\gl{3sg.ind} work‑\gl{def}\\
    \tln{The working one is a man.}
    \exsource[78]{Swadesh1939b}

  \end{xlist}

\end{exe}

Hardly a single typological survey of lexical categories or study of lexical polyfunctionality has failed to include these examples since \parentext{see especially the much-cited chapter by \citeauthor{SchachterShopen2007} \parentext{[1985] \citeyear{SchachterShopen2007}: 12}}. Yet we still do not know how representative these examples are of \idx{Nuuchahnulth} in general. What is more, lexical polyfunctionality is an areal feature of the entire Pacific Northwest. The nearby \idx{Chimakuan}, \idx{Chinookan}, \idx{Coosan}, \idx{Sahaptian}, \idx{Salishan}, and \idx{Tsimshianic} families as well as the isolate \idx{Kutenai} each exhibit lexical polyfunctionality to a presumably strong degree, since they have caught the attention of so many researchers in this regard \parentext{Chimakuan: \textcite[179]{Andrade1933}; Chinookan: \textcite{DuncanSwitzlerZenkfc}; Coosan: \textcite[318]{Frachtenberg1922}; Sahaptian: \textcite[142]{Wetzer1996}; Salishan: \textcite{Kuipers1968}, \textcite{Hebert1983}, \textcite{Kinkade1983}, \textcite{EijkHess1986}, \textcite{JelinekDemers1994}, \textcite{Mattina1996}, \textcite[§4.1.1]{Beck2002}, \textcite{Montler2003}, \textcite{Beck2013}, \textcite{DavisGillonMatthewson2014}; Tsimshianic: \textcite{DavisGillonMatthewson2014}; Kutenai: \textcite{Morgan1991}}. Again, we do not actually know whether this literature is truly representative of the pervasiveness of the phenomenon, or whether its \enquote{exotic} nature as compared to \idx{Indo-European} languages has simply garnered undue attention to the topic in this geographic region. Nuuchahnulth, being the most discussed of these languages, is therefore nearly obligatory to be included in a study such as this one.

The data used for the investigation of Nuuchahnulth come from a corpus of texts collected and edited by Toshihide Nakayama and published in \textcite{Little2003} and \textcite{Louie2003}. The corpus consists of 24 texts dictated by speakers Caroline Little and George Louie, containing 2,081 utterances and 8,366 tokens (comprising 4,216 distinct wordforms). The texts cover a variety of genres, including procedural texts, personal narratives, and traditional stories. I manually retyped these texts as \href{https://scription.digitallinguistics.io}{scription} files for analysis. Scription is a simple text format for representing interlinear glosses in a way that is both familiar to linguists and computationally parseable \parencite{Hieber2021b}. The resulting digitally searchable corpus is available on GitHub at \url{https://github.com/dwhieb/Nuuchahnulth}.

Other languages that would have been obvious choices for inclusion in this study are Riau Indonesian\index{Indonesian} (Austronesian > Malayo-Polynesian) \parencite{Gil1994}, \idx{Mundari} \parencites{EvansOsada2005}{HengeveldRijkhoff2005}, Classical Nahuatl\index{Nahuatl} (Uto-Aztecan) \parencites{Launey1994}{Launey2004}, and Central Alaskan Yup'ik\index{Yup'ik} (Eskimo-Aleut > Yupik) \parencites{Thalbitzer1922}{Sadock1999}{Mithun2017}. Each of these has generated contested claims about their polyfunctionality and the existence of polyfunctionality more generally. However, practicalities have limited me to examining just English and Nuuchahnulth for the time being. I leave investigations of other languages to future research and researchers.

Both the English and Nuuchahnulth corpora were converted to the \href{https://format.digitallinguistics.io}{Data Format for Digital Linguistics} (DaFoDiL) \parentext{a JSON format for representing linguistic data; \textcite{Hieber2021a}} for tagging and scripting purposes. This made it possible to use the \href{https://digitallinguistics.io}{Digital Linguistics} (DLx) ecosystem of tools and software to more quickly tag and analyze the data. More information about Digital Linguistics may be found at \url{https://digitallinguistics.io}.

The datasets, scripts, and source files for this dissertation are publicly available on GitHub at \url{https://github.com/dwhieb/dissertation}.

Turning now to results:

Regarding \ref{R1}, \enquote{How polyfunctional are lexical items in English and Nuuchahnulth?}, I find that \idx{English} and \idx{Nuuchahnulth} differ significantly not only in their overall degree of functional diversity, but also in how that diversity is realized. In English, the majority of items are polyfunctional, but only to a small degree. Most lexical items of English can be used for multiple discourse functions, but there is a strong tendency for each item to be used for primarily one function. The greatest degree of polyfunctionality appears between reference and modification, with many words sitting somewhere on a cline between prototypical referents and prototypical modifiers. Overall, English shows a consistent but somewhat marginal degree of polyfunctionality. In contrast, lexical items in Nuuchahnulth often exhibit a high degree of polyfunctionality, but primarily along the reference-predication axis; Nuuchahnulth lexical items are very freely used for both reference and predication, but only infrequently used as modifiers.\footnote{Crosslinguistically, modifiers are in general less frequent in discourse than referents or predicates \parencite[§3.3.2]{Croft1991}. However, Nuuchahnulth shows a low incidence of modification even when this fact is taken into account.} Property-denoting and quantity-denoting words appear much more frequently as referents and predicates than they do in modifying constructions. Nuuchahnulth thus shows a high degree of polyfunctionality, but primarily in just one dimension.

In relation to \ref{R2}, \enquote{Is there a correlation between degree of lexical polyfunctionality and size of the corpus?}, I find that once a sufficient number of tokens are encountered to establish a reliable functional diversity rating, that rating does not change noticeably as the size of the corpus continues to grow. The exact number of tokens it takes to determine a reliable functional diversity rating varies from word to word, likely due to the fact that some words appear in a wider variety of discourse contexts than others. While larger corpora do make it more likely to encounter \emph{some} polyfunctionality, the overall functional diversity rating of each word is synchronically fixed, suggesting that speakers know the specific functions that a word may be used for. The data for Nuuchahnulth are consistent with the findings for English, but the overall corpus size for Nuuchahnulth is too small to say with confidence that the same findings hold. The point of diminishing returns for functional diversity in Nuuchahnulth could be quite different from that of English.

For \ref{R3}, \enquote{Is there a correlation between degree of lexical polyfunctionality for an item and frequency (or corpus dispersion)?}, I find no significant correlations for either English or Nuuchahnulth. Given the available data, there is no evidence that polyfunctionality correlates with either frequency or corpus dispersion.

Lastly, \ref{R4} asks \enquote{How do the semantic properties of lexical items pattern with respect to their polyfunctionality?}. With respect to Nuuchahnulth, I find that property words and numerals and quantifiers are the most functionally diverse semantic class of items. Nearly all of the most polyfunctional items are of these semantic classes. Deictic expressions such as \txn{this}, \txn{that}, \txn{here}, \txn{there} also rank very highly in their functional diversity. I also find that there are strong correlations between morphologically marked aspect (durative, continuative, inceptive, etc.) and discourse function. In Nuuchahnulth, aspect markers may be used with either predicates or referents; they are not an exclusively verbal category. However, I find that the presence of any aspect marker does correlate strongly with predication, lending additional empirical evidence to \posscitet{HopperThompson1984} claim that items used in their prototypical function will show the inflectional behaviors typical of that function, and \posscitet{Croft1991} behavioral potential hypothesis. The momentaneous and telic aspect markers are the only ones in Nuuchahnulth which show any sort of tendency towards use with referents, while the durative is the only aspect marker to show any sort of tendency towards use with modifiers. Since aspect is a grammatical category that expresses how speakers construe the temporal structure of an event, these data suggest that polyfunctionality has a great deal to do with how speakers conceptualize or construe concepts—as an action, object, or property—as has been suggested by Croft (\citeyear[99]{Croft1991}; \citeyear[104]{Croft2001b}).

\idx{Nuuchahnulth} also has a definite suffix \txn{-ʔiː} used with referents. \textcite[48]{Nakayama2001} states that this suffix is used with action words being construed as objects. This observation suggests that the definite suffix may have a clarifying function, appearing whenever an action word is used for the atypical role of reference (as predicted by Croft's structural coding hypothesis; see \secref*{sec:2.4} for more details). One hypothesis that arises from applying typological markedness theory to Nuuchahnulth is that aspect markers which correspond to more object-like construals of an item (durative, telic, momentaneous) are more likely to be marked with the definite suffix. This turns out to be true, but only trivially so—only a tiny percentage (7.98\%) of tokens with definite markers also had aspect markers. However, this leads to the interesting observation that the definite marker and the aspect markers in Nuuchahnulth are \emph{almost} entirely mutually exclusive. They only rarely co-occur. These facts demonstrate that even in a language with rampant polyfunctionality, as this study shows Nuuchahnulth to be, that polyfunctionality nonetheless adheres to typological markedness patterns.

Each of the results reported above is discussed in more detail in \chref{ch:results}.

To summarize, this dissertation makes contributions in several areas. The first is methodological: this dissertation lays out a procedure for quantifying the degree of polyfunctionality for individual lexical items in a corpus that can be replicated for other languages and corpora (\chref{ch:methods}). The second is empirical and descriptive: I describe the extent of lexical polyfunctionality and the way it operates in English and Nuuchahnulth (\chref{ch:results}). The final contribution is analytical and theoretical: I argue that the data and statistical analysis presented in this dissertation support Croft's typological markedness theory of word classes, in which lexical categories such as noun, verb, and adjective are not in fact categories of particular languages as has been historically assumed, but instead are emergent patterns that arise from how speakers use object, action, and property words for different functions in discourse (reference, predication, and modification). Lexical items used for functions that are not prototypical of their meaning \emph{tend} to be more marked (morphologically, behaviorally, and/or frequentially), but this is not an absolute universal. Lexical polyfunctionality is the natural and expected result of the fact that these non-prototypical uses are \emph{not} always morphologically marked, even when they are marked in other ways (\chref{ch:conclusion}).

The remainder of this dissertation is organized as follows: \hyperref[ch:background]{Chapter 2: Background} summarizes previous definitions of lexical polyfunctionality and discusses their shortcomings. I propose an alternative, functionally-oriented definition that is consistent with cognitive and typological approaches to word classes instead. \hyperref[ch:background]{Chapter 3: Data \& Methods} describes in detail how the data were coded and analyzed for each of the major research questions (and contributing subquestions) in this study. I discuss factors that influenced how the data were coded and outline the various coding decisions that were made. I present and explain a measure of corpus dispersion that is used partly in place of, and partly as a complement to, raw frequencies of items. Lastly, I set forth a procedure for operationalizing and quantifying polyfunctionality in a crosslinguistically comparable way. \hyperref[ch:background]{Chapter 4: Results} presents the empirical findings from this study. I demonstrate how the methodological techniques from \chref{ch:methods} are applied to individual lexical items, and then present aggregated views of the data for English and Nuuchahnulth respectively. \hyperref[ch:background]{Chapter 5: Discussion \& Conclusion} considers the implications of the results in \chref{ch:results} for theories of lexical categories. I argue that the data support a typological-universal theory of word classes, and that lexical polyfunctionality should be viewed as a natural result of the cognitive and diachronic processes at work in language, rather than as an exceptional phenomenon. I conclude by discussing some limitations of the present study and avenues for future research, followed by closing remarks.
