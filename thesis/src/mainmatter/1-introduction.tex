\chapter{Introduction}
\label{ch:introduction}

\blockquote{This chapter motivates the need for research on lexical flexibility by situating it within broader concerns regarding linguistic categories more generally, and categories in human cognition. The specific problem addressed is our lack of understanding regarding what lexical flexibility looks like, and how it varies across languages. This thesis contributes to answering these questions via a quantitative corpus-based study of lexical flexibility in English (Indo-European > Germanic) and Nuuchahnulth (Wakashan > Southern Wakashan). It is the first study to examine lexical flexibility using natural discourse data from corpora. This chapter provides an overview of the thesis, including the specific research questions addressed, the data and methods used, a concise summary of the results, and a preview of the conclusions.}

\section{The \enquote{problem} of lexical flexibility}
\label{sec:1.1}

Word classes such as noun, verb, and adjective (traditionally called \dfn{parts of speech}) were once thought to be universal, easily identifiable, and easily understood. Today they are one of the most controversial and least understood aspects of language. While language scientists agree that word classes exist, there is much disagreement as to whether they are categories of individual languages, categories of language generally, categories of human cognition, categories of language science, or some combination of these possibilities \parencites[166]{Mithun2017}{Haspelmath2018}{Hieberforthcoming}. Lexical categorization—how languages assign words\footnote{In this thesis, I use the two terms \dfn{word} and \dfn{lexical item} interchangeably as a convenient cover terms for root, stem, or fully-inflected word. These terms do not here refer to phonological words, syntactic words, or any other concept of word. The reason for this vague usage is because languages vary as to which linguistic level bears category information. This issue is discussed more fully in \secref*{sec:2.3.2.3}.} to categories—is of central importance to theories of language because it is tightly interconnected with linguistic categorization generally, which in turn informs (and is informed by) our understanding of cognition. Categorization is a fundamental feature of human cognition \parencites[xi]{Taylor2003}[2--3]{LierRijkhoff2013}, and lexical categorization is perhaps the most foundational issue in linguistic theory \parencites[36]{Croft1991}[1]{VapnarskyVeneziano2017}.

One challenge for traditional theories of word classes is the existence of \dfn{lexical flexibility}—the use of a word in more than one discourse function with no overt derivational morphology, whether it is used to refer (like a noun), to predicate (like a verb), or to modify (like an adjective). In traditional terms, flexible words are those which may be used for more than one part of speech. (A more precise definition of lexical flexibility is given in \secref{sec:2.5}.) Examples of flexible words in several languages are shown below. In the examples, \textbf{N} stands for a word being used nominally, \textbf{V} for a word being used verbally, and \textbf{A} for a word being used adjectivally. The flexible word in each set of examples is shown with \em{emphasis}.

\begin{exe}

  \ex\label{ex:1.1}
  \exinfo{\idx{English} (Indo-European > Germanic)}
  \begin{xlist}

    \exi{\textbf{N:}} And the spots of \em{paint} would change every hundred degrees.
    \exsourcebelow[FrancisClem]{OANC}

    \exi{\textbf{V:}} One story does come to my mind though where you \em{painted} the foundation coating on the house and got tar all over you.
    \exsourcebelow[BorelRaymondHydellII]{OANC}

    \exi{\textbf{A:}} And it happened to be one of the rare \em{paint} jobs.
    \exsourcebelow[sw2236]{OANC}

  \end{xlist}

  \ex\label{ex:1.2}
  \exinfo{\idx{Mandinka} (Mande > Manding)}
  \begin{xlist}

    \exi{\textbf{N:}}
    \gll \em{Kuuráŋ}‑o      mâŋ          díyaa.\\
         \em{sick}‑\gl{def} \gl{pfv.neg} pleasant\\
    \tln{Sickness is not pleasant.}
    \exsource[46]{Creissels2017}

    \exi{\textbf{V:}}
    \gll Díndíŋ‑o       máŋ          \em{kuraŋ}.\\
         child‑\gl{def} \gl{pfv.neg} \em{sick}\\
    \tln{The child is not sick.}
    \exsource[46]{Creissels2017}

  \end{xlist}

  \ex\label{ex:1.3}
  \exinfo{\idx{Mundari} (Austroasiatic > Munda)}
  \begin{xlist}

    \exi{\textbf{N:}}
    \gll \em{buru}=ko                bai‑ke‑d‑a.\\
         \em{mountain}=\gl{3pl.subj} make‑\gl{compl}‑\gl{tr}‑\gl{ind}\\
    \tln{They made the mountain.}
    \exsource[354]{EvansOsada2005}

    \exi{\textbf{V:}}
    \gll saan=ko                \em{buru}‑ke‑d‑a.\\
         firewood=\gl{3pl.subj} \em{mountain‑}\gl{compl}‑\gl{tr}‑\gl{ind}\\
    \tln{They heaped up the firewood.}
    \exsource[355]{EvansOsada2005}

  \end{xlist}

  \ex\label{ex:1.4}
  \exinfo{\idx{Nuuchahnulth} (Wakashan > Southern Wakashan)}
  \begin{xlist}

    \exi{\textbf{N:}}
    \gllll watqšiƛ              ʔaƛimt            …\\
           watq‑ši(ƛ)           \em{ʔaƛa}‑imt      …\\
           swallow‑\gl{mom}     \em{two}‑\gl{past} …\\
           completely.swallowed two               …\\
    \tln{He swallowed two of them […]}
    \exsource[Qawiqaalth 57]{Louie2003}

    \exi{\textbf{V:}}
    \gllll wik̓aƛ        haʔukšiƛ     ʔaƛiičiƛ\\
           wik‑ʼaƛ      haʔuk‑ši(ƛ)  \em{ʔaƛa}‑ʽi·čiƛ\\
           not‑\gl{fin} eat‑\gl{mom} \em{two}‑\gl{incep}\\
           didn’t       ate          became.two\\
    \tln{He (Mink) didn’t eat them and the crabs became two.}
    \exsource[Mink 266]{Louie2003}

    \exi{\textbf{A:}}
    \gllll hiiɬtqyaap̓up             ʔaƛa      qʷayac̓iik\\
           hiɬ‑tqya·p̓i‑up           \em{ʔaƛa} qʷayac̓iːk\\
           there‑back‑\gl{mom.caus} \em{two}  wolf\\
           put.on.the.back          two       wolf\\
    \tln{Two wolves put (the dead wolf) on their back.}
    \exsource[FoodThief 46]{Louie2003}

  \end{xlist}

  \ex\label{ex:1.5}
  \exinfo{\idx{Quechua} (Quechuan)}
  \begin{xlist}

    \exi{\textbf{N:}}
    \gll rikaškaː \em{hatun}‑(kuna)‑ta\\
         I.saw    \em{big}‑(\gl{pl})‑\gl{acc}\\
    \tln{I saw the big one(s)}
    \exsource[17]{SchachterShopen2007}

    \exi{\textbf{V:}}
    \gll chay runa \em{hatun} (kaykan)\\
         that man  \em{big}   is\\
    \tln{that man is big}
    \exsource[17]{SchachterShopen2007}

    \exi{\textbf{A:}}
    \gll chay \em{hatun} runa\\
         that \em{big}   man\\
    \tln{that big man}
    \exsource[17]{SchachterShopen2007}

  \end{xlist}

  \ex\label{ex:1.6}
  \exinfo{\idx{Tongan} (Austronesian > Polynesian)}
  \begin{xlist}

    \exi{\textbf{N:}}
    \gll naʼe      lele e         kau         \em{fefiné}\\
         \gl{past} run  \gl{spec} \gl{pl.hum} \em{woman}.\gl{def}\\
    \tln{The women were running.}
    \exsource[134]{Broschart1997}

    \exi{\textbf{V:}}
    \gll naʼe      \em{fefine} kotoa e         kau         lelé\\
         \gl{past} \em{woman}  all   \gl{spec} \gl{pl.hum} run.\gl{def}\\
    \tln{The ones running were all female.}
    \exsource[134]{Broschart1997}

  \end{xlist}

  \ex{
    \label{ex:1.7}
    \exinfo{Central Alaskan Yup'ik\index{Yup'ik} (Eskimo-Aleut > Yup'ik)}
    \begin{xlist}

      \ex\label{ex:1.7a}
      \begin{tabularx}{\linewidth}[t]{ l p{5.25em} l }
        {         } & \txn{iqa‑}            & \tln{dirt}; \tln{be dirty}\\
        {         } & \txn{‑ngtak}          & \tln{very}\\
        \textbf{N:} & \em{\txn{iqa‑ngtak}}  & \tln{one that is very dirty}\\
        \textbf{V:} & \em{\txn{iqa‑ngtaq‑}} & \tln{be very dirty}\\
      \end{tabularx}
      \exsourcebelow[159]{Mithun2017}

      \ex\label{ex:1.7b}
      \begin{tabularx}{\linewidth}[t]{ l p{5.25em} l }
        {         } & \txn{tangerr‑}          & \tln{see}\\
        {         } & \txn{‑uaq}              & \tln{imitation, inauthentic}; \tln{pretend to, without serious purpose}\\
        \textbf{N:} & \em{\txn{tangerr‑uaq}}  & \tln{movie, vision, hallucination}\\
        \textbf{V:} & \em{\txn{tangerr‑uar‑}} & \tln{hallucinate, watch a movie}\\
      \end{tabularx}
      \exsourcebelow[159]{Mithun2017}

      \ex\label{ex:1.7c}
      \begin{tabularx}{\linewidth}[t]{ l p{5.25em} l }
        {         } & \txn{iqeq‑}        & \tln{corner of mouth}\\
        {         } & \txn{‑mik}         & \tln{thing held in one's mouth}; \tln{to put in one's}\\
        \textbf{N:} & \em{\txn{iq‑mik}}  & \tln{chewing tobacco}\\
        \textbf{V:} & \em{\txn{iq‑mig‑}} & \tln{put in one's mouth}\\
      \end{tabularx}
      \exsourcebelow[160]{Mithun2017}

    \end{xlist}
  }

\end{exe}

\noindent In the \idx{English} example in \exref{ex:1.1}, the predicative use of \txn{paint} takes the English Past Tense suffix \txn{-ed} like any prototypical verb in English, but there is no morpheme present that explicitly converts the word from noun to verb (or vice versa). The remaining examples illustrate the same situation for a variety of language families around the world. Even though in some cases there is inflectional morphology indicating the function of the word, none of these examples have explicit derivational morphology converting the target words from one function to another.

Flexible words like those in the examples above create an analytical problem for traditional theories of parts of speech. Traditional theories assume that words can be partitioned into mutually exclusive categories based on a clear set of criteria, an approach that has its roots in the Aristotelian tradition of defining a category via its necessary and sufficient conditions. Flexible words would seem to violate this assumption because they appear to be members of more than one category at once, and the criteria for classifying them yield conflicting results.

Researchers have proposed numerous solutions to this problem. The most common response is to adjust the selectional criteria so that only certain features are considered definitional of the class, allowing these researchers to dismiss other, potentially contradictory evidence as irrelevant \parentext{\textcite{Baker2003}; \textcite{Dixon2004}; \textcite{Palmer2017}; \textcite{Floyd2011} for \idx{Quechua}; \textcite{Chung2012} for \idx{Chamorro}}. It is also common to analyze different uses of a putatively flexible word as instances of \dfn{heterosemy}—that is, entirely distinct words which share the same form but belong to different word classes \parencite{Lichtenberk1991}. In this view, heterosemous words are related only historically, via a process of conversion or functional shift, in essence denying the existence of lexical flexibility \parencite{EvansOsada2005}. Another approach is to say that languages exhibiting flexibility have only some of the traditional categories. A notable example of this is Launey's \parencites{Launey1994}{Launey2004} analysis of Classical Nahuatl\index{Nahuatl} (Uto-Aztecan), which he calls an \dfn{omnipredicative} language. In this analysis, all lexical words are predicates, so there is just one giant class of verbs.

Some researchers enthusiastically embrace the existence of lexical flexibility and abandon a commitment to the traditional categories of noun, verb, and adjective. Instead they analyze flexible lexemes as belonging to a broader, flexible word class such as \enquote{flexibles}, \enquote{contentives} or \enquote{non-verbs}, etc. \parencites{HengeveldRijkhoff2005}{Luuk2010}. Other researchers abandon the commitment to word classes entirely. \idx{Mandarin} (Sino-Tibetan > Sinitic), \idx{Tagalog} (Austronesian > Philippine), \idx{Tongan} (Austronesian > Polynesian), Riau Indonesian\index{Indonesian} (Austronesian > Malayo-Polynesian), and Proto-Indo-European\index{Indo-European} have each been analyzed as lacking parts of speech by some researchers \parentext{see \textcite{Simon1937}, \textcite{McDonald2013}, and \textcite{Sun2020} for discussions of early analyses of Mandarin; \textcite{Gil1993} for Tagalog; \textcite{Broschart1997} for Tongan; \textcite{Gil1994} for Riau Indonesian; \textcite{Kastovsky1996} for Proto-Indo-European}. Within generative linguistics, the Distributed Morphology framework takes it as an assumption that all word roots are category-neutral \parencite{Siddiqi2018}. In a more functionalist orientation, \textcite{Farrell2001} argues that \emph{all} instances of flexible words (which he describes as cases of \enquote{functional shift}) involve roots underspecified for category.

Note that these differences in perspective do not arise from disagreements about the empirical facts. Researchers mostly agree on the empirical data, but disagree on the relative importance of various pieces of evidence, and on which criteria should be taken as diagnostic of a category \parencites[235]{Wetzer1992}[32]{Stassen1997}[58]{CroftvanLier2012}. Examples of contested languages include \idx{Iroquoian} \parencite{Chafe2012}, \idx{Mundari} \parencites{EvansOsada2005}{HengeveldRijkhoff2005}, \idx{Quechua} \parencites[17]{SchachterShopen2007}{Floyd2011}, and \idx{Sundanese} \parencites[352]{Robins1968}[62--63]{Hardjadibrata1985}, with many others that could be cited as well. It is rare that an argument for flexibility is refuted by linguistic facts alone \parentext{though see \citeauthor{Mithun2000}'s \parencite*{Mithun2000} response to \textcite{Sasse1988} regarding \idx{Cayuga}}.

Since analyses of lexical flexibility depend more on the theoretical commitments of the researchers involved rather than any crucial pieces of evidence, this leads to an intractable problem: researchers cannot agree on the criteria that should be considered diagnostic for a given category in a specific language (let alone crosslinguistically). Instead they partake in \dfn{methodological opportunism} \parencite[30]{Croft2001b}, choosing the evidence and criteria which best support their theoretical commitments. Discussions in the literature about the existence of a particular category in a particular language are therefore often unproductive, and devolve into debates about theoretical assumptions or the relevance or importance of various pieces of evidence, which are ultimately unresolvable \parencite[435]{Croft2005}.

This is particularly unfortunate because lexical flexibility is by no means an isolated or minor phenomenon. Additional examples like those above could be provided for many or perhaps even all the world's languages. Lexical flexibility is not as rare or marginal as traditional approaches to word classes lead one to believe. In a survey of word classes in 48 indigenous North American languages \parencite{Hieberforthcoming}, every one of the languages surveyed exhibited lexical flexibility in at least some area of the grammar (although not all authors analyzed these cases as such). In my own experience researching lexical flexibility over the last decade, I have yet to encounter a language that does not exhibit a degree of flexibility in at least some words, however marginally. The prevalence with which different areas of the grammars of the world's languages lack sensitivity to the distinctions between reference (nouns), predication (verbs), and modification (adjectives) suggests that the existence of lexical categories in a language is not necessarily a given \parencite{Hieberforthcoming}.

Indeed, given what we know from both cognitive science and diachronic linguistics, it would be surprising if clear-cut categories \emph{did} exist. Word meanings, lexical categories, and mental categories are all prototypal\footnote{In this thesis, I use the term \txn{prototypical} to mean \tln{having the properties of the prototype, exemplar, or central member of a category} and the term \txn{prototypal} to mean \tln{having a prototype structure, with central and less central members}. The term \txn{prototypal} is borrowed from the programming community, where it is used to describe programming languages (such as JavaScript) in which objects inherit properties from shared prototypes. Word classes may be described as prototypal, and their members as prototypical or non-prototypical.} \parencite{Taylor2003}, and language change is both gradual and gradient \parencites{HopperTraugott2003}{TraugottTrousdale2010} There will be more or less central members of any given category, and at any given point in time a word might be in a stage of transition or expansion from one category into another, meaning that it will show attributes of both.

Likewise, languages develop constructions dedicated to signaling the discourse functions of reference, predication, and modification over time, but at any given point in time, a language may have few or many of these constructions, and they may be at various stages of development \parencite{Vogel2000}. Given these facts, the real curiosity is how discourse functions come to be grammaticalized in language over time, not why it is that some languages lack such distinctions in certain areas of their grammars. Lexical flexibility is not so much a problem as it is a design feature of language. It is precisely the liminal categorial\footnote{In this thesis, I use the term \txn{categorical} to mean \tln{without exception; unconditional} and the term \txn{categorial} to mean \tln{having to do with categories}.} status of flexible words that makes them interesting:

\blockquote[{\cite[23]{Croft1991}}]{In the functionalist view, linguists should recognize the boundary status of the cases in question and try to understand why they are boundary cases. The major empirical fact that has led to concrete results for typology is the discovery that the cross-linguistic variation in such things as the basic grammatical distinctions is patterned.}

It is only recently that lexical flexibility has become an object of study in itself, rather than a problem to be solved. As explained above, most prior studies aim to advance a particular analysis rather than to expand empirical coverage of the phenomenon. While they often provide numerous examples, they are neither quantitative nor comprehensive. As yet, there are only a small number of empirical investigations into the extent and nature of lexical flexibility in individual languages (let alone crosslinguistically). What follows is a brief synopsis of the existing studies of this latter type.

\section{Previous research}
\label{sec:1.2}

The existing studies on the empirical extent of lexical flexibility are of two types: lexicon-based studies which examine dictionaries to determine whether words may be used for multiple functions, and corpus-based studies which examine whether and how often words are used for multiple functions in discourse.

An early lexicon-based study, though not explicitly focused on lexical flexibility, is \citeauthor{Croft1984}'s \parencite*{Croft1984} study of categories of \idx{Russian} (Indo-European > Balto-Slavic) word roots \parentext{summarized in \textcite[66]{Croft1991}}. \citeauthor{Croft1991} finds that Russian roots are unmarked, or among the least marked forms, when their semantic category (object, action, or property) aligns with their discourse function (reference, predication, or modification respectively). When roots are used for discourse functions that are atypical for their meaning—in other words, when they are used flexibly—they are marked in some way (or at least as marked as their prototypical uses). These data suggest that lexical flexibility is constrained in a principled way, by what Croft calls the \dfn{typological markedness of parts of speech} (explained in detail in \secref{sec:2.4}).

A study of \idx{Mundari} (Austroasiatic > Munda), \textcite{EvansOsada2005} conduct a dictionary analysis using a focused 105-word sample as well as a larger 5,000 word-sample. In the 105-word sample, 74 words (72\%) could be used as either noun or verb. In the larger sample, 1,953 words (52\%) could be used as both noun and verb. The complete figures for the large sample are shown in \tabref{tab:Evans-Osada-2005}. \citeauthor{EvansOsada2005} argue on the basis of these data that, because not all the words in the Mundari lexicon are flexible, Mundari is \emph{not} a flexible language. As with any whole-language typology, however, this is an oversimplification. To overlook the flexibility of these words ignores the behavior of a vast portion of the lexicon. It is exactly this flexible behavior which is of interest in this thesis. \citeauthor{EvansOsada2005}'s study constitutes an important contribution to our knowledge of the empirical extent of lexical flexibility across languages.

\begin{table}[h]
  \centering
  \caption[Percentage of words used as nouns, verbs, or both in Mundari (Austroasiatic > Munda)]{Percentage of words used as nouns, verbs, or both in \idx{Mundari} (Austroasiatic > Munda) \parencite[383]{EvansOsada2005}}
  \label{tab:Evans-Osada-2005}
  \begin{tabular}{ l r r }
    \toprule
    noun only     &   772 &  20\% \\
    verb only     & 1,099 &  28\% \\
    noun and verb & 1,953 &  52\% \\
    \midrule
    Total         & 3,824 & 100\% \\
    \bottomrule
  \end{tabular}
\end{table}

\textcite[163]{Mithun2017} also conducts a lexicon-based analysis of words roots in Central Alaskan Yup'ik\index{Yup'ik} (Eskimo-Aleut > Yupik) using \citeauthor{Jacobson2012}'s \parencite*{Jacobson2012} exhaustive dictionary, and shows that only a small minority of roots (12\%) are flexible, and can be used as both nouns and verbs. The results of this study are shown in \tabref{tab:Mithun-2017}. \citeauthor{Mithun2017} reports that the words in these groups cannot be characterized in any general or semantic way. \citeauthor{Mithun2017}'s finding that flexibility in Yup'ik is rather marginal is surprising given that Yup'ik was the focus of an extensive debate about whether the language distinguished nouns and verbs \parencite{Sadock1999}. The fixation with these marginal cases in the literature seems disproportionate to their actual frequency of occurrence, again illustrating the disconnect between research advancing a particular analysis and research aiming to improve empirical coverage of the phenomenon. Just as with \idx{Mundari}, however, it would be an oversight to simply ignore these flexible cases. Instead we should ask what accounts for the large difference in the extent of flexibility in the lexicons of Mundari versus Yup'ik.

\begin{table}[h]
  \centering
  \caption[Percentage of words used as nouns, verbs, or both in Central Alaskan Yup'ik (Eskimo-Aleut > Yupik)]{Percentage of words used as nouns, verbs, or both in Central Alaskan Yup'ik\index{Yup'ik} (Eskimo-Aleut > Yupik) \parencite[163]{Mithun2017}}
  \label{tab:Mithun-2017}
  \begin{tabular}{ l r }
    \toprule
    noun only     &  35\% \\
    verb only     &  53\% \\
    noun and verb &  12\% \\
    \midrule
    Total         & 100\% \\
    \bottomrule
  \end{tabular}
\end{table}

In summary, existing lexicon-based studies have yielded differing results, each contributing to our understanding of lexical flexibility, but there are still too few such studies to draw any general conclusions.

Corpus-based studies of lexical flexibility are also scarce. In a study of the discourse functions of property words in \idx{English} and \idx{Mandarin} (Sino-Tibetan > Sinitic), \textcite{Thompson1989} reports that predicative uses of adjectives are in fact more common than attributive (modifying) uses of adjectives in conversation. The resulting figures from this study are shown in \tabref{tab:Thompson-1989}. Some of the attributive adjectives reported in \tabref{tab:Thompson-1989} have \enquote{anaphoric head nouns} \parencite[258]{Thompson1989}, meaning that they are adjectives functioning to refer, so the figures presented are not entirely representative of the pragmatic functions of these words. The study also does not discuss the extent to which \emph{individual} words exhibit this predicate-modifier flexibility—we only have the data in aggregate—and it also excludes any prototypical nouns being used to modify. These methodological choices are appropriate for a study of the discourse uses of prototypical adjectives, but the result is that we cannot infer much about the extent of lexical flexibility in English or Mandarin from this study.

\begin{table}[h]
  \centering
  \caption[Distribution of functions of property words in English (Indo-European > Germanic) and Mandarin (Sino-Tibetan > Sinitic)]{Distribution of functions of property words in \idx{English} (Indo-European > Germanic) and \idx{Mandarin} (Sino-Tibetan > Sinitic) \parencite[253, 257]{Thompson1989}}
  \label{tab:Thompson-1989}
  \begin{tabular}{ l r r r r }
    \toprule
      {                    } & \multicolumn{2}{c}{English} & \multicolumn{2}{c}{Mandarin} \\
    \midrule
      predicative adjectives & 209 & 86\%                  & 243 & 71\% \\
      attributive adjectives &  34 & 14\%                  &  97 & 29\% \\
    \bottomrule
  \end{tabular}
\end{table}

Nonetheless, \citeauthor{Thompson1989}'s study suggests a functional underpinning to the observed flexibility in prototypical property words. She finds that property words have primarily two functions in discourse: 1) to introduce new referents; and 2) to predicate an attribute about a referent. It is therefore no surprise that property words in some languages have their own specialized constructions, since they represent a unique mix of referring and predicating functions. However it is equally unsurprising that some languages encode property concepts using either referring or predicating constructions, since prototypical adjectives exhibit behavior related to both functions.

A similar study to \citeauthor{Thompson1989}'s is \citeauthor{Croft1991}'s \parencite*[§2.5]{Croft1991} investigation of \dfn{textual markedness}, which refers to the fact that prototypical uses of a word are more frequent than non-prototypical uses of a word in texts \parentext{as might be expected by prototype theory; \cite[56]{Taylor2003}}. \citeauthor{Croft1991} counts the frequency with which object, action, and property words are used for each of the pragmatic functions of reference, predication, and modification in four languages: Quiché Maya\index{Maya} (Mayan > Quichean), \idx{North Efate} (a.k.a. Nguna; Austronesian > Malayo-Polynesian > Oceanic), and Ute (Uto-Aztecan > Numic). The resulting counts give confirmation to textual markedness theory. In all four languages, the most frequent use of words is in their prototypical function. Object words are most frequently used to refer, action words are most frequently used to predicate, and property words are most frequently used to modify. Like \citeauthor{Thompson1989}'s \parencite*{Thompson1989} study, however, we do not know these distributions for individual words. Additionally, \citeauthor{Croft1991}'s data include cases of overtly marked uses of words in non-prototypical functions, which would not be considered instances of lexical flexibility.

Finally, there are some studies which count the proportion of nouns vs. verbs. vs. adjectives in \idx{English} texts \parencites{Hudson1994}{PolinskyMagyar2020}. Again, the data are not disaggregated to the word level, so no firm conclusions can be draw about the extent of lexical flexibility.

In sum, no existing studies examine the distribution of pragmatic functions for individual words, and limit themselves to only flexible (morphologically unmarked) cases. To my knowledge, the studies just reviewed exhaust those that take an empirical approach to determining the extent of lexical flexibility in or across languages. There are numerous additional studies of lexical flexibility, but these either a) focus on particular analyses or theories of flexible words rather than attempt to expand the empirical coverage of lexical flexibility, as mentioned earlier; or b) focus on various dimensions of the \emph{behavior} of flexible words rather than studying the overall \emph{prevalence} of flexibility. This point is not a criticism, but simply a recognition of a lacuna in existing research. The emergent literature which treats lexical flexibility as a phenomenon of interest in its own right and applies empirical data to the task of understanding its behavior has advanced our knowledge of the various ways lexical flexibility can be realized, and what the constraints on that variation are. Existing research shows, for example, that lexical flexibility is constrained and shaped by the very principles that give rise to the crosslinguistic categories of noun, verb, and adjective in the first place \parencites{Croft2000}{Croft2005}{CroftvanLier2012}. This literature and its many findings are reviewed in \secref*{sec:2.3}.

There is however still much to discover about lexical flexibility. Most significantly, we do not yet know the overall prevalence of the phenomenon. Most grammatical descriptions of flexibility present a relatively small set of handpicked examples, so that we do not know how representative these examples are. \textcite[70]{Croft2001b} makes this point nicely:

\blockquote[{\cite[70]{Croft2001b}}]{How do we know that when we read a grammar of an obscure \enquote{flexible} language X that the author of the grammar has systematically surveyed the vocabulary in order to identify what proportion is flexible? If English were spoken by a small tribe in the Kordofan hills, and all we had was a 150 page grammar written fifty years ago, might it look like a highly flexible language?}

\noindent Equally significant (and equally unknown) is whether there are any commonalities among words or languages which exhibit greater flexibility than others. These questions are relevant even if one adopts the position that flexible uses of words are truly heterosemous, related only historically. There remains the question of how such rampant heterosemy arises in the first place. Are there patterns or principles that guide the emergence of heterosemous forms? Whether one prefers to analyze this phenomenon as conversion, zero derivation, functional shift, polycategoriality, heterosemy, acategoriality, or something else, the fact is we do not yet have a strong empirical grasp of just how this phenomenon is realized in the world's languages. This thesis is a first foray into filling that empirical gap. The following section describes the contribution made by this thesis to addressing this gap, and gives an overview of the present study.

\section{Overview of this study}
\label{sec:1.3}

This thesis is a quantitative corpus-based study of lexical flexibility in English (Indo-European > Germanic) and Nuuchahnulth (Wakashan > Southern Wakashan). It is exploratory and descriptive, with the primary goal of describing the prevalence of lexical flexibility within and across languages. The specific research questions investigated are as follows:

\begin{enumerate}[
  label      = {\textbf{R\arabic*:}},
  leftmargin = *,
  ref        = {R\arabic*}
]
  \item\label{R1} How flexible are words in English and Nuuchahnulth?
  \item\label{R2} Is there a correlation between degree of lexical flexibility for a word and frequency (or corpus dispersion)?
  \item\label{R3} How do the semantic properties of words pattern with respect to their flexibility?
\end{enumerate}

I explore each of these questions from several angles. \ref{R1}, \enquote{How flexible are words in English and Nuuchahnulth?} is the core focus of this thesis. To answer it, I count the frequency with which lexical words are used for each of the three functions of reference, predication, and modification in a corpus of spoken texts for each language. Each word is given a flexibility rating from 0 to 1 based on how evenly its uses are distributed across the three functions. A rating of 0 indicates that the word is highly inflexible, with all its occurrences being used for a single function; a rating of 1 indicates that the word is maximally flexible, with its occurrences evenly distributed across the three functions. By quantifying the flexibility of each word in this way, it then becomes possible to look for statistical correlations between the flexibility of a word and other factors, such as those addressed by the other two research questions. It also enables us to answer the question of just how pervasive flexibility is in the two languages.

\ref{R2}, \enquote{Is there a correlation between degree of lexical flexibility for a word and frequency (or corpus dispersion)?}, uses the flexibility ratings calculated in \ref{R1} to consider whether the flexibility of a word correlates with either its overall frequency or with its corpus dispersion. \dfn{Corpus dispersion} refers to how evenly/regularly the word appears in a corpus, a measure which is thought to more accurately capture the notion of frequency of exposure \parencites{Gries2008}{Griesforthcoming}. This question has three motivations: First, some researchers have claimed or implied that all words may exhibit flexibility if you examine enough tokens of the word \parencite[77]{MoselHovdhaugen1992}. If true, this would lend some empirical support to the claim that all words are to some degree flexible, or perhaps even acategorial. Second, higher-frequency words often preserve irregular or atypical forms or functions \parencite[Ch.~13]{Bybee2007}, such that words with higher frequencies might be more likely to retain their non-prototypical, flexible uses. Third, the fact that a word is flexible means that there is a wider range of constructions it can appear in. This could reasonably result in a higher overall frequency for flexible words. Each of these potential factors invite inquiry into the relationship between frequency and flexibility.

\ref{R3}, \enquote{How do the semantic properties of words pattern with respect to their flexibility?}, is investigated using a mix of quantitative and qualitative methods. Unlike the other two research questions, which are intended to capture the extent of flexibility in and across languages, \ref{R3} is an inquiry into the semantic \emph{behavior} of flexible (and inflexible) words. This research question is directly motivated by Croft's \parencites*{Croft1991}{Croft2000}{Croft2001b}{Croftforthcoming} typological markedness theory of lexical categories, which claims among other things that words used in non-prototypical functions (for example, a property word being used to refer, as a noun) will always show a semantic shift in the direction of the meaning typically associated with that function. So, if a property word is used to refer, its meaning should be more object-like than property-like; that is, it should mean something like \tln{an entity with the property X} rather than \tln{the abstract property X}. \citeauthor{Croft1991}'s \parencite*{Croft1991} seminal work in this area provides strong empirical evidence for this semantic markedness principle, but is nonetheless somewhat preliminary. Croft himself has in various places implored linguists to investigate the lexical semantics of these functional shifts further \parencites[440]{Croft2005}[70]{CroftvanLier2012}, but as yet little research has responded to this call \parentext{though see \textcite{Rogers2016} and \textcite{Mithun2017}}. Investigating the semantic patterns that appear in cases of lexical flexibility is therefore another contribution of this thesis, addressed by question \ref{R3}.

The preceding notes are simply a high-level summary of the principal research questions investigated in this thesis. A complete description of the methods used in answering each question is given in \chref{ch:methods}.

This study aims to be framework neutral in the sense of \textcite{Haspelmath2010b}. Its findings should be interpretable and of interest to researchers working in a range of linguistic theories and with different approaches to lexical categories. As mentioned in \secref{sec:1.2}, the results of this study do not depend on whether one analyzes lexical flexibility as polycategoriality, conversion, or something else. While my own perspective on language is decidedly functional, this is of little relevance to how I coded the data, the procedures for which are described in detail in \chref{ch:methods}. The relevant factors in this study are operationalized in a theory-neutral way (to the extent such a thing is possible), and I expect that my coding decisions for individual data points will be found largely unobjectionable. Thus some researchers may choose to view this study as an empirical investigation into the frequency of conversion in languages rather frequency or degree of lexical flexibility.

While the methods used in this study are compatible with a variety of theories of lexical flexibility, I nonetheless argue in \chref{ch:background} for a cognitively-informed, typological-constructional theory of word classes and flexible words. It is cognitively-informed in that it treats mental categories as \dfn{prototypal}, and recognizes the existence of various prototype effects in language. I also adopt a Radical Construction Grammar approach \parencite{Croft2001} in which the basic categories in language are \dfn{constructions} rather than \dfn{parts of speech} \parentext{see also \parencites{Langacker1987}{FillmoreKayOConnor1988}{Goldberg1995}}. In construction grammar, language is viewed as a structured taxonomic network of constructions, whether those constructions are \dfn{substantive} (like words and morphemes) or \dfn{schematic} (like grammatical relations).

Several principles guided the choice of data used for this study. First, a self-imposed requirement for this project is that of empirical accountability and replicability. It should be possible for other researchers to apply the measure of lexical flexibility defined in \chref{ch:methods} to new corpora, or to replicate the results of the present study on the existing dataset. As such, I only used data that were publicly available and, if possible, open access. Second, since the aim of this study is to investigate lexical flexibility in actual language \emph{use}, I rely solely on naturalistic data from spoken texts. This has the additional advantage of abetting comparison between other, less well documented languages, since the majority of corpora of minority languages consist mainly of spoken texts. Third, I sought to examine data from languages that have featured prominently in discussions of lexical flexibility in the literature, with the intention of offering a more expansive empirical foundation for future discussions. With these principles in mind, I chose to focus this study on English and Nuuchahnulth.

\idx{English} has at various times been described as both a highly flexible language with fluid category membership \parencites[47--48]{Crystal1967}{Vonen1994}[111]{Farrell2001}{Cannon1985} and a fairly rigid language with clearly-delineated categories \parencites[710]{Rijkhoff2007}[4, 11, 12]{SchachterShopen2007}[122, 126]{Velupillai2012}. It is used as a point of comparison for nearly every discussion of lexical flexibility, but we do not have a clear idea of just how flexible English words are. Its inclusion in this study is therefore well justified. The data for English are from the \href{http://www.anc.org/}{Open American National Corpus} (OANC), a 15-million word corpus of American English comprising numerous genres of both spoken and written data, all of which is open access \parencite{OANC}. This study uses just the spoken portion of the corpus, consisting of approximately 3.2 million words, which is itself composed of two distinct subcorpora—the \href{https://newsouthvoices.uncc.edu}{Charlotte Narrative \& Conversation Collection} (or simply \enquote{the Charlotte corpus}) and the \href{https://catalog.ldc.upenn.edu/LDC97S62}{Switchboard Corpus}.

\idx{Nuuchahnulth} (formerly referred to in the literature as Nootka) is a Wakashan language presently spoken by a hundred or so people on and around Vancouver Island, British Columbia, in the Pacific Northwest. Nuuchahnulth, together with the other members of the Wakashan family (especially Makah and Kwakʼwala / Kwakiutl) is one of the widely discussed languages in the literature on lexical flexibility \parencites{Swadesh1939b}{Jacobsen1979}{Braithwaite2015}. This is due largely to the following examples of flexible words from \textcite{Swadesh1939b}.

\exinfo{\idx{Nuuchahnulth} (Wakashan > Southern Wakashan)}
\begin{exe}

  \ex\label{ex:1.8}
  \begin{xlist}

    \ex
    \gll qo·ʔas‑ma        ʔi·ḥ‑ʔi\\
         man‑\gl{3sg.ind} large‑\gl{def}\\
    \tln{The large one is a man.}
    \exsource[78]{Swadesh1939b}

    \ex
    \gll ʔi·ḥ‑ma            ʔo·ʔas‑ʔi\\
         large‑\gl{3sg.ind} man‑\gl{def}\\
    \tln{The man is large.}
    \exsource[78]{Swadesh1939b}

  \end{xlist}

  \ex\label{ex:1.9}
  \begin{xlist}

    \ex
    \gll mamo·k‑ma         ʔo·ʔas‑ʔi\\
         work‑\gl{3sg.ind} man‑\gl{def}\\
    \tln{The man is working.}
    \exsource[78]{Swadesh1939b}

    \ex
    \gll ʔo·ʔas‑ma        mamo·k‑ʔi\\
         man‑\gl{3sg.ind} work‑\gl{def}\\
    \tln{The working one is a man.}
    \exsource[78]{Swadesh1939b}

  \end{xlist}

\end{exe}

Hardly a single typological survey of lexical categories or study of lexical flexibility has failed to include these examples since. Yet we still do not know how representative these examples are of \idx{Nuuchahnulth} in general. What is more, lexical flexibility is an areal feature of the entire Pacific Northwest. The nearby \idx{Chimakuan}, \idx{Chinookan}, \idx{Coosan}, \idx{Sahaptian}, \idx{Salishan}, and \idx{Tsimshianic} families as well as the isolate \idx{Kutenai} each exhibit lexical flexibility to a presumably strong degree, since they have caught the attention of so many researchers in this regard \parentext{Chimakuan: \textcite[179]{Andrade1933}; Chinookan: \textcite{DuncanSwitzlerZenk2023}; Coosan: \textcite[318]{Frachtenberg1922}; Sahaptian: \textcite[142]{Wetzer1996}; Salishan: \textcite{Kuipers1968}, \textcite{Hebert1983}, \textcite{Kinkade1983}, \textcite{EijkHess1986}, \textcite{JelinekDemers1994}, \textcite{Mattina1996}, \textcite[135--169]{Beck1999}, \textcite{Montler2003}, \textcite{Beck2013}, \textcite{DavisGillonMatthewson2014}; Tsimshianic: \textcite{DavisGillonMatthewson2014}; Kutenai: \textcite{Morgan1991}}. Again, we do not actually know whether this literature is truly representative of the pervasiveness of the phenomenon, or whether its \enquote{exotic} nature as compared to \idx{Indo-European} languages has simply garnered undue attention to the topic in this geographic region. Nuuchahnulth, being the most discussed of these languages, is therefore nearly obligatory to include in a study such as this one.

The data used for the investigation of Nuuchahnulth come from a corpus of texts collected and edited by Toshihide Nakayama and published in \textcite{Little2003} and \textcite{Louie2003}. The corpus consists of 24 texts dictated by speakers Caroline Little and George Louie, containing 2,081 utterances and 8,366 tokens (comprising 4,216 types). The texts cover a variety of genres, including procedural texts, personal narratives, and traditional stories. I manually retyped these texts as \href{https://scription.digitallinguistics.io}{scription} files for analysis. Scription is a simple text format for representing interlinear glosses in a way that is both familiar to linguists and computationally parseable \parencite{Hieber2020b}. The resulting digitally-searchable corpus is available on GitHub at \url{https://github.com/dwhieb/Nuuchahnulth}.

Other languages that would have been obvious choices for inclusion in this study are Riau Indonesian\index{Indonesian} (Austronesian > Malayo-Polynesian) \parencite{Gil1994}, \idx{Mundari} (Austroasiatic > Munda) \parencites{EvansOsada2005}{HengeveldRijkhoff2005}, Classical Nahuatl\index{Nahuatl} (Uto-Aztecan) \parencites{Launey1994}{Launey2004}, and Central Alaskan Yup'ik\index{Yup'ik} (Eskimo-Aleut > Yupik) \parencites{Thalbitzer1922}{Sadock1999}{Mithun2017}. Each of these has generated contested claims about their flexibility and the existence of flexibility more generally. However, practicalities have limited me to examining just English and Nuuchahnulth for the time being. I leave investigations of other languages to future research and researchers.

Both the English and Nuuchahnulth corpora were converted to the \href{https://format.digitallinguistics.io}{Data Format for Digital Linguistics} (DaFoDiL) \parentext{a JSON format for representing linguistic data; \textcite{Hieber2020a}} for tagging and scripting purposes. This made it possible to use the \href{https://digitallinguistics.io}{Digital Linguistics} (DLx) ecosystem of tools and software to more quickly tag and analyze the data. More information about Digital Linguistics may be found at \url{https://digitallinguistics.io}.

All of the datasets, scripts, and source files for this thesis are publicly available on GitHub at \url{https://github.com/dwhieb/dissertation}.

Turning now to results:

Regarding \ref{R1}, \enquote{How flexible are words in English and Nuuchahnulth?}, I find that \idx{English} and \idx{Nuuchahnulth} differ significantly not only in their overall degree of flexibility, but also in how that flexibility is realized. In English, the majority of words surveyed are flexible, but only to a small degree. Most lexical words of English can be used as nouns, verbs, or adjectives, but there is a strong tendency for each word to be used for primarily one function. English thus shows a consistent but somewhat marginal degree of flexibility. In contrast, most words in Nuuchahnulth are highly flexible, but primarily along the noun-verb axis; Nuuchahnulth words are very freely used as both nouns and verbs, but only infrequently used as adjectives. Nuuchahnulth thus shows a consistently high degree of flexibility, but primarily in just one dimension.

For \ref{R2}, \enquote{Is there a correlation between degree of lexical flexibility for a word and frequency (or corpus dispersion)?}, I find that higher frequency words are more flexible than lower frequency words, but that the effect is very small. The same facts hold when comparing degree of lexical flexibility with corpus dispersion. Words that are more evenly dispersed in a corpus have a slight tendency to be more flexible than those that are less evenly dispersed. These findings suggest that the degree of flexibility exhibited by a word does depend in part on how regularly speakers use it.

Lastly, \ref{R3} asks \enquote{How do the semantic properties of words pattern with respect to their flexibility?}. With respect to Nuuchahnulth, I find that property words, especially numerals and quantifiers, are the most flexible semantic class of words. Nearly all of the most flexible words denote property concepts. Deictic expressions such as \txn{this}, \txn{that}, \txn{here}, \txn{there} also rank very highly in their flexibility. I also find that there are strong correlations between morphologically marked aspect (durative, continuative, inceptive, etc.) and discourse function. In Nuuchahnulth, aspect markers may be used with either predicates or referents; they are not an exclusively verbal category. However, I find that the presence of any aspect marker does correlate strongly with predication, lending additional empirical evidence to \citeauthor{HopperThompson1984}'s \parencite*{HopperThompson1984} claim that items used in their prototypical function will show the inflectional behaviors typical of that function. The momentaneous and telic aspect markers are the only ones in Nuuchahnulth which show any sort of tendency towards use with referents, while the durative was the only aspect marker to show any sort of tendency towards use with modifiers. Since aspect is a grammatical category that expresses how speakers construe the temporal structure of an event, these data suggest that flexibility has a great deal to do with how speakers conceptualize or construe words—as an action, object, or property—as has been suggested by Croft (\citeyear[99]{Croft1991}; \citeyear[104]{Croft2001b}).

\idx{Nuuchahnulth} also has a definite suffix \txn{-ʔiː} used with referents. \textcite[48]{Nakayama2001} states that this suffix is used with action words being construed as objects. This observation suggests that the definite suffix may have a clarifying function, appearing whenever an action word is used for the atypical role of reference (as predicted by Croft's structural coding hypothesis; see \secref{sec:2.4} for more details). One hypothesis that arises from applying typological markedness theory to Nuuchahnulth is that aspect markers which correspond to more object-like construals of a word (durative, telic, momentaneous) are more likely to be marked with the definite suffix. This turns out to be true, but only trivially so—only a tiny percentage (7.98\%) of words with definite markers also had aspect markers. However, this leads to the far more interesting observation that the definite marker and the aspect markers in Nuuchahnulth are \emph{almost} entirely mutually exclusive. They only rarely co-occur. These facts demonstrate that even in a language with rampant flexibility, as this study shows Nuuchahnulth to be, flexibility is nonetheless bound by universal typological constraints.

To summarize, this thesis makes contributions in several areas. The first is methodological: this thesis lays out a procedure for quantifying lexical flexibility for individual words in a corpus that can be replicated for other languages and corpora (\chref{ch:methods}). The second is empirical and descriptive: I describe the extent of lexical flexibility and the manner in which it operates in English and Nuuchahnulth (\chref{ch:results}). The final contribution is analytical and theoretical: I argue that the data and statistical analysis presented in this thesis support Croft's typological markedness theory of word classes, in which lexical categories such as noun, verb, and adjective are not in fact categories of particular languages as has been historically assumed, but instead are emergent patterns that arise from how speakers use object, action, and property words for different functions in discourse (reference, predication, and modification). Words used for functions that are not prototypical of their meaning \emph{tend} to be more marked (morphologically, behaviorally, semantically, and/or frequentially) than prototypical uses, but this is not an absolute universal. Lexical flexibility is the natural and expected result of the fact that these non-prototypical uses are \emph{not} always \emph{morphologically} marked, even when they are marked in other ways (\chref{ch:conclusion}).

The remainder of this thesis is organized as follows: \hyperref[ch:background]{Chapter 2: Background} summarizes previous definitions of lexical flexibility and discusses their shortcomings. I propose an alternative, functionally-oriented definition that is consistent with cognitive and typological approaches to word classes instead. \hyperref[ch:background]{Chapter 3: Data \& Methods} describes in detail how the data were coded and analyzed for each of the major research questions (and contributing subquestions) in this study. I discuss factors that influenced how the data were coded, and outline the various coding decisions that were made. I present and explain a measure of corpus dispersion that is used partly in place of, and partly as a complement to, raw frequencies of words. Lastly, I set forth a procedure for operationalizing and quantifying lexical flexibility in a crosslinguistically comparable way. \hyperref[ch:background]{Chapter 4: Results} presents the empirical findings from this study. I demonstrate how the methodological techniques from \chref{ch:methods} are applied to individual words, and then present aggregated views of the data for English and Nuuchahnulth respectively. \hyperref[ch:background]{Chapter 5: Discussion \& Conclusion} considers the implications of the results in \chref{ch:results} for theories of lexical categories. I argue that the data support a typological-universal theory of word classes, and that lexical flexibility should be viewed as a natural result of the cognitive and diachronic processes at work in language, rather than as an exceptional phenomenon. I conclude by discussing some limitations of the present study and avenues for future research, followed by closing remarks.
