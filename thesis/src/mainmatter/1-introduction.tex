\chapter{Introduction}
\label{ch:introduction}

\blockquote{This chapter motivates the need for research on lexical flexibility by situating it within broader concerns regarding linguistic categories more generally, and categories in human cognition. The specific problem that this study seeks to address is our lack of understanding regarding what lexical flexibility looks like, and how it varies across languages. This thesis contributes to answering these questions via a quantitative corpus-based study of lexical flexibility in English (Indo-European)\index{English} and Nuuchahnulth (Wakashan). It is the first study to examine lexical flexibility using natural discourse from corpus data. This chapter provides an overview of the thesis, including the specific research questions addressed, the data and methods used, a concise summary of the results, and a preview of the conclusions.}

\section{The \enquote{problem} of lexical flexibility}

Word classes such as noun, verb, and adjective (traditionally called \dfn{parts of speech}) were once thought to be universal, easily identifiable, and easily understood. Today they are one of the most controversial and least understood aspects of language. While language scientists generally agree that word classes exist, there is much disagreement as to whether they are categories of individual languages, categories of language generally, categories of human cognition, categories of language science, or some combination of these possibilities \addcite{Mithun 2017: 166; Haspelmath 2018; Hieber forthcoming: 1}. Lexical categorization—how languages separate words into categories—is of central importance to theories of language because it is tightly interconnected with linguistic categorization generally, which in turn informs (and is informed by) our understanding of cognition. Categorization is a fundamental feature of human cognition \addcite{Taylor 2003: xi}, and lexical categorization is perhaps the most foundational issue in linguistic theory \addcite{Croft 1991: 36; Vapnarsky \& Veneziano 2017: 1}.

One challenge for traditional theories of word classes is the existence of \dfn{lexical flexibility}—the use of a word in more than one discourse function, whether to refer (as a noun), to predicate (as a verb), or to modify (as an adjective). In traditional terms, flexible words are those which may be used for more than one part of speech. (A more precise definition of lexical flexibility is given in \addcrossref{Sec. XX [of Ch. 2]}.) Examples of flexible words in several languages are shown below.

\vspace{1em}
\todo[
  caption = {add lexical flexibility examples to intro},
  inline,
  size = normalsize
]{\begin{spacing}{1}Give examples here. Discuss all the examples together in the lead out. Examples: English; Nuuchahnulth: Kingfisher 202\end{spacing}}

Flexible words like those in the examples above create an analytical problem for traditional theories of parts of speech. Traditional theories assume that words can be partitioned into mutually exclusive categories on the basis of a clear set of criteria, an approach that has its roots in the Aristotelian tradition of defining a category via its necessary and sufficient conditions. Flexible words would seem to violate this assumption because they appear to be members of more than one category at once, and the criteria for classification yield conflicting results.

Researchers have proposed numerous solutions to this problem. The most common response is to adjust the selectional criteria so that only certain features are considered definitional of the class, allowing these researchers to dismiss other, potentially contradictory evidence as irrelevant \addcite{Baker 2003; Dixon 2004; Floyd 2011 for Quechua; Chung 2012 for Chamorro; Palmer 2017}. It is also common to analyze different uses of a putatively flexible word as instances of \dfn{heterosemy}—that is, entirely distinct words which share the same form but belong to different word classes \addcite{Lichtenberk 1991}. In this view, heterosemous words are related only historically, via a process of conversion or functional shift, in essence denying the existence of lexical flexibility \addcite{Evans \& Osada 2005}. Another approach is to claim that, while all words can be neatly categorized, some words in some languages may nonetheless be used for functions typically associated with other categories. A notable example of this is Launey's \addcite{1994?, 2001?} analysis of Classical Nahuatl, which he calls an \dfn{omnipredicative} language. In this analysis, Classical Nahuatl has the traditional, clearly-delineated word classes of noun and verb, but allows for any word to function as a verb regardless of its category (hence the term \dfn{omnipredicative}). The reverse is not true however; only some verbs may function as nouns. \TODO[confirm Launey's analysis]{Make sure that this is an accurate description of the empirical facts of Classical Nahuatl.} This difference in behavior is taken as the basis for a categorical\footnote{Throughout this thesis, I use the term \txn{categorical} to mean \tln{without exception; unconditional} and the term \txn{categorial} to mean \tln{having to do with categories}.} distinction between nouns and verbs.

Some researchers enthusiastically embrace the existence of lexical flexibility and abandon a commitment to the traditional categories of noun, verb, and adjective. Instead they analyze flexible lexemes as belonging to a broader, flexible word classes such as \enquote{flexibles}, \enquote{contentives} or \enquote{non-verbs}, etc. \addcite{Hengeveld \& Rijkhoff 2005; Luuk 2010?}. Other researchers abandon the commitment to word classes entirely. Mandarin, Tagalog, Tongan, and Riau Indonesian have each been analyzed as lacking parts of speech \addcite{Simon [1937], McDonald [2013], and Sun [2020] for discussions of early analyses of Mandarin; Gil [XXXX] for Tagalog; Broschart [XXXX] for Tongan; Gil [XXXX] for Riau Indonesian}. Within generative linguistics, the Distributed Morphology framework takes it as an assumption that all word roots are category-neutral \addcite{Siddiqi 2018}.

Note that these differences in perspective do not arise from disagreements about the empirical facts of each language. Researchers mostly agree on the empirical data, but disagree on the relative importance of various pieces of evidence, and on which criteria should be taken as diagnostic of a category \addcite{Croft \& van Lier 2012: 58}. \TODO[add examples of disagreements]{Add example from Stassen [1997: 32] about two researchers coming to different conclusions about Sudanese. Also the example of Quechua or perhaps Iroquoian (Chafe). Also Mundari. Chamorro: Topping and Chung. Among many others.} It is rare that an argument for flexibility is refuted on the basis of the linguistic facts alone \addcite{cite Mithun's response to Sasse's analysis of Cayuga as a flexible language}.

Since analyses of lexical flexibility depend more on the particular theoretical commitments of the researchers involved rather than any particular crucial pieces of evidence, this leads to an intractable problem: researchers cannot agree on the criteria that should be considered diagnostic for a given category in a specific language (let alone crosslinguistically). Instead they partake in \dfn{methodological opportunism} \addcite{Croft 2003?: ??}, choosing the evidence and criteria which best support their theoretical commitments. Discussions in the literature about the existence of a particular category in a particular language are therefore often unproductive, and devolve into debates about theoretical assumptions or the relevance or importance of various pieces of evidence, which are ultimately unresolvable \addcite{Croft 2005: 435}.

This is particularly unfortunate because lexical flexibility is by no means an isolated or minor phenomenon. Additional examples like those above could be provided for many or perhaps even all of the world's languages. Lexical flexibility is not as rare or marginal as traditional approaches to word classes lead one to believe. In a survey of word classes in 48 indigenous North American languages \addcite{Hieber forthcoming}, every one of the languages surveyed exhibited lexical flexibility in at least some area of the grammar (although not all authors analyzed these cases as such). In my own experience researching lexical flexibility over the last decade, I have yet to encounter a language that does not exhibit a degree of flexibility in at least some words, however marginally. The prevalence with which different areas of the grammars of the world's languages lack sensitivity to the distinctions between reference (nouns), predication (verbs), and modification (adjectives) suggests that the existence of lexical categories in a language is not necessarily a given \addcite{Hieber forthcoming}.

Indeed, given what we know from both cognitive science and diachronic linguistics, it would be surprising if clear-cut categories \emph{did} exist. Word meanings, lexical categories, and mental categories are all prototypal \TODO[caption]{add footnote about prototypal vs. prototypical} \addcite{Taylor 2003}, and language change is both gradual and gradient \addcite{Gradualness and gradience in grammaticalization; Grammaticalization (Hopper \& Traugott); Diachronic Construction Grammar}. There will be more or less central members of any given category, and at any given point in time a word might be in a stage of transition or expansion from one category into another, meaning that it will show attributes of both. Given these facts, the real curiosity is how discourse functions come to be grammaticalized in language over time, not why it is that some languages lack such distinctions in certain areas of their grammars. Lexical flexibility is not so much of a problem as it is a design feature of language. It is precisely the liminal categorial status of flexible words that makes them interesting:

\blockquote[\addcite{Croft 1991: 23}]{In the functionalist view, linguists should recognize the boundary status of the cases in question and try to understand why they are boundary cases. The major empirical fact that has led to concrete results for typology is the discovery that the cross-linguistic variation in such things as the basic grammatical distinctions is patterned.}

It is only recently that lexical flexibility has become an object of study in itself, rather than a problem to be solved. As explained above, most prior studies aim to advance a particular analysis rather than to expand empirical coverage of the phenomenon. While they often provide numerous examples, they are neither quantitative nor comprehensive. As yet, there are only a small number of empirical investigations into the extent and nature of lexical flexibility in individual languages (let alone crosslinguistically). What follows is a brief synopsis of the existing studies of this latter type.

\section{Previous research}

The existing studies on the empirical extent of lexical flexibility are of two types: lexicon-based studies which examine dictionaries to determine whether words may be used for multiple functions, and corpus-based studies which examine whether and how often words are used for multiple functions in discourse.

An early lexicon-based study, though not explicitly focused on lexical flexibility, is Croft's \addcite{1984} study of categories of Russian word roots (summarized in \addcite{Croft 1991: 66}). Croft finds that Russian roots are unmarked, or among the least marked forms, when their semantic category (object, action, or property) aligns with their discourse function (reference, predication, or modification respectively). When roots are used for discourse functions that are atypical for their meaning—in other words, when they are used flexibly—they are marked in some way (or at least as marked as their prototypical uses). These data suggest that lexical flexibility is constrained in a principled way, by what Croft calls the \dfn{typological markedness of parts of speech} (explained in detail in \addcrossref{Sec. XX [in Chapter 2]}).

In arguing that Mundari is \emph{not} a flexible language, Evans \& Osada \addcite{2005} conduct a dictionary analysis using a focused 105-word sample as well as a larger 5,000 word-sample. In the 105-word sample, 74 words (72\%) could be used as either noun or verb. In the larger sample, 1,953 words (52\%) could be used as both noun and verb. The complete figures for the large sample are shown in \tabref{tab:Evans-Osada-2005}. Evans \& Osada argue on the basis of these data that, because not all the words in the Mundari lexicon are flexible, Mundari cannot be considered a flexible language. As with any whole-language typology, however, this is an oversimplification. To overlook the flexibility of these words ignores the behavior of a vast portion of the lexicon. It is exactly this behavior which is of interest in this thesis. Evans \& Osada's study constitutes an important contribution to our knowledge of the empirical extent of lexical flexibility across languages.

\begin{table}[h]
  \centering
  \caption[Percentage of words used as nouns, verbs, or both in Mundari (Austroasiatic > Munda; India)]{Percentage of words used as nouns, verbs, or both in Mundari (Austroasiatic > Munda; India) (Evans \& Osada 2005: 383)}
  \label{tab:Evans-Osada-2005}
  \begin{tabular}{ l r r }
    \toprule
    noun only     &   772 &  20\% \\
    verb only     & 1,099 &  28\% \\
    noun and verb & 1,953 &  52\% \\
    \midrule
    Total         & 3,824 & 100\% \\
    \bottomrule
  \end{tabular}
\end{table}

\addcite{Add Evans \& Osada [2005: 383] citation to caption in this table.}

Creissels \addcite{2017} is a careful lexicon-based of flexibility in Mandinka (Mande; West Africa). While Mandinka has nominal and verbal constructions that allow the predicative and referring functions of words to be distinguished unambiguously, it is not as easy to separate word stems themselves into similar classes, owing to the fact that no Mandinka lexemes are used exclusively in verbal constructions—all Mandinka lexemes may occur in nominal constructions as well. While Creissels does not dispute this fact, he shows that there is a crucial distinction to be made between two classes of word stems: 1) those whose nominal use is predictable and therefore analyzable as a case of \enquote{morphologically unmarked nominalization} (zero-marked conversion) from one category (verb) to another (noun)—these are always event nominalizations; and 2) those whose meaning in nominal constructions is idiosyncratic and therefore not predictable. Creissels calls the former \dfn{verbal} words and the latter \dfn{verbo-nominal}. He states that both word classes exhibit categorial flexibility, just of different natures. There is also a small set of nominal words used marginally as verbs. These cases are always semantically predictable. Even individual senses of a word can sometimes show varying behavior as to their flexibility. Although Creissels' study unfortunately does not provide counts of the different stem types, it nonetheless adds to our understanding of lexical flexibility by showing how it may have varied realizations, within a single language or even a single word.

Mithun \addcite{2017: 163} also conducts a lexicon-based analysis of words {{roots? stems?}} in Central Alaskan Yup'ik (Alaska; Eskaleaut) using Jacobson's \addcite{year?} exhaustive dictionary, and shows that only a small minority of {{roots? stems?}} (12\%) exhibit flexibility and can be used as both nouns and verbs. The results of this study are shown in \tabref{tab:Mithun-2017}. The words in these groups cannot be characterized in any general or semantic way \addcite{Mithun 2017: 163}. Mithun's finding that flexibility in Yup'ik is rather marginal is surprising given that Yup'ik was the focus of an extensive debate about whether the language distinguished nouns and verbs \addcite{just cite Jacobson here}. The fixation with these marginal cases in the literature seems disproportionate to their actual frequency of occurrence, again illustrating the disconnect between research advancing a particular analysis and research aiming to improve empirical coverage of the phenomenon. Just as with Mundari, however, it would be an oversight to simply ignore these flexible cases. Instead we should ask what accounts for the large difference in the extent of flexibility in the lexicons of Mundari versus Yup'ik.

\begin{table}[h]
  \centering
  \caption[Percentage of words used as nouns, verbs, or both in Central Alaskan Yup'ik (Eskaleaut > Eskimo; Alaska)]{Percentage of words used as nouns, verbs, or both in Central Alaskan Yup'ik (Eskaleaut > Eskimo; Alaska) (Mithun 2007: 163)}
  \label{tab:Mithun-2017}
  \begin{tabular}{ l r }
    \toprule
    noun only     &  35\% \\
    verb only     &  53\% \\
    noun and verb &  12\% \\
    \midrule
    Total         & 100\% \\
    \bottomrule
  \end{tabular}
\end{table}

\addcite{Add Mithun [2017: 163] citation to caption in this table.}

In summary, existing lexicon-based studies have yielded a range of results, each contribution to our understanding of lexical flexibility, but there are still too few such studies to draw any general conclusions as of yet.

Corpus-based studies of lexical flexibility are also scarce. In a study of the discourse functions of property words in English and Mandarin (Sino-Tibetan > Sinitic; China), Thompson \addcite{1989} reports that predicative uses of adjectives are in fact more common than attributive (modifying) uses of adjectives in conversation. The resulting figures from this study are shown in \tabref{tab:Thompson-1989}.

\begin{table}[h]
  \centering
  \caption[Distribution of functions of property words in English (Indo-European > Germanic; England) and Mandarin (Sino-Tibetan > Sinitic; China)]{Distribution of functions of property words in English (Indo-European > Germanic; England) and Mandarin (Sino-Tibetan > Sinitic; China) (Thompson 1989: 253, 257)}
  \label{tab:Thompson-1989}
  \begin{tabular}{ l r r r r }
    \toprule
                             & \multicolumn{2}{c}{English} & \multicolumn{2}{c}{Mandarin} \\
    \midrule
      predicative adjectives & 209 & 86\%                  & 243 & 71\% \\
      attributive adjectives &  34 & 14\%                  &  97 & 29\% \\
    \bottomrule
  \end{tabular}
\end{table}

\noindent Some of the attributive adjectives reported in \tabref{tab:Thompson-1989} have \enquote{anaphoric head nouns} \addcite{Thompson 1989: 258}, meaning that they are adjectives functioning to refer, so the figures presented are not entirely representative of the pragmatic functions of these words. The study also does not discuss the extent to which \emph{individual} words exhibit this predicate-modifier flexibility—we only have the data in aggregate—and it also excludes any prototypical nouns being used to modify. These methodological choices are appropriate for a study of the discourse uses of prototypical adjectives, but the result is that we cannot infer much about the extent of lexical flexibility in English or Mandarin from this study.

Nonetheless, Thompson's study suggests a functional underpinning to the observed flexibility in prototypical property words. She finds that property words have primarily two functions in discourse: 1) to introduce new referents; and 2) to predicate an attribute about a referent. It is therefore no surprise that property words in some languages have their own specialized constructions, since they represent a unique mix of referring and predicating functions. Likewise it is unsurprising that languages would encode property concepts using either referring or predicating constructions, since prototypical adjectives exhibit behavior related to both functions.

A similar study is Croft's \addcite{1991: Sec. 2.5} investigation of \dfn{textual markedness}, which refers to the fact that prototypical uses of a word are more frequent than non-prototypical uses of a word in texts (as is generally predicted by prototype theory; \addcite{Taylor 2003: ??}). Croft counts the frequency with which object, action, and property words are used for each of the pragmatic functions of reference, predication, and modification, and the resulting counts give confirmation to textual markedness theory. Moreover, the data partially elucidate how frequently words of different semantic classes are used for multiple pragmatic functions. Like Thompson's \addcite{1989} study, however, we do not know these distributions for individual words. Additionally, Croft's data include cases of overtly marked uses of words in non-prototypical functions, which would not be considered instances of lexical flexibility.

There are also some studies which count the proportion of nouns vs. verbs. vs. adjectives in texts \addcite{Hudson 1994; Polinsky \& Magyar 2020}, but again the data are not disaggregated to the word level, so no firm conclusions can be draw about the extent of lexical flexibility.

In sum, no existing studies examine the distribution of pragmatic functions for individual words, or limit themselves to only flexible (morphologically unmarked) cases. To my knowledge, the studies just reviewed exhaust those that take an empirical approach to determining the extent of lexical flexibility in or across languages. There are numerous additional studies of lexical flexibility, but these either a) focus on particular analyses or theories of flexible words rather than attempt to expand the empirical coverage of lexical flexibility, as mentioned earlier; or b) focus on various dimensions of the \emph{behavior} of flexible words rather than studying the overall \emph{prevalence} of flexibility. This point is not a criticism, but simply a recognition of a lacuna in existing research. The emergent literature which treats lexical flexibility as a phenomenon of interest in its own right and applies empirical data to the task of understanding its behavior has advanced our knowledge of the various ways lexical flexibility can be realized, and what the constraints on that variation are. Existing research shows, for example, that lexical flexibility is constrained and shaped by the very principles that give rise to the crosslinguistic categories of noun, verb, and adjective in the first place \addcite{Croft 2000; Croft 2005; Croft \& van Lier 2012}. This literature and its many findings are reviewed in \addcrossref{Sec. XX [in Chapter 2]}.

There is however still much to discover about lexical flexibility. Most significantly, we do not yet know the overall prevalence of the phenomenon. Most grammatical descriptions of flexibility present a relatively small set of handpicked examples, so that we do not know how representative these examples are. Croft \addcite{Croft 2001: 70} makes this point nicely:

\blockquote[\addcite{Croft 2001: 70}]{How do we know that when we read a grammar of an obscure \enquote{flexible} language X that the author of the grammar has systematically surveyed the vocabulary in order to identify what proportion is flexible? If English were spoken by a small tribe in the Kordofan hills, and all we had was a 150 page grammar written fifty years ago, might it look like a highly flexible language?}

\noindent Equally as significant (and equally as unknown) is whether there are any commonalities among words or languages which exhibit greater flexibility than others. These questions are relevant even if one adopts the position that flexible uses of words are truly heterosemous, related only historically. There remains the question of how such rampant heterosemy arises in the first place. Are there patterns or principles to the emergence of heterosemous forms? Whether one prefers to analyze analyzes this phenomenon as conversion, zero derivation, functional shift, polycategoriality, heterosemy, acategoriality, or something else, the fact is we do not yet have a strong empirical grasp of just how this phenomenon is realized in the world's languages. This thesis is a first foray into filling that empirical gap. The following section describes the contribution made by this thesis to addressing this gap, and gives an overview of the present study.

\section{Overview of this study}

This thesis is a quantitative corpus-based study of lexical flexibility in English (Indo-European > Germanic; England) and Nuuchahnulth (Wakashan; Pacific Northwest). It is primarily exploratory and descriptive, with the goal of describing the prevalence of lexical flexibility within and across languages.
