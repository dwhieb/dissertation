\chapter{Introduction}
\label{ch:introduction}

\blockquote{This chapter motivates the need for research on lexical flexibility by situating it within broader concerns regarding linguistic categories more generally, and categories in human cognition. The specific problem that this study seeks to address is our lack of understanding regarding what lexical flexibility looks like, and how it varies across languages. This thesis contributes to answering these questions via a quantitative corpus-based study of lexical flexibility in English (Indo-European) and Nuuchahnulth (Wakashan). It is the first study to examine lexical flexibility using natural discourse from corpus data. This chapter provides an overview of the thesis, including the specific research questions addressed, the data and methods used, a concise summary of the results, and a preview of the conclusions.}

Word classes such as noun, verb, and adjective were once thought to be universal, easily identifiable, and easily understood. Today they are one of the most controversial and least understood aspects of language. While language scientists generally agree that word classes exist, there is much disagreement as to whether they are categories of individual languages, categories of language generally, categories of human cognition, categories of language science, or some combination of these possibilities \addcite{Mithun 2017: 166; Haspelmath 2018; Hieber forthcoming: 1}. Lexical categorization—how languages separate words into categories—is of central importance to theories of language because it is tightly interconnected with linguistic categorization more generally, which in turn informs (and is informed by) our understanding of cognition. Categorization is a fundamental feature of human cognition \addcite{Taylor 2003: xi}, and lexical categorization is perhaps the most foundational issue in linguistic theory \addcite{Croft 1991: 36; Vapnarsky \& Veneziano 2017: 1}.
