\chapter{Results}
\label{ch:results}

\blockquote{This chapter reports the results of applying the procedures described in \hyperref[ch:methods]{Chapter 3: Data \& Methods}. I begin by demonstrating for the reader how to interpret the ternary plots used to visually represent the degree of lexical flexibility for individual items (\ref{R1}) (\secref*{sec:4.2}). Next I look at the flexibility of lexical items in English and Nuuchahnulth, both independently and in comparison (\secref*{sec:4.3}). I then I investigate whether lexical flexibility depends on corpus size (\ref{R2}) (\secref*{sec:4.4}). Next, examine the relationship between the degree of lexical flexibility and frequency / dispersion (\ref{R3}) (\secref*{sec:4.5}). Finally, I discuss the behavior of flexible items with respect to their semantics (\ref{R4}) (\secref*{sec:4.6}).}

\section{Introduction}
\label{sec:4.1}

This chapter presents the empirical findings from this study, answering the research questions posed in \chref{ch:introduction}. I employ a useful visualization for displaying information about lexical flexibility called a \dfn{ternary plot} or \dfn{triangle plot}; I explain how these ternary plots are to be read in \secref*{sec:4.2}. \secref*{sec:4.3} focuses on answering \ref{R1}, \enquote{How flexible are lexical items in English and Nuuchahnulth?}, both individually and in comparison. \secref*{sec:4.3} is dedicated to answering \ref{R2}, \enquote{Is there a correlation between degree of lexical flexibility and size of the corpus?}, and \secref*{sec:4.4} answers \ref{R3}, \enquote{Is there a correlation between degree of lexical flexibility for a lexical item and frequency (or corpus dispersion)?}. In the final section (\secref{sec:4.6}), I look at the semantic behavior of more and less flexible items (question \ref{R4}).

\section{Interpreting the results}
\label{sec:4.2}

In \secref*{sec:3.4.1} I describe the procedure for quantifying the lexical flexibility of an item in a corpus using a Shannon diversity index ($H$). While the resulting values nicely align with our intuitions about when a lexical item is more or less flexible, some information is lost in the process. Reducing the lexical flexibility of an item to a single number obscures the fact that items can be equally flexible in different ways. Consider the fictional frequency data for two different stems in \tabref{tab:equal-flexibility-stems}. Stem A displays a great deal of reference-predicate flexibility, but very few instances of use as a modifier. Stem B, in contrast, displays extensive reference-modifier flexibility, but few instances of use as a predicate. However, the overall flexibility ratings of the two stems are the same.

\TODO[inline]{add table}

One way to address this reduction in fidelity is to report frequencies and corpus dispersions for each function in addition to the overall flexibility rating for each stem. I provide this information in \appref{app:100-item-samples} alongside each item's flexibility rating. However, it is also possible to visualize the relative usage of an item for each discourse function in an intuitive way by using a \dfn{ternary plot} (also called a \dfn{triangle plot}, \dfn{simplex plot}, \dfn{Gibbs triangle}, or \dfn{de Finetti diagram}). A ternary plot depicts the ratios of three variables as points within an equilateral triangle. Each corner of the triangle corresponds to one of the three possible categories (in this case, reference, predication, or modification). The closer a data point is to a particular corner, the larger the ratio of that category is. To illustrate with an example: \figref{fig:difficult} is a ternary plot for the functions of the word \txn{difficult} in English, along with the underlying frequency data and resulting flexibility rating.

\TODO[inline]{add ternary plot for `difficult', along with underlying data}

\noindent Because the word \txn{difficult} only appears as a modifier in the corpus, it has a flexibility rating of $0$. In the ternary plot, this is evident from the fact that the plot point for \txn{difficult} sits in the modification corner of the triangle.

Compare the plot for \txn{difficult} in \figref{fig:Eng-difficult} to that of \txn{away} in \figref{fig:Eng-away}.

\TODO[inline]{add ternary plot for `away', along with underlying data}

\noindent The stem \txn{anything} also has a flexibility rating of $0$ because all of its tokens are used for reference. Even though its flexibility rating is the same as that of \txn{difficult}, it is plotted in a different corner of the ternary plot (reference).

\figref{fig:Eng-childhood} shows a case where a stem (\txn{childhood}) is flexible between reference and modification, but not predication.

\TODO[inline]{add ternary plot for `childhood', along with underlying data}

\noindent A perfectly flexible item which has equal use as a referent, predicate, and modifier, would sit exactly in the center of the triangle. The Nuuchahnulth stem \txn{ʔu·q} \tln{good} is one such case, shown in \figref{fig:Nuu-good}. The closer a point is towards the center of the triangle, the more flexible it is.

\TODO[inline]{add ternary plot for Nuuchahnulth 'good', along with underlying data}

Finally, remember from \chref{ch:methods} that corpus dispersion is a better measure of frequency of exposure than just raw frequency. Thus in addition to relative frequency data, I also report corpus dispersions for the discourse functions of each lexical item in \appref{app:100-item-samples}. Note that the corpus dispersions are calculated separately for each discourse function (in addition to the overall corpus dispersion of the lexical item). A particular lexical item might be used for one function evenly throughout the corpus, and thus have a low $DP$ for that function, but might only be used for another function in one or two texts, thus giving that function a high $DP$. The ratios of these corpus dispersions for each function can be plotted on a ternary plot just like frequency. Plots based on corpus dispersions are sometimes notably different from plots based on frequencies, as \figref{sec:plot-frequency-vs-dispersion} illustrates. In most cases however the plots are identical or near-identical. As such, for the remainder of this study I will use ternary plots based on corpus dispersion rather than frequency, noting where the two diverge only when relevant.

\TODO[inline]{compare ternary plots for frequency and dispersion where they differ significantly}

\section{R1: Degree of lexical flexibility}
\label{sec:4.3}

In this section I examine the degree of lexical flexibility for words in English and Nuuchahnulth from several angles, both independently and in comparison, using the lexical flexibility ratings calculated with the methods in \secref*{sec:3.4.1}. The result of these calculations for the 100-item samples are shown in \appref{app:100-item-samples}, and a selection of the results appear in \tabref{tab:English-sample} for English and \tabref{tab:Nuuchahnulth-sample} for Nuuchahnulth.

\TODO[inline]{add English-sample table, based on the one in your colloquium talk}

\TODO[inline]{add Nuuchahnulth-sample table, based on the one in your colloquium talk}

\figref{fig:histogram-100-items} visualizes these flexibility ratings for the 100-item samples from English (lefthand side) and Nuuchahnulth (righthand side). The top portion of each figure is a histogram showing the number of lexical items at different flexibility ratings, providing an approximate representation of the distribution of the flexibility ratings. Beneath the histograms are boxplots showing the median flexibility rating for each language.

\TODO[inline]{add the histograms (with boxplots) for the 100-item samples}

\noindent \figref{fig:histogram-small-corpus} shows the same visualizations for the small corpus samples.

\TODO[inline]{add the histograms (with boxplots) for the small corpus samples}

\noindent Tables \ref{tab:100-item-stats} and \ref{tab:small-corpus stats} summarize basic descriptive statistics for each of the samples.

\TODO[inline]{tables showing min, max, mean, median for each sample}

One immediately obvious observation to be made from these flexibility ratings is that individual lexical items may vary widely in their flexibility, both within and across languages. While this finding is entirely unsurprising, the results very well could have been otherwise. The way Nuuchahnulth is often described, one might expect all the lexical items in the language to fall within a more limited range of high-flexibility values. This is clearly not the case. Flexibility ratings for Nuuchahnulth range from the theoretical minimum of $0$ to a maximum of $.920$ or $0.000$ depending on the sample.\TODO{update these values} However, \TODO{XX (XX\%)} of the small corpus Nuuchahnulth sample have a flexibility rating of $0$ (\TODO{XX or XX\% for the 100-item sample}), potentially challenging the claim that all Nuuchahnulth stems are flexible.

Likewise, those who claim that English parts of speech are well defined must confront the fact that the range of flexibility values for English is nearly the same as for Nuuchahnulth for both samples: $0$ on the lower end and $.919$ or $0.000$ on the upper end.\TODO{update these values} In fact, in both samples there are fewer English stems with a flexibility rating of $0$ than there are Nuuchahnulth stems with a flexibility rating of $0$.\TODO{verify that this is true} In this respect, then, English could be viewed as more flexible than Nuuchahnulth. Of course, it may be that this difference is due to the large difference in corpus sizes between English and Nuuchahnulth, an issue which is explored in \secref*{sec:4.4}.

Thus the answer to the question, \enquote{Are some lexical items more flexible than others?} is unsurprisingly \enquote{yes}. To pose a related question, \enquote{Can it be shown empirically and quantitatively that some lexical items are more flexible than others, as many linguists have claimed?}. The answer is again, \enquote{yes}. If we want to evaluate the claim that some languages are more or less flexible than others, it must be possible to quantify that flexibility at the level of the individual lexical item and compare them in a meaningful way. The data and methods in this thesis show that this is indeed possible, and that we can provide clear empirical answers to these kinds of questions. The flexibility of individual lexical items varies widely both within and between languages.

A slightly different question than whether individual stems vary in their flexibility is whether they exhibit flexibility to any substantive degree in the first place. Or, to invert the question, is lexical flexibility a marginal / rare phenomenon which has merely been given disproportionate attention in the literature? A quick look at the flexibility data above shows that this is not the case. When lexical items in English and Nuuchahnulth exhibit flexibility, it is typically not to a marginal degree. A stem is more likely to have a flexibility of, say, $.2$ than something like $.002$. According to a one-sided, one-sample sign test, the median flexibility in all samples differs highly significantly from zero (see the summary table in \tabref{tab:English-vs-Nuuchahnulth-median}).

% (English: $V = 4371$, $p < .001$; Nuuchahnulth: $V = 20503$, $p < .001$).
\TODO[inline]{summarize the statistical results comparing the median flexibility in a table}

This result may seem obvious, but it must be recognized that the result could have been different. Flexibility for English stems could have been so marginal as to not significantly deviate from zero. This would have supported an analysis of lexical flexibility in English as mere occasional language play, something exceptional rather than rampant or productive. The data show otherwise: lexical flexibility is a prevalent feature of both English and Nuuchahnulth, though to different degrees.

Another obvious question to ask of these data is whether English and Nuuchahnulth differ in their overall flexibility. The answer to this is clearly \enquote{yes}, as the mean and median flexibility ratings in Tables \ref{tab:100-item-stats} and \ref{tab:small-corpus-stats} show. The mean flexibility ratings are higher for Nuuchahnulth than English in both samples. \TODO{verify this} But to reduce the entire lexicon of a language to a single average obscures important details. While it is certainly interesting that Nuuchahnulth stems are on net more flexible than English stems, the way in which the two languages exhibit flexibility is arguably the more interesting finding from this study.

How then is lexical flexibility realized in English and Nuuchahnulth? In addition to the histograms in Figures \ref{fig:histogram-100-items} and \ref{fig:histogram-small-corpus}, the ternary plots in Figures \ref{fig:ternary-100-items} and \ref{fig:ternary-small-corpus} illustrate just how flexibility operates in the two languages. In these figures, each lexical item is represented by a single point on the ternary plot.

\TODO[inline]{add ternary plots comparing English vs. Nuuchahnulth for both samples}

Beginning with English, we can see that most lexical items exhibit some flexibility, but to a relatively small degree. After zero-flexibility cases, the next most frequent flexibility rating is in the $0$–$0.05$ range. This is evident from the ternary plots, where lexical items tend to cluster near (but not precisely on) the corners for their most prototypical functions. Interestingly, the small English corpus appears to show \emph{more} flexibility than the 100-item sample. This could be an effect of the specific words chosen, but it could also be the case that it takes a certain number of tokens for the prototypical function of an item to become evident. This possibility is examined further in \secref*{sec:4.4}.

Nuuchahnulth differs from English in several notable ways. First, a much higher proportion of items display no flexibility whatsoever. However, for those items which do exhibit flexibility, the average flexibility rating is generally higher than that of English stems. In both samples, the biggest cluster of items with non-zero flexibility ratings have ratings around $.6$. English items with non-zero flexibility, by comparison, generally have ratings closer to $.2$. Thus for Nuuchahnulth lexical items are either totally inflexible or generally strongly flexible.

This bifurcation of the data is very likely due to the small size of the Nuuchahnulth corpus, as the discussion in \secref*{sec:4.4} suggests. It may be that Nuuchahnulth words are generally highly flexible, but that more tokens are needed to see this trend. Alternatively, it may be that certain Nuuchahnulth stems are strongly associated with a specific discourse function and thus inflexible, while others are generally flexible. This would suggest a probabilistic division of Nuuchahnulth stems into two classes: those that are productively flexible, and those that are not.

This second possibility would challenge existing analyses of Nuuchahnulth. The existence of a productively flexible class of stems would be counterevidence to the many claims that Nuuchahnulth word classes can in fact be clearly defined using selectional criteria such as ability to take possession or the definite suffix \addcite{citations}. Similarly, \textcite[???]{Nakayama2001} characterizes word classes in Nuuchahnulth as \enquote{strong statistical tendencies} \TODO{get exact phrasing and page for this}. For many Nuuchahnulth stems, however, there is no clear prototypical use. Many stems are used roughly as equally for predication as they are for reference, making it difficult to assess which use is more basic or unmarked.

As the ternary plots for Nuuchahnulth in \figref{fig:ternary-100-items} and \figref{fig:ternary-small-corpus} make clear, the distribution of lexical items across functions in Nuuchahnulth differs strongly from that of English. For starters, there is very little clustering around prototypical functions in the corners, in direct contrast to English. Secondly, Nuuchahnulth shows very little flexibility in the modification direction, but rampant flexibility along the reference-predication axis. For the small corpus sample in particular, there is a smooth cline of values between reference and predication. Nuuchahnulth stems sit anywhere on a continuum from prototypical referents to prototypical predicates, but none show prototypical modifier behavior.

These findings nicely reflect the intuitions of many researchers about these two languages. English is mostly rigid, but most words exhibit a marginal degree of flexibility. English words are \emph{primarily} associated with one discourse function, but not exclusively so. Nuuchahnulth, by contrast, shows a very high degree of reference-predicate flexibility. However, Nuuchahnulth stems are not frequently used for modification. This is in line with the analysis of most researchers regarding lexical categories in Nuuchahnulth. \textcite[50]{Nakayama2001}, for example, says that the categories noun and verb must be recognized for Nuuchahnulth, but that there is not sufficient evidence to justify an adjective category, even as a statistical tendency. He instead treats \enquote{adjectivals} as a subclass of verbs; the central location of the points in the Nuuchahnulth plot in \figref{fig:ternary-small-corpus}, however, suggest that Nuuchahnulth modifiers are as nounlike as they are verblike. The low frequency with which stems are used for modification also mirrors the results from \posscitet[88--89]{Croft1991} four-language survey of the textual frequency of different lexical classes. He also finds that \textquote[{\cite[88--89]{Croft1991}}]{the overall frequency of roots denoting properties and occurrences of modifiers is extremely low compared to the frequencies of object and action roots and of referring expressions and predications}.

\section{R2: Lexical flexibility and corpus size}
\label{sec:4.4}

It seems intuitively plausible that the more tokens of a word one encounters, the more likely one is to find flexible uses of a word. With a large enough corpus, all items would exhibit flexibility. This has been claimed by \textcite[77]{MoselHovdhaugen1992}. It may be the case that larger corpora are statistically more flexible than smaller corpora. However, to my knowledge this claim has never been tested empirically. In this section I examine the results of comparing the number of tokens encountered for a stem to its cumulative flexibility rating, the question being, \enquote{Does the cumulative flexibility for the lexical item increase as one encounters more tokens?}.

It is important to understand here that number of tokens encountered is not the same thing as frequency. A very low-frequency word could nonetheless have a high number of tokens in a given corpus if that corpus is very large. Likewise, a high-frequency word could have a low number of tokens in a small corpus.

Only stems with a frequency of at least 4 were included (see \secref{sec:3.4.1} for the motivation behind this restriction). For English, I used the 100-item sample, and for Nuuchahnulth I used the entire corpus. Using a script and going sequentially through the corpora, each time I encountered a new token of a lexical item, I recalculated its flexibility and recorded that value and token frequency at that point in the corpus.

\figref{fig:cumulative-flexibility-words-English} shows the result of these calculations for the the ten most frequent words in the English corpus, and \figref{fig:cumulative-flexibility-words-Nuuchahnulth} shows the same for Nuuchahnulth. The number of tokens encountered is shown on the x-axis, and the cumulative flexibility is shown on the y-axis. I show only the most frequent words here merely because they provide the clearest visual representation of the data; more comprehensive (but more difficult to read) plots for each language are given in \figref{fig:cumulative-flexibility-all-English} and \figref{fig:cumulative-flexibility-all-Nuuchahnulth}. For ease of visualization, a version of the English data with $\log_{10}$ frequency on the x-axis is also given in \figref{fig:cumulative-flexibility-English-log}.

\TODO[inline]{all five plots of cumulative flexibility}

The first thing to notice from both plots of high-frequency words is that it takes a certain number of tokens for the flexibility of a word to become evident and stable. For English, the trend lines are generally no longer stochastic after $\sim$1,000 tokens encountered (this can be more easily seen in \figref{fig:cumulative-flexibility-English-log}). It we take 1,000 tokens as a reliability threshhold for determining the flexibility of a lexical item, then no Nuuchahnulth item appears with sufficient frequency in the corpus to be certain of its flexibility. That said, the flexibility of the ten words in the Nuuchahnulth sample appears to be relatively stable after 50–75 tokens. There are some words in the English sample which achieve a relatively stable flexibility rating as early as 100 tokens as well. One way to interpret these data is that, since some stems appear in a wider range of discourse contexts than others, one requires a higher number of tokens before their overall flexibility becomes evident; in contrast, the flexibility of stems that appear in a relatively small range of discourse contexts should become clear right away.

The second observation to make regarding these data on cumulative frequency is that, once the trend line for cumulative flexibility becomes smooth, it stays flat. This is in direct contradiction to the hypothesis that words will seem more flexible as one encounters more of their tokens. If this were true, we would expect to see a continual and gradual increase in flexibility for many of the stems in the dataset, and this is not the case.

On the other hand, by the time one encounters 5,000 tokens of a word in English, there are no stems with a flexibility of zero. English flexibility ratings cluster in the lower range ($\sim0.3$), but when sufficient tokens are encountered, there do not seem to be any truly inflexible words. Therefore it does seem to be true (for English at least) that words will eventually display \emph{some} flexibility as the size of the corpus increases, but not that the overall flexibility of the word will increase.

We can also look at the data for each language in aggregate. \figref{fig:cumulative-mean-flexibility-English} shows the cumulative mean flexibility for English per token encountered. Each time a new token of a lexical item was encountered, I calculated the current flexibility ratings of each lexical item encountered up to that point, and calculated their average. The resulting plot shows number of tokens encountered on the x-axis and mean flexibility for the entire corpus up to that point on the y-axis. \figref{fig:cumulative-mean-flexibility-Nuuchahnulth} shows parallel data for Nuuchahnulth. Both graphs clearly show that the average flexibility of the corpus does not increase as the corpus grows larger. Instead it remains flat after a sufficent number of tokens are encountered.

\TODO[inline]{insert plots for English and Nuuchahnulth}

To summarize, once enough tokens of a word are encountered as to give a reliable flexibility rating, that flexibility rating does not increase as the number of tokens encountered continues to grow. Lexical items appear to have (synchronically) fixed degrees of flexibility, that vary from word to word. Logically, aggregating this data at the language level produces the same result: languages have (synchronically) fixed degrees of flexibility, that vary from language to language.

\section{R3: Lexical flexibility and frequency / dispersion}
\label{sec:4.5}

In this section I examine the interactions between lexical flexibility, token frequency, and corpus dispersion for individual lexical items. Given that many linguistic phenomena correlate with frequency / corpus dispersion, it is reasonable to investigate whether lexical flexibility displays such correlations as well. Are high frequency or evenly dispersed words more flexible than low frequency or unevenly dispersed words? This is an interesting question in part because if such a correlation were found the direction of causation could go in either direction. It may be that stems are more frequent precisely because they are more flexible—there is a wider range of discourse contexts that they can occur in. On the other hand, it could be that high frequency words are more cognitively accessible and therefore more prone to novel uses in discourse. Or, in contrast, a higher frequency could also result in a greater degree of entrenchment, so that high frequency words are less likely to be flexible.

To investigate the possible interactions among flexibility, frequency, and dispersion I deploy a Generalized Additive Model (GAM) in order to account for the possibility of interactions not just between flexibility and frequency / dispersion, but for interactions between frequency and dispersion as well. For example, it may be the case that there are correlations between flexibility and dispersion, but only for high frequency words. A Generalized Additive Model allows for the exploration of multiple interactions in this way.

Frequency is represented in this model as $\log_2$ of the relative frequency of the stem. Since relative frequency and corpus dispersion utilize different scales, I also use a tensor smooth to examine the combined contribution of frequency and dispersion to flexibility, over and above their individual contributions. I again used the 100-item sample for the English model, and the entire corpus for Nuuchahnulth.

\figref{fig:interaction-heat-English} and \figref{fig:interaction-heat-Nuuchahnulth} show heat maps of the interactions of the three variables for English and Nuuchahnulth, respectively. The x-axis shows $log_2$ of relative frequency, and the y-axis shows corpus dispersion as Deviation of Propotions ($DP$), with more evenly dispersed items to the bottom of the scale and less evenly dispersed items to the top of the scale. Light-colored areas indicate a high degree of flexibility, while dark-colored areas indicate a low degree of flexibility.

\TODO[inline]{interaction heat maps of English and Nuuchahnulth}

\figref{fig:interaction-3D-English} and \figref{fig:interaction-3D-Nuuchahnulth} are 3D representations of the same data, rotated for ease of visualization. $\log_2$ relative frequency is shown on the x-axis (with higher relative frequency to the left), flexibility is shown on the y-axis (with higher flexibility at the top of the scale), and corpus dispersion is shown on the z-axis (with more evenly dispersed values further away).

In English, high frequency, evenly dispersed items appear to have low flexibility ratings, while low frequency, unevenly dispersed items appear to have high flexibility ratings. However, none of the interactions for the English model are significant. The reason for this becomes apparent when looking at the same 3D interaction plot but with maps added at a standard deviation of 2, as in \figref{fig:interaction-SD-English}. There is so much variability in the data for English that no firm conclusions can be drawn.

The model for Nuuchahnulth, on the other hand, shows a couple of significant interactions. First, higher-frequency words show a greater degree of flexibility than lower-frequency words. This correlation is highly significant ($F = 37.582$, $p < .001$). Corpus dispersion, however, shows only a marginally significant correlation with flexibility ($F = 2.384$, $p < .1$), so no firm conclusion can be drawn regarding the direct relationship between corpus dispersion and flexibility. However, the combined interaction of corpus dispersion and relative frequency does correlate with flexibility, above and beyond the contribution provided by relative frequency alone ($F = 2.979$, $p < .05$). Low frequency, unevenly dispersed items have low flexibility ratings, while high frequency, evenly dispersed items have higher flexibility ratings.

These results for Nuuchahnulth should not be accepted unquestioningly as representative of the overall state of affairs for the language, however. Remember from \secref{sec:4.4} that a certain minimum threshhold of number of tokens is required in order to be certain of that word's flexibility. Given the relatively low frequencies of items in the Nuuchahnulth corpus, the flexibility ratings of many stems are likely inaccurate. In particular, the high incidence of items with zero-flexibility ratings is almost undoubtedly due to the small number of tokens encountered for those items. In fact, the 3D interaction plot for Nuuchahnulth in \figref{fig:interaction-SD-Nuuchahnulth} shows that as stems increase in frequency, the standard deviation for their flexibility ratings grow dramatically, resembling that of English. Therefore it is likely that the strong correlations that currently appear for the Nuuchahnulth data would disappear with a larger corpus.

In summary, the data on lexical flexibility and frequency / corpus dispersion are not clear enough to draw any firm conclusions regarding their interactions.

\section{R4: The semantics of lexical flexibility}
\label{sec:4.6}

\subsection{English}
\label{sec:4.6.1}

In this section I take a brief look at the semantics of lexical flexibility, in particular whether there are semantic commonalities to high or low flexibility words. I restrict myself here to aspects of the semantics of lexical items which can be discerned from the existing data and annotations used to answer other research questions for this project. Little additional data coding or annotation was done for the specific purpose of answering this research question. This section is therefore primarily exploratory, with the aim of discovering just what conclusions can be drawn about the semantics of lexical flexibility using merely the simple annotations of discourse functions prepared for this study. I begin with English before moving on to Nuuchahnulth.

The first observation about the semantics of lexical flexibility is purely anecdotal but nonetheless merits comment: the second most flexible word in the 100-item sample of English is \txn{back}, used 272 times for reference, 54 times for predication, and 143 times for modification, with a flexibility rating of $.844$. Going into this study, I postulated that body part terms would display a high degree of flexibility. The motivation for this hypothesis is that body part terms commonly undergo metaphorical extension into other domains, and in general make themselves available for all sorts of extensions of meaning. This is undoubtedly due to the fact that our experience of the world is necessarily mediated through our own bodies \addcite{Metaphors we live by, if it's relevant}. The methods I chose to adopt in thesis prevented any detailed exploration of this hypothesis, but it is notable that the only body part term in either of the 100-item samples is one of the single most flexible items in this study, anecdotally supporting the hypothesis that body part terms are in general highly flexible.

Several semantic classes stand out as being among the lowest flexibility words in the 100-item English sample: indefinites; adult human animates (less so for non-adult humans, as the data for \txn{child} shows); property words denoting size, age, or physical properties; and words of cognition and perception all have flexibility ratings lower than $0.100$, and most are within the 25 lowest flexibility words in the sample. Indefinites in particular rank lowest among the ratings (all with a flexibility rating of $0$). \tabref{tab:English-low-flexibility} shows the statistical data for each of the indefinite terms in the sample, and their rank in terms of flexibility (out of the 100 items sampled).

\TODO[inline]{finalize this table; the data already here are correct; use headings to separate semantic classes}
% indefinites
% stem       & rel freq & dispersion & flexibility & ref   & pred & mod\\
% anything   & 0.755    & 0.449      & 0.000       & 2081  & 0    & 0  \\
% everything & 0.606    & 0.518      & 0.000       & 1960  & 0    & 0  \\
% something  & 1.665    & 0.341      & 0.000       & 5092  & 0    & 0  \\
% thing      & 3.277    & 0.267      & 0.000       & 10649 & 0    & 0  \\

It is easy to see why some of these classes of words would have such low flexibility ratings: each is highly prototypical of one particular discourse function. Adult human animates are the most canonical type of noun crosslinguistically \addcite{sources for this}, while \txn{thing} and its variants are the most generic terms there are for referents. Words denoting size, age, or physical properties are among the core semantic classes for modifiers crosslinguistically \parencite{Dixon1977}. It is entirely unsurprising that these categories of words, then, would always be construed by speakers in the discourse functions that they are the most prototypical exemplars of. At the same time, these data show that such classification is not absolute. Even words that are strongly prototypical of a given discourse function are still occasionally used for other functions.

It is less clear why words of cognition or perception have low flexibility ratings, except that in each case there is a corresponding overtly-derived referential term which potentially blocks the use of the word as a referent: \txn{enjoy} is blocked by \txn{enjoyment}; \txn{believe} is blocked by \txn{belief}; \txn{hate} is blocked by \txn{hatred}; \txn{know} is blocked by \txn{knowledge}; and so on. These referential counterparts do not necessarily \emph{prevent} the use of these stems as referents (e.g. \textit{to be in the \textbf{know}}), but they are likely a significant contributing factor. Regardless, it is unclear why these words have morphologically derived referential counterparts but most of the highest-flexibility words such as \txn{paint}, \txn{work}, \txn{order}, and \txn{transfer} do not.

\subsection{Nuuchahnulth}
\label{sec:4.6.2}

A first observation about the semantics of lexical flexibility in Nuuchahnulth is that roots and stems display nearly identical behavior in regard to their flexibility. That is, a given stem will have roughly the same flexibility rating as the root it is derived from. \figref{fig:Nuuchahnulth-roots-vs-stems} compares the flexibility ratings of roots (the lefthand plot) and stems (the righthand plot), showing how similar the two distributions are. \TODO[options]{X statistical test shows that the two distributions do not differ significantly from each other; stats}. These data suggest that the primary determinant of the discourse function of words in Nuuchahnulth is the semantics of the root. The lexical affixes for which Wakashan languages are so famous appear to add semantic detail without generally altering the discourse function of the word.

\TODO[inline]{plots of roots vs. stems in Nuuchahnulth}

Turning now to the semantic classes that align with high and low flexibility, one class in particular stands out as being especially flexible: property-denoting words, and especially numerals and quantifiers. 12 of the top 20 most flexible stems in Nuuchahnulth are property words. With few exceptions, property-denoting words in Nuuchahnulth have high flexibility ratings, above $0.5$. All of the core deictic stems in Nuuchahnulth also feature in the top 25 most flexible words. The statistical data for both these classes of stems, along with their rank in terms of flexibility, are listed in \tabref{tab:Nuuchahnulth-high-flexibility}.

\TODO[inline]{table of high-flexibility Nuuchahnulth words, divided into headings with property words and deictic stems}

What accounts for the consistently high flexibility rating for numerals and quantifiers? First, Nuuchahnulth does not have any dedicated morphosyntactic constructions that express the function of modification, except that modifiers precede their head syntactically and take no inflectional affixes when they do so. Yet while this syntactic construction is available to speakers, its use is fairly uncommon. Instead, speakers avail themselves of two strategies for communicating property concepts: a) lexical affixation, and b) construing property concepts as either referents or predicates.

Wakashan languages and the languages of the Pacific Northwest in general are well known for their use of \dfn{lexical affixes}, affixes with concrete lexical meanings rather than grammatical / functional ones \parencite{Mithun1997}. \addcite{Mithun. 1997. Lexical affixes and morphological typology, in Essays on language function and language type} Nuuchahnulth's very large set of lexical suffixes allows speakers to use property-denoting roots in complex stems, where the root denotes the property being attributed, and the lexical suffix denotes the referent being modified. Example \exref{ex:4.1} shows several such uses of property-denoting roots.

\TODO[inline]{add examples for good-things, all-things, and two-canoe; maybe replace / supplement all-things with another more canonical modifier like big}

\noindent The use of property-denoting roots with lexical affixes is by far the most common strategy for attributing properties to referents in Nuuchahnulth. The choice of strategy between a bare modifier and the use of lexical affixes is intimately connected with information flow in discourse. Already-activated discourse referents are typically expressed through lexical affixes, whereas newly-introduced discourse referents are presented as independent noun phrases \addcite{Mithun - The evolution of noun incorporation}. \textcite[144]{Nakayama2001} also shows that referentiality is a key deciding factor between the two constructions.

The other manner by which speakers express property concepts is with either referring or predicating constructions. The fact that speakers use \emph{either} referring or predicating constructions (as opposed to just referring constructions or just predicating constructions) likely has to do with the dual function of property concepts identified by \textcite{Thompson1989}. In a corpus analysis of English and Mandarin, \citeauthor{Thompson1989} finds that property words have primarily two functions in discourse: to introduce new discourse-manipulable referents, and to predicate attributes of an already-known referent. In English these two functions are realized via attributive adjectives and predicative adjectives respectively. Nuuchahnulth appears to follow a similar pattern: when a property word is used to introduce a new referent into the discourse, it typically appears as an independent word modifying a nominal head. As with the English data from \posscitet{Thompson1989} study, the head is typically a semantically empty or generic referent whose primary function is to serve as a carrier of the property word. \exref{ex:4.2} shows a few examples of this phenomenon in Nuuchahnulth.

\TODO[inline]{examples of adjectives as separate words, from the section on modification in Toshi's grammar}

\noindent By contrast, when a property is being predicated of an already-established discourse referent, the lexical affix strategy is used instead.

Focusing on just the numerals, we find a potential trend: for the most part, the flexibility of numerals decreases as their numeric values increase. The cardinal numbers and their flexibility ratings are shown in numeric order in \tabref{tab:Nuuchahnulth-numerals}. Given the low frequencies involved for these stems, it would be unwise to make strong claims about the potential trend in this table. However, the data are at the very least suggestive of the idea that cardinal numerals adhere to an implicational hierarchy, wherein the flexibility of a numeral decreases as its numeric value increases. If true, this trend would be in line with other well-documented implicational universals for cardinal numerals \parencites{DehaeneMiller1992}[141]{Croft2003} \addcite{DehaeneMiller1992: https://www.sciencedirect.com/science/article/abs/pii/001002779290030L}.

\TODO[inline]{add table of Nuuchahnulth numerals}

Much like English, animate human beings are generally among the lower-flexibility stems in Nuuchahnulth (below $0.5$), although their ratings are still higher than those for English. A selection of animate human stems and their flexibility ratings are shown in \tabref{tab:Nuuchahnulth-low-flexibility}. Of particular note is the fact that the word \txn{quːʔas} \tln{person, man} has one of the lowest flexibility ratings in the Nuuchahnulth corpus (excluding those with ratings of zero). Yet this was the very stem that \textcite{Swadesh1939b} used to demonstrate Nuuchahnulth's extreme flexibility! This is an excellent example of why we need more empirical coverage for the study of lexical flexibility—this claim about the flexibility of the word \tln{person, man} in Nuuchahnulth has been repeated verbatim for nearly a century, but entirely unbacked by the kind of comprehensive data needed to support it. The marginal flexibility of \tln{person, man} and other human animates does however illustrate that even highly prototypical referents exhibit degrees of flexibility.

% stem     & gloss  & rel freq & dispersion & flexibility & ref   & pred & mod\\
% ɬuːcma   & wife   & 2.988    & 0.559      & 0.477       & 18    & 5    & 0  \\
% ḥaw̓iɬ    & chief  & 4.184    & 0.549      & 0.417       & 26    & 6    & 0  \\
% ḥaːkʷa·ƛ & girl   & 2.869    & 0.868      & 0.158       & 23    & 1    & 0  \\
% quːʔas   & person & 9.682    & 0.341      & 0.106       & 78    & 2    & 0  \\
