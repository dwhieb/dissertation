\chapter{Results}
\label{ch:results}

\blockquote{This chapter reports the results of applying the procedures described in \hyperref[ch:methods]{Chapter 3: Data \& Methods}. I begin by demonstrating for the reader how to interpret the ternary plots used to visually represent the degree of lexical flexibility for individual items (\secref*{sec:4.2}). Next I take an aggregate look at the behavior of English and Nuuchahnulth respectively in terms of their lexical flexibility (\ref{R1}) (Sections \hyperref[sec:4.3]{4.3} \& \hyperref[sec:4.4]{4.4}). In \secref*{sec:4.5}, I look at general trends for both languages. First I present the results of the investigation of the relationship between the degree of lexical flexibility and frequency / dispersion (\ref{R2}) (\secref*{sec:4.5.1}). Finally, I discuss the behavior of flexible items with respect to their semantics (\ref{R3}) (\secref*{sec:4.5.2}).}

\section{Introduction}
\label{sec:4.1}

This chapter presents the empirical findings from this study, answering the three main research questions posed in \chref{ch:introduction}. I employ a useful visualization for displaying information about lexical flexibility called a \dfn{ternary plot} or \dfn{triangle plot}; I explain how these ternary plots are to be read in \secref*{sec:4.2}. Sections \hyperref[sec:4.3]{4.3} and \hyperref[sec:4.4]{4.4} focus on answering \ref{R1}, \enquote{How flexible are lexical items in English and Nuuchahnulth?}, both individually and in comparison. \secref*{sec:4.5} is dedicated to answering questions \ref{R2} and \ref{R3}: \enquote{Is there a correlation between degree of lexical flexibility for a lexical item and frequency (or corpus dispersion)?} (\secref*{sec:4.5.1}), and \enquote{How do the semantic properties of lexical items pattern with respect to their flexibility?} (\secref*{sec:4.5.2}).

\section{Interpreting the results}
\label{sec:4.2}

\section{English}
\label{sec:4.3}

\section{Nuuchahnulth}
\label{sec:4.4}

\section{General results}
\label{sec:4.5}

\subsection{Lexical flexibility and frequency / dispersion}
\label{sec:4.5.1}

\subsection{The semantics of lexical flexibility}
\label{sec:4.5.2}
