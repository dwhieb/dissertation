\chapter{Results}
\label{ch:results}

\blockquote{This chapter reports the results of applying the procedures described in \hyperref[ch:methods]{Chapter 3: Data \& Methods}. I begin by demonstrating for the reader how to interpret the ternary plots used to visually represent the degree of lexical flexibility for individual items (\secref*{sec:4.2}). Next I look at the flexibility of lexical items in English and Nuuchahnulth, both independently and in comparison (\secref*{sec:4.3}). I then I investigate whether lexical flexibility depends on corpus size (\ref{R2}) (\secref*{sec:4.4}). Next, examine the relationship between the degree of lexical flexibility and frequency / dispersion (\ref{R3}) (\secref*{sec:4.5}). Finally, I discuss the behavior of flexible items with respect to their semantics (\ref{R3}) (\secref*{sec:4.6}).}

\section{Introduction}
\label{sec:4.1}

This chapter presents the empirical findings from this study, answering the research questions posed in \chref{ch:introduction}. I employ a useful visualization for displaying information about lexical flexibility called a \dfn{ternary plot} or \dfn{triangle plot}; I explain how these ternary plots are to be read in \secref*{sec:4.2}. \secref*{sec:4.3} focuses on answering \ref{R1}, \enquote{How flexible are lexical items in English and Nuuchahnulth?}, both individually and in comparison. \secref*{sec:4.3} is dedicated to answering \ref{R2}, \enquote{Is there a correlation between degree of lexical flexibility and size of the corpus?}, and \secref*{sec:4.4} answers \ref{R2}, \enquote{Is there a correlation between degree of lexical flexibility for a lexical item and frequency (or corpus dispersion)?}. In the final section (\secref{sec:4.6}), I look at the semantic behavior of more and less flexible items.

\section{Interpreting the results}
\label{sec:4.2}

In \secref*{sec:3.4.1} I describe the procedure for quantifying the lexical flexibility of an item in a corpus using a Shannon diversity index ($H$). While the resulting values nicely align with our intuitions about when a lexical item is more or less flexible, some information is lost in the process. Reducing the lexical flexibility of an item to a single number obscures the fact that items can be equally flexible in different ways. Consider the fictional frequency data for two different stems in \tabref{tab:equal-flexibility-stems}. Stem A displays a great deal of reference-predicate flexibility, but very few instances of use as a modifier. Stem B, in contrast, displays extensive reference-modifier flexibility, but few instances of use as a predicate. However, the overall flexibility ratings of the two stems are the same.

\TODO[inline]{add table}

One way to address this reduction in fidelity is to report frequencies and corpus dispersions for each function in addition to the overall flexibility rating for each stem. I provide this information in \appref{app:100-item-samples} alongside each item's flexibility rating. However, it is also possible to visualize the relative usage of an item for each discourse function in an intuitive way by using a \dfn{ternary plot} (also called a \dfn{triangle plot}, \dfn{simplex plot}, \dfn{Gibbs triangle}, or \dfn{de Finetti diagram}). A ternary plot depicts the ratios of three variables as points within an equilateral triangle. Each corner of the triangle corresponds to one of the three possible categories (in this case, reference, predication, or modification). The closer a data point is to a particular corner, the larger the ratio of that category is. To illustrate with an example: \figref{fig:difficult} is a ternary plot for the functions of the word \txn{difficult} in English, along with the underlying frequency data and resulting flexibility rating.

\TODO[inline]{add ternary plot for `difficult', along with underlying data}

\noindent Because the word \txn{difficult} only appears as a modifier in the corpus, it has a flexibility rating of $0$. In the ternary plot, this is evident from the fact that the plot point for \txn{difficult} sits in the modification corner of the triangle.

Compare the plot for \txn{difficult} in \figref{fig:Eng-difficult} to that of \txn{away} in \figref{fig:Eng-away}.

\TODO[inline]{add ternary plot for `away', along with underlying data}

\noindent The stem \txn{anything} also has a flexibility rating of $0$ because all of its tokens are used for reference. Even though its flexibility rating is the same as that of \txn{difficult}, it is plotted in a different corner of the ternary plot (reference).

\figref{fig:Eng-childhood} shows a case where a stem (\txn{childhood}) is flexible between reference and modification, but not predication.

\TODO[inline]{add ternary plot for `childhood', along with underlying data}

\noindent A perfectly flexible item which has equal use as a referent, predicate, and modifier, would sit exactly in the center of the triangle. The Nuuchahnulth stem \txn{ʔu·q} \tln{good} is one such case, shown in \figref{fig:Nuu-good}. The closer a point is towards the center of the triangle, the more flexible it is.

\TODO[inline]{add ternary plot for Nuuchahnulth 'good', along with underlying data}

Finally, remember from \chref{ch:methods} that corpus dispersion is a better measure of frequency of exposure than just raw frequency. Thus in addition to relative frequency data, I also report corpus dispersions for the discourse functions of each lexical item in \appref{app:100-item-samples}. Note that the corpus dispersions are calculated separately for each discourse function (in addition to the overall corpus dispersion of the lexical item). A particular lexical item might be used for one function evenly throughout the corpus, and thus have a low $DP$ for that function, but might only be used for another function in one or two texts, thus giving that function a high $DP$. The ratios of these corpus dispersions for each function can be plotted on a ternary plot just like frequency. Plots based on corpus dispersions are sometimes notably different from plots based on frequencies, as \figref{sec:plot-frequency-vs-dispersion} illustrates. In most cases however the plots are identical or near-identical. As such, for the remainder of this study I will use ternary plots based on corpus dispersion rather than frequency, noting where the two diverge only when relevant.

\section{R1: Degree of lexical flexibility}
\label{sec:4.3}

\section{R2: Lexical flexibility and corpus size}
\label{sec:4.4}

\section{R3: Lexical flexibility and frequency / dispersion}
\label{sec:4.5}

\section{R4: The semantics of lexical flexibility}
\label{sec:4.6}
