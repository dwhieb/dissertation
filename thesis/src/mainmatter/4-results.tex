\chapter{Results}
\label{ch:results}

\blockquote{This chapter reports the results of applying the procedures described in \hyperref[ch:methods]{Chapter 3: Data \& Methods}. I begin by demonstrating for the reader how to interpret the ternary plots used to visually represent the degree of lexical flexibility for individual items (\secref*{sec:4.2}). Next I look at the flexibility of lexical items in English and Nuuchahnulth, both independently and in comparison (\secref*{sec:4.3}). I then I investigate whether lexical flexibility depends on corpus size (\ref{R2}) (\secref*{sec:4.4}). Next, examine the relationship between the degree of lexical flexibility and frequency / dispersion (\ref{R3}) (\secref*{sec:4.5}). Finally, I discuss the behavior of flexible items with respect to their semantics (\ref{R3}) (\secref*{sec:4.6}).}

\section{Introduction}
\label{sec:4.1}

This chapter presents the empirical findings from this study, answering the research questions posed in \chref{ch:introduction}. I employ a useful visualization for displaying information about lexical flexibility called a \dfn{ternary plot} or \dfn{triangle plot}; I explain how these ternary plots are to be read in \secref*{sec:4.2}. \secref*{sec:4.3} focuses on answering \ref{R1}, \enquote{How flexible are lexical items in English and Nuuchahnulth?}, both individually and in comparison. \secref*{sec:4.3} is dedicated to answering \ref{R2}, \enquote{Is there a correlation between degree of lexical flexibility and size of the corpus?}, and \secref*{sec:4.4} answers \ref{R2}, \enquote{Is there a correlation between degree of lexical flexibility for a lexical item and frequency (or corpus dispersion)?}. In the final section (\secref{sec:4.6}), I look at the semantic behavior of more and less flexible items.

\section{Interpreting the results}
\label{sec:4.2}

\section{R1: Degree of lexical flexibility}
\label{sec:4.3}

\section{R2: Lexical flexibility and corpus size}
\label{sec:4.4}

\section{R3: Lexical flexibility and frequency / dispersion}
\label{sec:4.5}

\section{R4: The semantics of lexical flexibility}
\label{sec:4.6}
