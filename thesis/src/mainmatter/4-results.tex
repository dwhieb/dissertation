\chapter{Results}
\label{ch:results}

\blockquote{This chapter reports the results of applying the procedures described in \hyperref[ch:methods]{Chapter 3: Data \& Methods}. I begin by demonstrating for the reader how to interpret the ternary plots used to visually represent the degree of lexical flexibility for individual items (\secref*{sec:4.2}). Next I look at the flexibility of lexical items in English and Nuuchahnulth, both independently and in comparison (\secref*{sec:4.3}). I then I investigate whether lexical flexibility depends on corpus size (\ref{R2}) (\secref*{sec:4.4}). Next, examine the relationship between the degree of lexical flexibility and frequency / dispersion (\ref{R3}) (\secref*{sec:4.5}). Finally, I discuss the behavior of flexible items with respect to their semantics (\ref{R3}) (\secref*{sec:4.6}).}

\section{Introduction}
\label{sec:4.1}

This chapter presents the empirical findings from this study, answering the research questions posed in \chref{ch:introduction}. I employ a useful visualization for displaying information about lexical flexibility called a \dfn{ternary plot} or \dfn{triangle plot}; I explain how these ternary plots are to be read in \secref*{sec:4.2}. \secref*{sec:4.3} focuses on answering \ref{R1}, \enquote{How flexible are lexical items in English and Nuuchahnulth?}, both individually and in comparison. \secref*{sec:4.3} is dedicated to answering \ref{R2}, \enquote{Is there a correlation between degree of lexical flexibility and size of the corpus?}, and \secref*{sec:4.4} answers \ref{R2}, \enquote{Is there a correlation between degree of lexical flexibility for a lexical item and frequency (or corpus dispersion)?}. In the final section (\secref{sec:4.6}), I look at the semantic behavior of more and less flexible items.

\section{Interpreting the results}
\label{sec:4.2}

In \secref*{sec:3.4.1} I describe the procedure for quantifying the lexical flexibility of an item in a corpus using a Shannon diversity index ($H$). While the resulting values nicely align with our intuitions about when a lexical item is more or less flexible, some information is lost in the process. Reducing the lexical flexibility of an item to a single number obscures the fact that items can be equally flexible in different ways. Consider the fictional frequency data for two different stems in \tabref{tab:equal-flexibility-stems}. Stem A displays a great deal of reference-predicate flexibility, but very few instances of use as a modifier. Stem B, in contrast, displays extensive reference-modifier flexibility, but few instances of use as a predicate. However, the overall flexibility ratings of the two stems are the same.

\TODO[inline]{add table}

One way to address this reduction in fidelity is to report frequencies and corpus dispersions for each function in addition to the overall flexibility rating for each stem. I provide this information in \appref{app:100-item-samples} alongside each item's flexibility rating. However, it is also possible to visualize the relative usage of an item for each discourse function in an intuitive way by using a \dfn{ternary plot} (also called a \dfn{triangle plot}, \dfn{simplex plot}, \dfn{Gibbs triangle}, or \dfn{de Finetti diagram}). A ternary plot depicts the ratios of three variables as points within an equilateral triangle. Each corner of the triangle corresponds to one of the three possible categories (in this case, reference, predication, or modification). The closer a data point is to a particular corner, the larger the ratio of that category is. To illustrate with an example: \figref{fig:difficult} is a ternary plot for the functions of the word \txn{difficult} in English, along with the underlying frequency data and resulting flexibility rating.

\TODO[inline]{add ternary plot for `difficult', along with underlying data}

\noindent Because the word \txn{difficult} only appears as a modifier in the corpus, it has a flexibility rating of $0$. In the ternary plot, this is evident from the fact that the plot point for \txn{difficult} sits in the modification corner of the triangle.

Compare the plot for \txn{difficult} in \figref{fig:Eng-difficult} to that of \txn{away} in \figref{fig:Eng-away}.

\TODO[inline]{add ternary plot for `away', along with underlying data}

\noindent The stem \txn{anything} also has a flexibility rating of $0$ because all of its tokens are used for reference. Even though its flexibility rating is the same as that of \txn{difficult}, it is plotted in a different corner of the ternary plot (reference).

\figref{fig:Eng-childhood} shows a case where a stem (\txn{childhood}) is flexible between reference and modification, but not predication.

\TODO[inline]{add ternary plot for `childhood', along with underlying data}

\noindent A perfectly flexible item which has equal use as a referent, predicate, and modifier, would sit exactly in the center of the triangle. The Nuuchahnulth stem \txn{ʔu·q} \tln{good} is one such case, shown in \figref{fig:Nuu-good}. The closer a point is towards the center of the triangle, the more flexible it is.

\TODO[inline]{add ternary plot for Nuuchahnulth 'good', along with underlying data}

Finally, remember from \chref{ch:methods} that corpus dispersion is a better measure of frequency of exposure than just raw frequency. Thus in addition to relative frequency data, I also report corpus dispersions for the discourse functions of each lexical item in \appref{app:100-item-samples}. Note that the corpus dispersions are calculated separately for each discourse function (in addition to the overall corpus dispersion of the lexical item). A particular lexical item might be used for one function evenly throughout the corpus, and thus have a low $DP$ for that function, but might only be used for another function in one or two texts, thus giving that function a high $DP$. The ratios of these corpus dispersions for each function can be plotted on a ternary plot just like frequency. Plots based on corpus dispersions are sometimes notably different from plots based on frequencies, as \figref{sec:plot-frequency-vs-dispersion} illustrates. In most cases however the plots are identical or near-identical. As such, for the remainder of this study I will use ternary plots based on corpus dispersion rather than frequency, noting where the two diverge only when relevant.

\section{R1: Degree of lexical flexibility}
\label{sec:4.3}

In this section I examine the degree of lexical flexibility for words in English and Nuuchahnulth from several angles, both independently and in comparison, using the lexical flexibility ratings calculated with the methods in \secref*{sec:3.4.1}. The result of these calculations for the 100-item samples are shown in \appref{app:100-item-samples}, and a selection of the results appear in \tabref{tab:English-sample} for English and \tabref{tab:Nuuchahnulth-sample} for Nuuchahnulth.

\TODO[inline]{add English-sample table, based on the one in your colloquium talk}

\TODO[inline]{add Nuuchahnulth-sample table, based on the one in your colloquium talk}

\figref{fig:histogram-100-items} visualizes these flexibility ratings for the 100-item samples from English (lefthand side) and Nuuchahnulth (righthand side). The top portion of each figure is a histogram showing the number of lexical items at different flexibility ratings, providing an approximate representation of the distribution of the flexibility ratings. Beneath the histograms are boxplots showing the median flexibility rating for each language.

\TODO[options]{add the histograms (with boxplots) for the 100-item samples}

One immediately obvious observation to be made from these flexibility ratings is that individual lexical items may vary widely in their flexibility, both within and across languages. While this finding is entirely unsurprising, the results very well could have been otherwise. The way Nuuchahnulth is often described, one might expect all the lexical items in the language to fall within a more limited range of high-flexibility values. This is clearly not the case. Flexibility ratings for Nuuchahnulth range from the theoretical minimum of $0$ to a maximum of $.920$. However, \TODO{XX (XX\%)} of the 100 lexical items in the Nuuchahnulth sample have a flexibility rating of $0$, potentially challenging the claim that all Nuuchahnulth stems are flexible.

Likewise, those who claim that English parts of speech are well defined must confront the fact that the range of flexibility values for English is nearly exactly the same as for Nuuchahnulth: $0$ on the lower end and $.919$ on the upper end. In fact, there are fewer English stems with a flexibility rating of $0$ than there are Nuuchahnulth stems with a flexibility rating of $0$. In terms of the mathematical mode, then, English could be viewed as more flexible than Nuuchahnulth. Of course, it may be that this difference is due to the large difference in corpus sizes between English and Nuuchahnulth, an issue which is explored in \secref*{sec:4.4}.

\TODO[inline]{Report the range and mode of the small corpus samples, and show the histograms / boxplots for them.}

Thus the answer to the question, \enquote{Are some lexical items more flexible than others?} is unsurprisingly \enquote{yes}. Or, to pose the question another way, \enquote{Can it be shown empirically and quantitatively that some lexical items are more flexible than others, as many linguists have claimed?}. The answer is again, \enquote{yes}. If we want to evaluate the claim that some languages are more or less flexible than others, it must be possible to quantify that flexibility at the level of the individual lexical item and compare them in a meaningful way. The data and methods in this thesis show that this is indeed possible, and that we can indeed provide clear empirical answers to these kinds of questions. The flexibility of individual lexical items does indeed vary widely both within and between languages.

A slightly different question than whether individual stems vary in their flexibility is whether they exhibit flexibility to any substantive degree in the first place. Or, to invert the question, is lexical flexibility a marginal / rare phenomenon which has merely been given disproportionate attention in the literature? A quick look at the flexibility data above shows that this is not the case. When lexical items in English and Nuuchahnulth exhibit flexibility, it is typically not to a marginal degree. A stem is more likely to have a flexibility of, say, $.2$ than something like $.002$. According to a one-sided, one-sample sign test, the median flexibility of both English and Nuuchahnulth differs highly significantly from zero (English: $V = 4371$, $p < .0001$; Nuuchahnulth: $V = 20503$, $p < .0001$).

This result may seem obvious, but it must be recognized that it could have been different. Flexibility for English stems could have been so marginal as to not significantly deviate from zero. This would have supported an analysis of lexical flexibility in English as mere occasional language play, something exceptional rather than rampant or productive. This is clearly not the case. These data show that lexical flexibility is in fact prevalent in both English and Nuuchahnulth.

\section{R2: Lexical flexibility and corpus size}
\label{sec:4.4}

\section{R3: Lexical flexibility and frequency / dispersion}
\label{sec:4.5}

\section{R4: The semantics of lexical flexibility}
\label{sec:4.6}
