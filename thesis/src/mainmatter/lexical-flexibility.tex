Within the framework of typological markedness asymmetries, we can provide a structural definition of lexical flexibility as follows:

\begin{description}
  \item[lexical flexibility] The use of a lexical item (root, stem, or inflected word) in more than one discourse function (reference, predication, or modification) with zero coding for that function.
\end{description}

\noindent This definition qualifies as a valid \dfn{comparative concept} in the sense of \textcite{Haspelmath2010a} because it is couched in terms of universal \emph{functions} rather than language-specific \emph{structures} \parencite{Croft2016}. It also has the advantage of being intentionally equivocal with respect to the lexical and cognitive unity of the item, and with respect to the morphological level (root, stem, or inflected word) at which the flexibility is realized. In some cases when a single lexical form appears in more than one discourse function, speakers may have a close cognitive association between the two uses. This is most likely the case for the predicative and referential uses of the word \txn{run} in the phrases \txn{I run every morning} and \txn{I'm going for a run} respectively. In other cases, speakers may have little to no awareness of the diachronic connection between uses of a form. For example, the use of \txn{run} in the sense of \txn{to run a print job} is extremely distant from the prototypical \enquote{fast pedestrian motion} sense in the semantic network for that form \parentext{\cite[74]{Gries2006}; see also \figref{fig:semantic-map-run}}. It is unlikely that these two senses are closely cognitively connected by most speakers, even though they both share a predicating function. The definition of lexical flexibility given here allows for any degree of semantic shift. \citeauthor{Croft2001b} discusses this possibility explicitly: \textquote[{\cite[68]{Croft2001b}}]{It of course a priori possible to construct a typological classification of parts-of-speech systems using only structural coding and allowing any degree of semantic shift.}. Of course, I am not concerned here with constructing a classification of parts of speech—quite the opposite, in fact. The point is to provide a structural definition of lexical flexibility that does \emph{not} rely on any notion of parts of speech. This definition is simply intended to delimit those cases where a language does not provide formal indicators of discourse function, specifically the constructions that use a zero coding strategy (no derivational affixes, no copulas, no overtly marked forms like gerunds, participles, infinitives, etc.).

Allowing for any degree of semantic shift does \emph{not} imply that semantic shift is in any way unimportant for understanding lexical flexibility. On the contrary, semantic shift is a key component of the process of functional expansion (see below). Semantic shift becomes a descriptive \foreign{desideratum}, i.e. something that may be analyzed and described empirically when studying the use of a lexeme across different discourse functions. Moreover, by first providing a structural definition of lexical flexibility without regard to the degree and type of semantic shifts involved, we are in a position to then compare the semantic shifts that occur in cases of lexical flexibility with those that occur in cases of overt derivation. This raises the intriguing question of whether semantic shifts in flexible cases differ in principled ways from overt derivation. \textcite[165]{Mithun2017} shows that for \idx{Central Alaskan Yup'ik} the types of semantic relationships between flexible uses of words mirror those between basic and derived words. This would suggest that functional expansion follows the same principles as overt derivation. However, much more research is needed in this area.

As we have seen, a great abundance of evidence also shows that the meaning of any given combination of form and discourse function is a matter of convention, and often highly idiosyncratic (\secref{sec:2.3.2.4}; \secref{sec:2.3.3.2}). This fact suggests that flexible items are not truly \enquote{flexible} in the sense that speakers can use any lexical item for any discourse function and expect hearers to be able to infer their meaning from context. We know that item-specific gaps in usage exist. Certainly, novel cases of forms being used in new discourse functions do occur, or else it would not be possible for functional shift to happen in the first place. But these cases are necessarily restricted by the cognitive limits on our ability to deal with ambiguity. If it were truly the case that any lexical item could be used in any discourse function at any time, it would barely be possible for hearers to interpret the intended pragmatic effects of each word. Instead, flexibility must rely on a degree of \emph{conventionalization}. Conventionalization in turn implies \emph{time}—conventionalization is a diachronic process. Thus, \emph{lexical flexibility can be understood as a synchronic pattern resulting from the diachronic process of functional expansion}, where functional expansion is defined as follows:

\begin{description}
  \item[functional expansion] A diachronic expansion in the use of a lexical item (root, stem, or inflected word) into a new discourse function (reference, predication, or modification) with zero coding for that function.
\end{description}

\noindent Cases of lexical flexibility therefore arise whenever a new combination of form and discourse function is conventionalized in a community of speakers. This understanding of lexical flexibility is in line with cognitive research on lexicalization and constructional change. Functional expansion occurs because of speakers' need to construe concepts in different ways—as objects, actions, or properties. The semantic shifts that occur during functional expansion are the result of coercion by the new constructional context, and the resultant meaning then becomes conventionalized as the meaning of that particular form in that particular discourse function \parencite[108]{Croft1991}.

This understanding of lexical flexibility also makes a prediction regarding the interaction of flexibility and frequency. Though I know of no studies on the matter, it seems empirically true that functional \emph{expansion} is drastically more common than functional \emph{shift}. That is, it is relatively common for a lexical item to expand from one disourse function into another, but it is rare for a lexical item to undergo a wholesale shift from one function to another. Functional expansion therefore increases the number of contexts that a lexical item appears in. It is thus a reasonable hypothesis that the overall frequency of the lexical item will increase as well, being the sum of the individual senses across different discourse functions. More flexible items may have a higher overall frequency than less flexible items. Of course, this is merely a hypothesis. It may be that functional expansion into new contexts is offset by a concommitant decrease in frequency for the original function, or that this varies depending on the lexical item. These questions are explored quantitatively in \secref*{sec:4.5}.

With this understanding of lexical flexibility and functional expansion in place, we are now in a position to reframe the major research question of this dissertation: How often, diachronically, have words in English and Nuuchahnulth become conventionalized in new discourse functions with zero coding? Answering this question generates new research questions in turn, just a few of which are mentioned below.

\begin{itemize}
  \item Why do some languages have so many more words that underwent functional expansion than other languages? What diachronic processes give rise to rampant lexical flexibility?
  \item Are there semantic commonalities to the lexical items which frequently undergo functional expansion across languages?
  \item Are the semantic shifts in cases of functional expansion more productive or predictable for certain semantic classes of words than others?
\end{itemize}

As to the first point specifically: If lexical flexibility is the result of a diachronic process, it should be possible to enumerate some of the specific pathways which give rise to it. Here I will mention just a few. One pathway is insubordination, whereby subordinate clauses in a language are reanalyzed as main clauses \parencites{Evans2007}{Mithun2008}{EvansWatanabe2016} (see also \secref*{sec:2.3.2.2}). Insubordination frequently results in formal similarities between noun phrases and verb phrases, and this formal ambiguity can abet the functional expansion of lexical items from referential to predicative uses and vice versa.

A second pathway to lexical flexibility is relexicalization (or more precisely, reconventionalization). This is the process that occurred in the case of morphological verbs being reanalyzed as nouns in many North American languages (see \secref{sec:2.3.2.3}) and certain \idx{English} plurals like \txn{brethren} or \txn{arms}. In these cases, the conventionalized meaning associated with the form changed (e.g. from \idx{Cayuga} \tln{it hauls logs} to \tln{horse}), and that meaning is reflected by its use in the new discourse context.

A third pathway is topicalization, exemplified in the \idx{Wakashan} family. \textcite[122, 142]{Jacobsen1979} observes the formal similarity between the Definite Article and the Third Person Singular Indicative markers in Wakashan languages, and argues for a diachronic connection between the two. It is likely that cleft constructions such as \tln{it was the dog that ran} became so common that speakers started to reanalyze the topicalized cleft as a definite noun phrase, \tln{the dog}, thereby creating a formal similarity between referring expressions and predicating expressions.

Each of these pathways results in the functional expansion of lexical items into new discourse contexts with no new overt structural coding. Of course, functional expansion can also occur without any other accompanying grammatical changes. This happens in any instance where speakers simply use stems in new discourse functions. Lexical flexibility is the natural and expected result of the fact that non-prototypical uses of lexical items are \emph{not} always structurally marked—as allowed for by the fact that typological markedness is implicational and not absolute—even if they are marked in other ways. The use of additional structural coding in cases of functional expansion is not obligatory, but merely a statistical tendency. Lexical flexibility occupies the theoretical space where structural coding asymmetries fail to apply.

When viewed in this light, \emph{lexical flexibility is not so much a problem as it is a design feature of language}. The presence of lexical flexibility should be \emph{expected} in every language, not treated as exotic. The cognitive-typological approach outlined in this chapter inverts the lexical flexibility question: the interesting question is not why some languages fail to make distinctions in parts of speech (thereby framing lexical flexibility as a \emph{deficit} in a way similar to colonial researchers), but rather why languages develop specialized constructions for different discourse functions in the first place \parentext{see \textcite{Gil2012} for an attempt to answer this question for predication}. Lexical flexibility exists in any area of the grammar where specialized discourse-function–indicating morphology has yet to develop, or where such distinctions have been leveled as a result of diachronic changes. Flexibility should therefore be considered the default state of affairs for language. Gil \parencites*{Gil2005}{Gil2006} has in fact argued, partially on the basis of data from highly flexible languages, that early human language must have been \dfn{isolating-monocategorial-associational} before the development of dedicated function-indicating morphology.

The idea that the \enquote{natural state} of language is monocategorial or acategorial would seem to conflict with the point made above that lexical flexibility can result from diachronic processes, but the two positions are not mutually exclusive. Languages develop strategies for indicating different discourse functions, but languages are also subject to counteracting pressures. This is a classic case of competing motivations: on the one hand, the frequency with which speakers need to perform the discourse functions of reference, predication, and modification all but ensures the development of strategies dedicated to indicating those functions; on the other hand, speakers need to construe states of affairs in various ways—as objects, actions, or properties—creating pressures which have the potential to level or cut across those formally marked distinctions. Reconventionalization and the reanalysis of cleft constructions could both be viewed as diachronic processes motivated by this latter pressure.

In sum, then, lexical flexibility is a natural result of the cognitive and diachronic forces at work in language. Defining lexical flexibility in terms of typological markedness (or more accurately, the lack of formal marking for otherwise marked uses) provides a crosslinguistically applicable definition of the phenomenon which avoids methodological opportunism while still recognizing that lexical flexibility requires some degree of semantic shift and conventionalization. This cognitive-typological definition of lexical flexibility is a key theoretical contribution of this dissertation. With this definition in place, the remainder of this dissertation turns to exploring the prevalence of lexical flexibility in \idx{English} and \idx{Nuuchahnulth}.
