\clearpage
\phantomsection
\section*{Conventions}
\label{sec:conventions}
\addcontentsline{toc}{section}{Conventions}

This note documents the conventions I have adopted regarding linguistic data, terminology, and presentation of data throughout this thesis.

\subsection*{Interlinear Examples}

% (non-)standardization of grammatical terminology
It is well known that the world's languages realize widely different sets of morphosyntactic categories \parencites[58]{Whaley1997}{Haspelmath2007}. Moreover, even when these categories bear the same name, they may differ drastically in their behavior \parencite[9]{Dixon2010}. It is the subject of much debate whether these language-specific categories can be mapped onto each other or compared in any useful way \parencites{Croft1995}[10--15]{Song2001}[13--19]{Croft2003}{Haspelmath2010a}{Haspelmath2010c}{Newmeyer2010}{Stassen2011}[308--310]{Hieber2013}{Croft2014}{Plank2016}[44--58]{Song2018}. Recognizing these difficulties, I have made no attempt to standardize the linguistic terminology used in examples from different languages. I have, however, standardized the abbreviations used to refer to those terms. For example, even though one researcher may abbreviate Subject as \gl{subj} and another researcher abbreviate it as \textsc{sub}, I nonetheless gloss all Subject morphemes as \gl{subj}. See the \hyperref[ch:abbreviations]{List of Abbreviations} for a complete list of glossing abbreviations.

% orthography
I have not attempted to standardize the transcription systems and orthographies used in examples. All examples are given as transcribed in their original source. The reader should consult those original sources for further details regarding orthography.

% interlinear glossed examples
In all interlinear glossed examples, I follow the formatting conventions (but not necessarily the recommended abbreviations) of the Leipzig Glossing Rules \parencite{BickelComrieHaspelmath2015}. The source of each example is always provided after the example itself.

\subsection*{Prose}

% capitalization
It is increasingly common in typological studies to write language-particular terms and categories with an initial capital letter, and to write terms that refer to language-general or semantic/functional concepts (e.g. the crosslinguistic notion of subject) in lowercase \parencites[10]{Comrie1976}[47 (fn. 3), 141]{Bybee1985}[66]{Croft2000}[674]{Haspelmath2010a}[535]{Croft2014}. For example, the \idx{English} Participle suffix \txn{-ing} is, obviously, specific to English, and does not exist in any other language; therefore it is capitalized and written as \txn{Participle}. If, however, a writer is discussing the category of participles generally and crosslinguistically, not specific to any particular language, the term is written in lowercase as \txn{participle}. I follow these same capitalization conventions in this thesis.

\subsection*{Quotations}

% emphasis in quotes
Within quotations, \emph{italics} indicate emphasis in the original, while \qem{boldface} indicates my emphasis.
