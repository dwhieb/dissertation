\clearpage
\phantomsection
\section*{Abstract}
\label{sec:abstract}
\addcontentsline{toc}{section}{Abstract}

\begin{center}

  \doctitle

  by

  \theauthor

\end{center}

This dissertation is a quantitative corpus-based study of lexical flexibility in English (Indo-European) and Nuuchahnulth (Wakashan). \dfn{Lexical flexibility} is the capacity of lexical items to serve in more than one discourse function—reference, predication, or modification (traditionally noun, verb, or adjective).

In this dissertation I develop a procedure and metric for quantifying the lexical flexibility of words in a corpus and apply that metric to English and Nuuchahnulth by analyzing the discourse function of nearly 400,000 tokens of English and nearly 9,000 tokens of Nuuchahnulth. I find that the two languages differ drastically in not only their degree of lexical flexibility, but the way in which that flexibility is realized. This study advances the discussion of lexical flexibility---as well as parts of speech more generally---by adding a new kind of empirical evidence to the discussion (quantitative corpus-based data), and in doing so provides answers to several longstanding and much-debated questions about how lexical categories operate in English and Nuuchahnulth.
