\section*{Abstract}
\label{sec:abstract}
\addcontentsline{toc}{section}{Abstract}

\begin{center}

  \doctitle

  by

  \theauthor

\end{center}

1-4 sentences summarizing and foreshadowing all the research in your dissertation.

- concise
- compelling

Components:

1. issues / research questions
1. variables / constructs
1. sample / sources / objects studied
1. theory
1. methdology

Questions:

* What is your dissertation about?

  Lexical flexibility (in discourse)

* Why are you conducting this dissertation project?

  Nobody has ever done a corpus-based examination of the degree of lexical flexibility.

* Why should anybody care about your subject?

  This is the first time someone has used quantitative, corpus-based data to explore lexical flexibility. It provides new empirical findings which may require adjusting our theories of lexical flexibility as well as lexical categories and linguistic categorization more generally.

* What is the context that makes it important for you to pursue this topic?

  This is a foundational issue in our understanding of lexical categories. Our theories of lexical categories need to be based on sound empirical data.

* Which theories or methodologies will you use to research your topic?

  This study is meant to be framework-neutral (in the sense of Haspelmath). Its findings should be interpretable and challenging to a range of approaches to lexical categories. Those who view flexibility as a matter of unmarked conversion or zero derivation should find the empirical results of this study equally as interesting as those who would interpret these data as evidence of flexible lexemes. The explanandum is simply different in the two cases. In a conversion framework, the question is why words and languages vary so much in their conversion behavior. In a zero-derivation framework, the question is whether zero-derived cases behave similarly to cases of marked derivation.

  At the same time, I acknowledge that my own approach is unquestioningly cognitive-functional. I view linguistic categories not as rigidly-defined abstract grammatical notions, but rather as emergent, prototypal cognitive associations between words, with varying degrees of strength depending on the similarity of any given set of words. (cite cognitive literature)

* What are your data and sources?

  The data for this study consist of corpora of two languages---English and Nuuchahnulth. For English, I used the spoken portion of the Open American National Corpus (cite, also link), totaling 1.2 million words. For Nuuchahnulth, I used the collection of 24 texts elicited by Toshihide Nakayama and published in \addcite{Toshi's texts}, totaling 8,300 words.

* What will the contribution or implications of your dissertation be?

  The primary finding of this dissertation is that individual words, as well as entire languages, can and do differ drastically in terms of how lexemes behave with regard to their flexibility. The empirical data show that Nuuchahnulth is indeed a highly flexible language, as has been claimed, but that the nature of this flexibility is somewhat different than has been previously proposed (i.e. it shows noun-verb flexibility but not flexibility in the modification direction). The data also finally provide an answer to just how flexible English is. English has been claimed to be varying flexible or rigid by different authors \addcite{English as flexible vs. rigid}. The data in this study show that English does indeed show a degree of flexibility, but somewhat marginally. Most words of English exhibit some degree of flexibility, but only to a small degree. To ignore either this flexibility or its small degree would be to mischaracterize the nature of lexical categories in English. The overall pattern of flexibility in English is simply what it is.

  Taken together, this evidence suggests that strictly categorical approaches to lexical categories are too simplistic. Lexical categories are prototypal, cognitive categories.

\todo[
  caption = {\href{https://trello.com/c/DBE4nD65/26-abstract}{add Abstract}},
  inline,
  size = normalsize
]{The abstract should include 1) a brief statement of the problem; 2) a description of the methods and procedures used to gather data or study the problem; 3) a condensed summary of the findings. The abstract should be double-spaced. The recommended length is 1--2 pages. (\href{https://trello.com/c/DBE4nD65/26-abstract}{add Abstract})}
