\clearpage
\phantomsection
\section*{Abstract}
\label{sec:abstract}
\addcontentsline{toc}{section}{Abstract}

\begin{center}

  \doctitle

  by

  \theauthor

\end{center}

This dissertation is a qualitative corpus-based study of lexical flexibility in English (Indo-European) and Nuuchahnulth (Wakashan). \dfn{Lexical flexibility} is the capacity of lexical items to serve more than one discourse function—reference, predication, or modification (traditionally noun, verb, or adjective) with no overt derivational morphology.

Flexible words pose a problem for many theories of parts of speech because they cross-cut traditional part-of-speech boundaries, resisting clear classification. In response to this problem, many researchers have proposed new part-of-speech schemes with greater or lesser numbers of lexical categories. More recently, however, many researchers have come to treat lexical flexibility as an object of study in its own right. However, our understanding of how lexical flexibility operates, how it emerges diachronically, how prevalent it is, and how much it varies across the world's languages, is still nascent.

This study contributes new empirical data to the study of lexical flexibility. I analyze approximately 400,000 tokens of English and 9,000 tokens of Nuuchahnulth for their discourse function (reference, predication, or modification) in order to determine the overall prevalence of lexical flexibility in each language. I present a metric for quantitatively measuring lexical flexibility of each stem in a corpus that can be applied consistently across lexemes and languages for crosslinguistic comparison. I then apply this technique to English and Nuuchahnulth.

The data suggest that English and Nuuchahnulth differ significantly not just in their overall degree of flexibility, but also in the way that flexibility is realized. Most English stems exhibit lexical flexibility to a small degree, but otherwise center around a clear prototype. By contrast, most Nuuchahnulth stems exhibit a high degree of lexical flexibility, but primarily between reference and predication. Nuuchahnulth stems show very few uses of modification in discourse. I also show that the degree of flexibility for each lexical item is synchronically fixed, suggesting that lexemes have a conventionalized set of discourse uses rather than productively appearing in whatever context is appropriate. I also investigate the relationship between lexical flexibility and either relative frequency or corpus dispersion, but find no clear correlations.

In both English and Nuuchahnulth, human animates are consistently low in flexibility, in line with the status of human animates as prototypical referents in discourse crosslinguistically. English and Nuuchahnulth display opposite tendencies for property words, however. In English, property words are among the low-flexibility items, whereas in Nuuchahnulth property words are consistently among the highest-flexibility items. I suggest that this difference is due to a lack of dedicated morphological modifying constructions in Nuuchahnulth.

The findings in this dissertation present a strong case for reversing the traditional perspective on lexical flexibility: rather than treating lexical flexibility as a relatively exceptional problem to be solved, I argue that lexical flexibility is a central and prevalent feature of the world's languages. Lexical flexibility exists anywhere a language has yet to develop dedicated morphosyntactic constructions for distinct discourse functions, or where those constructions have been diachronically leveled.
