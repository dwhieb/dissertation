\section*{Abstract}
\label{sec:abstract}
\addcontentsline{toc}{section}{Abstract}

\begin{center}

  \doctitle

  by

  \theauthor

\end{center}

\subsection*{Focus Statement}

Do some languages have more flexible parts of speech than others? Can the lexical flexibility of words and languages be quantified and captured empirically? This dissertation is a quantitative corpus-based study of lexical flexibility in English\index{English} (Indo-European) and Nuuchahnulth\index{Nuuchahnulth} (Wakashan). I develop a quantitative metric for determining the lexical flexibility of words in a corpus, and apply that metric to English and Nuuchahnulth. I find that the two languages differ drastically in not only their degree of lexical flexibility, but the way in which that flexibility is realized. This study advances the discussion of lexical flexibility---as well as parts of speech more generally---by adding a new kind of empirical evidence to the discussion (quantitative corpus-based data), thereby providing answers to  several longstanding and much-debated questions about how lexical categories operate in English and Nuuchahnulth.

\todo[
  caption = {\href{https://trello.com/c/DBE4nD65/26-abstract}{add Abstract}},
  inline,
  size = normalsize
]{The abstract should include 1) a brief statement of the problem; 2) a description of the methods and procedures used to gather data or study the problem; 3) a condensed summary of the findings. The abstract should be double-spaced. The recommended length is 1--2 pages. (\href{https://trello.com/c/DBE4nD65/26-abstract}{add Abstract})}
