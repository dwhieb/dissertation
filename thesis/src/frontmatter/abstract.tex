\clearpage
\phantomsection
\section*{Abstract\markboth{Abstract}{}}
\label{sec:abstract}
\addcontentsline{toc}{section}{Abstract}
\thispagestyle{plain}

\begin{center}

  \doctitle

  by

  \theauthor

\end{center}

This dissertation is a qualitative corpus-based study of lexical polyfunctionality (also known as lexical flexibility or polycategoriality) in English (Indo-European) and Nuuchahnulth (Wakashan). Polyfunctional lexical items are those which appear in more than one discourse function—reference, predication, or modification (traditionally noun, verb, or adjective)—with zero coding for that function (often referred to as \dfn{conversion} or \dfn{zero derivation}).

Polyfunctional words pose a problem for many theories of parts of speech because they cross-cut traditional part-of-speech boundaries, resisting clear classification. In response to this problem, many researchers have proposed new part-of-speech schemes with a greater or fewer number of lexical categories. More recently, however, many researchers have come to treat lexical polyfunctionality as an object of study in its own right. However, our understanding of how polyfunctionality operates, how it emerges diachronically, how prevalent it is, and how much it varies across the world's languages, is still nascent.

This study contributes new empirical data to the study of lexical polyfunctionality. I analyze approximately 380,000 tokens of English and 8,300 tokens of Nuuchahnulth for their discourse function (reference, predication, or modification) in order to determine the overall prevalence of lexical polyfunctionality in each language. I present a metric for quantitatively measuring the functional diversity of each stem in a corpus which can be applied consistently across lexemes and languages for crosslinguistic comparison. I then apply this technique to English and Nuuchahnulth.

The data suggest that English and Nuuchahnulth differ significantly not just in their overall functional diversity / degree of polyfunctionality, but also in the way that polyfunctionality is realized. Most English stems exhibit lexical polyfunctionality to a small degree, but otherwise center around a clear prototype. By contrast, most Nuuchahnulth stems exhibit a high degree of lexical polyfunctionality, but primarily between reference and predication. Nuuchahnulth stems show very few uses of modification in discourse. I also show that the functional diversity for each lexical item is synchronically fixed, suggesting that lexemes have a conventionalized set of discourse uses rather than productively appearing in whatever context is appropriate. I also investigate the relationship between lexical polyfunctionality, relative frequency, and corpus dispersion, but find no clear correlations.

In both English and Nuuchahnulth, human animates are consistently low in functional diversity, in line with the status of human animates as prototypical referents in discourse crosslinguistically. English and Nuuchahnulth display opposite tendencies for property words, however. In English, property words are among the least polyfunctional items, whereas in Nuuchahnulth quantifiers and property words are consistently among the most polyfunctional items. I suggest that this difference is due to a lack of a dedicated morphological strategy for indicating modification in Nuuchahnulth.

The findings in this dissertation present a strong case for reversing the traditional perspective on lexical polyfunctionality: rather than treating lexical polyfunctionality as a relatively exceptional problem to be solved, I argue that lexical polyfunctionality is a central and prevalent feature of the world's languages. Lexical polyfunctionality exists anywhere a language has yet to develop dedicated morphological strategies for distinct discourse functions, or where those constructions have been diachronically leveled.
