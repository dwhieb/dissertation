\chapter*{List of Languages}
\label{ch:languages}
\addcontentsline{toc}{section}{List of Languages}

The following table provides information about each language mentioned in this thesis: the name of the language in English, the \href{https://iso639-3.sil.org/}{International Standards Organization (ISO) 693-3 language code}, and the \href{https://glottolog.org/}{Glottolog} code \parencite{HammarstromForkelHaspelmath2019}. Genealogical information follows the format \texttt{family > phylum}.

\renewcommand{\arraystretch}{1}

\begin{table}[h]
  \onehalfspacing
  \begin{tabularx}{\linewidth}{ l l l l l }
    \textbf{Language Name (English)} & \textbf{ISO 639-3} & \textbf{Glottocode} & \textbf{Genetic Affiliation}\\
    \midrule
    Ancient Greek                    & \texttt{grc}       & \texttt{anci1242}   & Indo-European > Hellenic\\
    Cayuga                           & \texttt{cay}       & \texttt{cayu1261}   & Iroquoian > Northern Iroquoian\\
    Central Alaskan Yup'ik           & \texttt{esu}       & \texttt{cent2127}   & Eskimo-Aleut > Yupik\\
    Chamorro                         & \texttt{cha}       & \texttt{cham1312}   & Austronesian > Malayo-Polynesian\\
    Classical Nahuatl                & \texttt{nci}       & \texttt{clas1250}   & Uto-Aztecan > Nahuan\\
    English                          & \texttt{eng}       & \texttt{stan1293}   & Indo-European > Germanic\\
    Latin                            & \texttt{lat}       & \texttt{lati1261}   & Indo-European > Italic\\
    Kutenai                          & \texttt{kut}       & \texttt{kute1249}   & isolate\\
    Mandarin                         & \texttt{cmn}       & \texttt{mand1415}   & Sino-Tibetan > Sinitic\\
    Mandinka                         & \texttt{mnk}       & \texttt{mand1436}   & Mande > Manding\\
    Mundari                          & \texttt{unr}       & \texttt{mund1320}   & Austroasiatic > Munda\\
    North Efate                      & \texttt{llp}       & \texttt{nort2836}   & Austronesian > Oceanic\\
    Nuuchahnulth                     & \texttt{nuk}       & \texttt{nuuc1236}   & Wakashan > Southern Wakashan\\
    Quechua                          & \texttt{qwe}       & \texttt{quec1387}   & Quechuan\\
    Quiché Maya                      & \texttt{quc}       & \texttt{kich1262}   & Mayan > Quichean\\
    Riau Indonesian                  & \texttt{ind}       & \texttt{indo1316}   & Austronesian > Malayan\\
    Russian                          & \texttt{rus}       & \texttt{russ1263}   & Indo-European > Balto-Slavic\\
    Sundanese                        & \texttt{sun}       & \texttt{sund1251}   & Austronesian > Malayo-Polynesian\\
    Tagalog                          & \texttt{tgl}       & \texttt{taga1280}   & Austronesian > Philippine\\
    Tongan                           & \texttt{ton}       & \texttt{tong1325}   & Austronesian > Polynesian\\
    Ute                              & \texttt{ute}       & \texttt{utee1244}   & Austronesian > Polynesian\\
  \end{tabularx}
\end{table}
