\chapter*{A Note on Linguistic Conventions}
\label{ch:conventions}
\addcontentsline{toc}{section}{A Note on Linguistic Conventions}

This brief note documents the conventions I have adopted regarding linguistic data and terminology throughout this thesis.

It is well known that the world's languages realize widely different sets of morphosyntactic categories \citeps[58]{Whaley1997}{Haspelmath2007}. Moreover, even when these categories bear the same name, they may differ drastically in their behavior \citep[9]{Dixon2010}. It is the subject of much debate whether these language-specific categories can be mapped onto each other or compared in any useful way \citeps[13--19]{Croft2003}{Haspelmath2010a}{Haspelmath2010b}{Newmeyer2010}[308--310]{Hieber2013}{Croft2014}{Plank2016}. Recognizing these difficulties, I have made no attempt to standardize the linguistic terminology in the interlinear glossed examples throughout this thesis. I have, however, standardized the abbreviations used to refer to those terms. For example, even though one researcher may abbreviate Subject as \gl{subj} and another researcher abbreviate it as \textsc{sub}, I nonetheless gloss all Subject morphemes in this thesis as \gl{subj}. See the \hyperref[ch:abbreviations]{List of Abbreviations} (p.~\pageref{ch:abbreviations}) for a complete list of glossing abbreviations. I have also followed the standard formatting conventions of the Leipzig Glossing Rules \citep{BickelComrieHaspelmath2015} throughout.

In contrast, I have made no attempt to standardize the transcription systems and orthographies used in the examples throughout this thesis. All examples are given as transcribed in their original source. The reader should consult those original sources for further details regarding orthography. The source of each interlinear glossed example in this thesis is provided following the example itself.

It is an increasingly common convention in typological studies to write terms and categories that are particular to specific languages with an initial capital letter, while writing terms that refer to language-general or semantic/functional concepts (e.g. the crosslinguistic notion of subject) in lowercase \citeps[674]{Haspelmath2010a}[535]{Croft2014}. For example, the English Participle suffix \tx{-ing} is, obviously, specific to English, and does not exist in any other language; therefore it capitalized and written as \tx{Participle}. If, however, a writer is discussing the category of participles generally and crosslinguistically, not specific to any particular languages, the term is written in lowercase as \tx{participle}. I follow these same capitalization conventions in this thesis.
