\chapter*{A Note on Linguistic Conventions}
\addcontentsline{toc}{section}{A Note on Linguistic Conventions}

It is well known that the world's languages realize widely different sets of morphosyntactic categories \parencites[58]{Whaley1997}{Haspelmath2007}. Moreover, even when these categories bear the same name, they may differ drastically in their behavior \parencite[9]{Dixon2010}. It is the subject of much debate whether these language-specific categories can be mapped onto each other or compared in any useful way \parencites[13--19]{Croft2003}{Haspelmath2010a}{Haspelmath2010b}{Newmeyer2010}[308--310]{Hieber2013}{Croft2014}{Plank2016}. Recognizing these difficulties, I have made no attempt to standardize the linguistic terminology in the interlinear glossed examples throughout this thesis. I have, however, standardized the abbreviations used to refer to those terms. For example, even though one researcher may abbreviate Subject as \gl{subj} and another researcher abbreviate it as \textsc{sub}, I nonetheless gloss all Subject morphemes in this thesis as \gl{subj}.

It has become an increasingly common convention in typological studies to label language\hyp{}specific constructions with initial capital letters (e.g. the English Subject construction), while terms that refer to language-general or semantic/functional concepts (e.g. the crosslinguistic notion of subject) are given in lowercase \parencites[674]{Haspelmath2010a}[535]{Croft2014}. I also follow this capitalization convention in this thesis.
