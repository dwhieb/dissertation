\chapter{Glossing Conventions}

It is well known that the world's languages realize widely different sets of morphosyntactic categories \addref{typological introductions, maybe Dixon's BLT}. Moreover, even when these categories bear the same name, they may differ drastically in their behavior \addref{Haspelmath? Dixon? Others?}. It is the subject of much debate whether these language-specific categories can be mapped onto each other or compared in any useful way \addref{Haspelmath 2010, other publications on this debate}. Recognizing these difficulties, I have made no attempt to standardize the linguistic terminology in the interlinear glossed examples throughout this thesis. I have, however, standardized the abbreviations used to refer to those terms. For example, even though one researcher may abbreviate Subject as \gl{subj} and another researcher abbreviate it as \textsc{sub}, I nonetheless gloss all Subject morphemes in this thesis as \gl{subj}.

\section{Glossing Abbreviations}

The following table provides the meaning of each abbreviation used in interlinear glossed examples throughout this thesis.

\newpage

\begin{multicols}{2}

  \begin{tabular}{ p{5em} l }
    \gl{subj} & subject\\
  \end{tabular}

\end{multicols}
